\label{app:entrevista-policia}
\begin{longtable}{|l|p{6cm}|p{6cm}|}
    \caption{Entrevista aplicada al teniente coronel de estado mayor Christian Iván Quintana Guerra} \label{tab:entrevista-policia'tab}                                                                                                                                                                                                                                                                                                                                                                                                                                                                                                                                                 \\

    \hline \multicolumn{1}{|c|}{\textbf{N. Pregunta}} & \multicolumn{1}{|c|}{\textbf{Respuesta}}                                                                                                                                                                                                                                                                                                                                                         & \multicolumn{1}{|c|}{\textbf{Observación}}                                                                                                                                                                                   \\ \hline
    \endfirsthead

    \multicolumn{3}{c}%
    {{\normalfont \tablename\ \thetable{} -- continuación de la página anterior}}                                                                                                                                                                                                                                                                                                                                                                                                                                                                                                                                                                                                       \\
    \hline \multicolumn{1}{|c|}{\textbf{N. Pregunta}} & \multicolumn{1}{|c|}{\textbf{Respuesta}}                                                                                                                                                                                                                                                                                                                                                         & \multicolumn{1}{|c|}{\textbf{Observación}}                                                                                                                                                                                   \\ \hline
    \endhead

    \hline \multicolumn{3}{|r|}{{Continua en la siguiente página}}                                                                                                                                                                                                                                                                                                                                                                                                                                                                                                                                                                                                                      \\ \hline
    \endfoot


    \hline \multicolumn{3}{|p{13cm}|}{{
            \textbf{Conclusion:} La entrevista resalta la importancia crítica de una sólida estructura de datos y un enfoque meticuloso en la
            interpretación y presentación de la información en el ámbito de Business Intelligence (BI). Desde la definición de KPI relevantes hasta la selección
            adecuada de herramientas de análisis y visualización, cada aspecto destaca la necesidad de una estrategia integral que permita convertir los datos en
            acciones concretas. Tambien se señala que una efectiva estrategia de BI requiere no solo tecnología avanzada, sino también una comprensión
            profunda de las necesidades del usuario y una atención constante a la calidad y confiabilidad de los datos.
    }}                                                                                                                                                                                                                                                                                                                                                                                                                                                                                                                                                                                                                                                                                  \\ \hline
    \endlastfoot

    1                                                 & Todos los eventos delictivos que sucedan en el territorio provienen de una demanda ciudadana. Con base en la información recibida del demandante, se recopilan datos georreferenciados de los eventos delictivos, así como horas, fechas, los tipos de personas involucradas en el evento y el modus operandi, lo cual permite realizar los análisis respectivos y tomar las acciones adecuadas. & La precisión en la recopilación de datos es vital para la gestión de seguridad pública, especialmente a través de la georreferenciación y la categorización detallada de eventos.                                            \\
    2                                                 & La calidad aproximada se estima en alrededor del 70\%, tomando en cuenta que los ciudadanos en varios casos confunden los diferentes delitos, siendo el caso más común la confusión entre robo y hurto, lo que resulta en reportes inconsistentes.                                                                                                                                               & La calidad de los datos sobre delitos se ve afectada por la inconsistencia en los reportes ciudadanos                                                                                                                        \\
    3                                                 & Uno de los desafíos que se encuentra comúnmente es que los demandantes proporcionen los datos necesarios para realizar un trabajo más efectivo.                                                                                                                                                                                                                                                  & La obtención de datos completos y precisos de los demandantes representa un desafío significativo para la policía en la recopilación y análisis de datos.                                                                    \\
    4                                                 & Los datos son recopilados de forma continua en una matriz de Excel.                                                                                                                                                                                                                                                                                                                              & Los datos sobre delitos se recopilan en una matriz de Excel de forma continua, sugiriendo una falta de sistemas más avanzados de gestión de datos.                                                                           \\
    5                                                 & El sistema debería proporcionar una información completa sobre un evento delictivo específico con la cual se pueda identificar datos de los agresores, tales como modus operandi, nacionalidad, etnia, altura y contextura. Esto permitiría identificar posibles sospechosos de estar involucrados en algún delito.                                                                              & Existe una necesidad de sistemas de análisis de delitos que proporcionen información detallada sobre los agresores para facilitar la identificación de sospechosos.                                                          \\
    6                                                 & Contar con las protecciones necesarias en las bases de datos.                                                                                                                                                                                                                                                                                                                                    & Se reconoce la importancia de medidas sólidas de seguridad y privacidad de datos, aunque no se especifican detalladamente.                                                                                                   \\
    7                                                 & Esto facilitaría el poder compartir información entre las diferentes unidades con la finalidad de realizar un trabajo más eficiente.                                                                                                                                                                                                                                                             & Los sistemas de análisis de delitos pueden mejorar la colaboración entre unidades policiales al facilitar el intercambio eficiente de información.                                                                           \\
    8                                                 & El nivel de capacitación debe ser alto y constante, dado que uno de los mayores problemas que se enfrenta actualmente es que las promociones más recientes de los miembros de la policía cuentan con conocimientos limitados en cuanto a las nuevas tecnologías.                                                                                                                                 & Se destaca la necesidad de una capacitación constante para el personal policial en el uso efectivo de sistemas de análisis de delitos, especialmente para abordar la brecha tecnológica entre las promociones más recientes. \\
\end{longtable}
