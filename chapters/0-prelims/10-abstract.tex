\newpage
\chapter*{abstract}
\addcontentsline{toc}{chapter}{\bfseries \uppercase{ABSTRACT}}
The city of Ambato faces significant challenges in the collection and analysis of crime and casualty data,
which negatively impacts public safety decision-making. Traditional methods, based on manual records and
fragmented systems, are inefficient and error-prone, affecting accuracy and timeliness in tracking crime trends.
\bigbreak

This research project proposes the development of a system that combines a mobile application and a web platform,
with the objective of optimizing the collection and analysis of geo-referenced crime data and creating a Business
Intelligence (BI) analytical model. The system seeks to improve operational efficiency, reduce human error and provide
a tool that facilitates administration and access to information for both authorities and citizens.
\bigbreak

The project was developed using the XP (eXtreme Programming) agile development methodology, which allowed for greater flexibility
and adaptability to changes during the development process. This approach facilitated rapid iterations and the incorporation
of continuous feedback, ensuring that the final product met requirements as system development progressed.
\bigbreak

The project is divided into two main applications: web and mobile. Flutter was used for the development of the mobile
application, due to its cross-platform performance. For the web application, NextJS and NestJS were used, technologies
that allow the creation of robust and scalable web applications. Additionally, a PostgreSQL geospatial database was
used to efficiently manage large volumes of georeferenced data and develop a BI analytical model to visualize and
analyze crime trends.
\vfill
\textbf{Keywords:} Rappid Application Development, web system, mobile application, Business Intelligence,
geo-referencing, public safety, Flutter, NextJS, NestJS
