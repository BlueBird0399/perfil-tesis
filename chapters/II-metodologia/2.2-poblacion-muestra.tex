\subsection{Población y muestra}
La población de estudio está conformada por los incidentes delictivos ocurridos en la ciudad de Ambato. Al no contar con una estimación
adecuada sobre la cantidad de incidentes requerida para estipular la población, se optó por definir una muestra infinita con población desconocida.
\begin{itemize}
    \item n = tamaño de la muestra
    \item Z = nivel de confianza, un nivel de confianza habitual es de 90\% con una puntuación estándar de 1.645
    \item p = probabilidad de éxito o proporción esperada
    \item q = probabilidad de fracaso
    \item pq = varianza de la población, cada una equivale a 0.50, (p=q=0.50)
    \item e = error de estimación máximo aceptable, el error habitual es de 5\% con una puntuación estándar de 0.05
\end{itemize}
\[
    n=\frac{Z^2 \cdot p \cdot q}{e^2}
\]\\
\[
    n=\frac{(1.645)^2 \cdot 0.50 \cdot 0.50}{(0.05)^2}
\]\\
\[
    n=270.60 \approx 270
\]
Con un nivel de confianza del 90\% y un error de estimación máximo aceptable del 5\%, se obtiene una muestra aproximada de 270 incidentes. \\ \\