\section{Conclusiones}
\begin{itemize}
      \item En base a la entrevista realizada al coronel de la policía, se logró identificar el proceso actual que se lleva
            a cabo al momento de reportar un siniestro, lo cual permitió determinar las necesidades y requerimientos necesarios
            para la implementación del sistema de obtención de datos georreferenciados. La información obtenida a través de la
            entrevista realizada al ingeniero experto en BI permitió identificar las características y funcionalidades necesarias
            que deben tener el modelo analítico de BI y el sistema de reportería, así como los medios y formatos correctos para
            visualizar la información obtenida.

      \item La aplicación de frameworks de desarrollo demuestran su uso como una herramienta eficaz la cual permite
            acelerar el proceso creación de software. Estos ofrecen un marco de trabajo que facilita la implementación de
            aplicaciones. En el caso de este proyecto se utilizaron tecnologías basadas en JavaScript y Dart, las cuales
            cuales cuentan con una amplia comunidad de desarrolladores y una documentación extensa los cuales pueden
            ofrecer soporte en caso de errores o dudas. Sin embargo, la elección de un framework debe ser cuidadosa, ya que
            que este debe adaptarse a las necesidades del proyecto y las capacidades del equipo de desarrollo.

      \item La precisión de los datos georreferenciados es un factor importante en la obtención de información precisa
            para el análisis de datos. Con la finalidad de evaluar la precisión de los datos obtenidos, se emplearon técnicas
            estadísticas para determinar la fiabilidad de la información. Mediante el empleo de un intervalo de confianza
            se pudo determinar que en un 95\% de los casos, la precisión de los datos georreferenciados se encuentra en un
            rango entre 7.08 y 8.02 metros con respecto a la ubicación real del incidente.

      \item La implementación de un modelo analítico de BI, empleando la metodología Hefesto para la construcción de un Data
            Warehouse, ha demostrado ser una estrategia efectiva para optimizar el rendimiento en la gestión de datos. El
            diseño del proceso ETL, llevado a cabo con la herramienta de Visual Studio, ha permitido la consolidación
            eficiente de datos provenientes de diversas fuentes, asegurando su compatibilidad y adecuada inserción en el
            modelo dimensional. La creación del cubo OLAP y su visualización mediante Power BI han facilitado un análisis
            detallado de la tendencia de siniestros.

      \item La metodología de desarrollo ágil XP permitió un desarrollo rápido y eficiente, gracias a su enfoque iterativo
            y a la participación activa de los usuarios, lo cual facilitó la adaptación de los requerimientos a medida que se
            avanzaba en el desarrollo del sistema.

\end{itemize}