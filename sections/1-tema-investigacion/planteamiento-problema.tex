\subsection{Planteamiento del problema}
En \cite{rummensMobilePhoneData2021}, desde el 2020 se ha notado un significativo incremento en la cantidad de
campos de estudio que llevan a cabo el proceso de toma de decisiones basadas en la información extraída de grandes
conjuntos de datos. Según \cite{ahishakiyePerformanceAnalysisBusiness2017}, para asegurar que el proceso de toma
de decisiones conduzca a los resultados óptimos, es fundamental el uso de Business Intelligence (BI), el cual
emplea un conjunto de herramientas que facilitan la adquisición, almacenamiento, análisis y entrega de la información
necesaria.

\bigbreak
En \cite{reidDevelopingModelPerceptions2020} se menciona que en el ámbito social, el incremento de la delincuencia,
el miedo y la inseguridad ciudadana respecto a la zona en la que residen son temas que afectan directamente la calidad
de vida de las personas. En este contexto, \cite{rummensMobilePhoneData2021} señala que el análisis de datos sobre la
delincuencia ha empezado a tomar una mayor relevancia, siendo estos datos usados de forma objetiva para fundamentar
políticas, así como estrategias policiales y operaciones tácticas con el fin de reducir y prevenir la delincuencia
mediante la toma de decisiones oportunas y precisas.

\bigbreak
La carencia de fuentes adecuadas de datos georreferenciados, con los cuales realizar análisis, dificulta la capacidad de
la ciudadanía y las fuerzas del orden para tomar precauciones necesarias \cite{tasnimNovelMultiModuleApproach2022}. Según
\cite{vieiraCrimePredictionPrevention2022}, los datos georreferenciados o espacio-temporales son un valioso recurso utilizado
por analistas criminalistas. Estos datos son la base para la identificación de patrones delictivos en áreas de alta concentración
de delitos a través del mapeo de zonas críticas o puntos calientes (hotspots). El autor en \cite{rummensMobilePhoneData2021}
menciona que el uso de software permite la adquisición de datos delictivos de forma más precisa y eficiente, obteniendo información
útil que permita identificar los puntos focales del crimen.

\bigbreak
% georreferenciados para el control de la delincuencia \cite{ariasCrimePunishmentLatin2023}.
En el estudio de \cite{unitednationsReferenceFrameworkSecurity2022}, según la Comisión Económica para América Latina y el Caribe (CEPAL),
en América Latina y el Caribe, la georreferenciación se posiciona como la clave para realizar análisis más profundos y detallados sobre la
delincuencia. Esta técnica además facilita la visualización de la distribución espacio-temporal de variables relacionadas con el crimen,
permitiendo su análisis posterior. En ese sentido, el autor de \cite{ariasCrimePunishmentLatin2023} menciona que desde 1990 se han establecido
paradigmas de entes policiales orientados a la comunidad, los cuales emplean estrategias contra la delincuencia haciendo uso de datos
georreferenciados.

\bigbreak
Según \cite{tayupantaBOLETINSEMESTRALHOMICIDIOS2023}, el boletín semestral de homicidios intencionados en Ecuador, presentado por la
Fundación Panamericana para el Desarrollo (PADF), menciona que en 2022, el país registró un total de 4603 homicidios intencionales,
marcando así el registro histórico más alto hasta el momento. Se observó un promedio diario de 10 casos, junto a una tasa de 25.9
homicidios por cada 100 000 habitantes. Sin embargo, en el primer semestre de 2023, los casos de homicidios intencionados registrados
descendieron a 3599, lo que equivale a un promedio diario de 19 casos, casi el doble de los registros obtenidos en 2022.
Debido a esta tendencia, se prevé que la cifra total de homicidios supere los 7000 casos, alcanzando una tasa de 35 homicidios por
cada 100 000 habitantes. Esto sitúa a Ecuador como uno de los tres países más violentos en América Latina, solo por detrás de Venezuela y
Honduras. Por lo tanto, en \cite{narvaezGESTIONINFORMACIONCRIMINOLOGICA2018} se plantea la necesidad de abordar activamente la seguridad
en beneficio de la ciudadanía ecuatoriana, proponiendo el uso de tecnologías que permitan establecer estadísticas, interfaces gráficas
y mapas con información georreferenciada.

% Desde el año 2011 el Departamento de Análisis de Información del Delito (DAID) de la policía nacional ha
% trabajado en la implementación de un sistema de información respecto a los incidentes ocurridos en el territorio nacional.


% El sistema
% David desarrollado por el DAID se encarga de generar información actualizada de crímenes la cual es usada como base para la toma de
% decisiones en el ámbito de la seguridad ciudadana. Sin embargo, este sistema no se encuentra disponible para el público en general
% por lo cual en \cite{velozSpatialPatternsCrime} se menciona que en ciudades como Quito se han realizado estudios sobre las tendencias
% y patrones delictivos empleando herramientas de análisis espacial mediante sistemas de información geográfica. En la ciudad de Riobamba
% también se han realizado estudios relacionados al análisis de los delitos, en este caso en \cite{escuderoSpatiallyCorrelatedModel2022}
% se propuso estructurar un modelo con el cual realizar un conteo de crímenes basados en la correlación geográfica y temporal de los mismos.

\bigbreak
En la ciudad de Ambato, se evidencia la carencia de sistemas que permitan establecer una fuente confiable de datos georreferenciados para
analizar exhaustivamente los delitos y sus tendencias criminales. El autor de \cite{alvarezCompendioEconomicoSocial2022} determinó, a
través de encuestas a la ciudadanía del cantón Ambato, que el 26.30\% de la población ha sido víctima de actos delictivos, situando a
Ambato en el octavo puesto a nivel cantonal en Ecuador en cuanto al número de denuncias por robos. Según \cite{quintanaAnalisisSistematicoAumento2023},
la cantidad de incidentes registrados en Ambato afecta el nivel de confianza de la ciudadanía, debido al temor que sienten de ser víctimas
de algún incidente y a la preocupación por la integridad de sus familiares.


% Siendo así que más del 92.8\% de la población ambateña se considera propenso a ser víctima de algún delito.