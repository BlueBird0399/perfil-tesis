Police
\begin{itemize}
    \item ¿Qué tipo de información o datos específicos sobre incidentes delictivos considera más críticos para la gestión efectiva de la seguridad pública?
    \item ¿Cuál es la disponibilidad y calidad de los datos actuales sobre incidentes delictivos? ¿Existen desafíos específicos en la recolección o integridad de los datos?
    \item ¿Cuáles son los principales desafíos que enfrenta actualmente el departamento de policía en la recopilación, análisis y utilización de datos relacionados con delitos?
    \item ¿Con qué frecuencia y en qué formato se recopilan actualmente los datos sobre incidentes delictivos? ¿Hay algún requisito específico en términos de temporalidad y presentación de la información?
    \item ¿Qué características o funcionalidades le gustaría ver en un sistema de análisis de incidentes delictivos diseñado para el departamento de policía en términos de visualización de datos, generación de informes, capacidad de análisis, etc.?
    \item ¿Cuál es el nivel de detalle requerido en los informes sobre incidentes delictivos? ¿Existen segmentaciones específicas que sean de particular interés?
    \item ¿Qué medidas de seguridad y privacidad de datos considera esenciales para garantizar el manejo adecuado de la información sensible relacionada con incidentes delictivos?
    \item ¿Cómo cree que un sistema de análisis de incidentes delictivos podría mejorar la colaboración y coordinación entre diferentes unidades o departamentos dentro de la policía?
    \item ¿Qué nivel de capacitación considera necesario para que el personal del departamento de policía pueda utilizar eficazmente un sistema de análisis de incidentes delictivos?
\end{itemize}

¿Qué tipo de información o datos específicos sobre incidentes delictivos considera más críticos para la gestión efectiva de la seguridad pública?
¿Cuál es la disponibilidad y calidad de los datos actuales sobre incidentes delictivos? ¿Existen desafíos específicos en la recolección o integridad de los datos?
¿Cuáles son los principales desafíos que enfrenta actualmente el departamento de policía en la recopilación, análisis y utilización de datos relacionados con delitos?
¿Con qué frecuencia y en qué formato se recopilan actualmente los datos sobre incidentes delictivos? ¿Hay algún requisito específico en términos de temporalidad y presentación de la información?
¿Qué características o funcionalidades le gustaría ver en un sistema de análisis de incidentes delictivos diseñado para el departamento de policía en términos de visualización de datos, generación de informes, capacidad de análisis, etc.?
¿Cuál es el nivel de detalle requerido en los informes sobre incidentes delictivos? ¿Existen segmentaciones específicas que sean de particular interés?
¿Qué medidas de seguridad y privacidad de datos considera esenciales para garantizar el manejo adecuado de la información sensible relacionada con incidentes delictivos?
¿Cómo cree que un sistema de análisis de incidentes delictivos podría mejorar la colaboración y coordinación entre diferentes unidades o departamentos dentro de la policía?
¿Qué nivel de capacitación considera necesario para que el personal del departamento de policía pueda utilizar eficazmente un sistema de análisis de incidentes delictivos?