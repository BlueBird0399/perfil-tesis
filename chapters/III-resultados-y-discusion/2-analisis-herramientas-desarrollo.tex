\section{Análisis de herramientas de desarrollo}
En este apartado se presenta el análisis y selección de las herramientas de desarrollo que se utilizarán para la
implementación del proyecto.

\subsection{Análisis y selección de la metodología de desarrollo}

En la Tabla \ref{tab:metodologias} se presenta una comparación entre las metodologías de desarrollo de software ágil
y tradicional. Se evalúan aspectos como el estilo de gestión, la gestión del conocimiento, el modelo de desarrollo,
la estructura organizativa y el control de calidad, entre otros. En este análisis, se destaca que la metodología ágil
es la más adecuada para el desarrollo del proyecto actual. Esta metodología permite realizar cambios en el desarrollo
del software de manera rápida y eficiente, además de adaptarse fácilmente a equipos de desarrollo pequeños y medianos,
manteniendo una planificación y control de calidad permanente en iteraciones a corto plazo. En contraposición, la
metodología tradicional requiere una planificación exhaustiva y detallada, lo cual no es apropiado para el desarrollo
de software adaptativo de alta calidad en equipos pequeños. Esto se debe a que la metodología tradicional presupone
que los sistemas sean completamente definidos y predecibles, y cualquier cambio en el desarrollo puede resultar en
un alto costo de reinicio.

\bigbreak
Por consiguiente, se optó por la metodología ágil para el desarrollo del presente proyecto.

\begin{longtable}{|p{5cm}|p{5cm}|p{5cm}|}
    \caption[Análisis y comparación entre metodologías de desarrollo de software ágil y tradicional]{Análisis y comparación entre metodologías de desarrollo de software ágil y tradicional \cite{stoicaSoftwareDevelopmentAgile2013}.} \label{tab:metodologias}                                                                                                                              \\

    \hline \multicolumn{1}{|c|}{\textbf{Criterio}} & \multicolumn{1}{|c|}{\textbf{Metodología tradicional}}                                                                             & \multicolumn{1}{|c|}{\textbf{Metodología ágil}}                                                                                                                                                     \\ \hline
    \endfirsthead

    \multicolumn{3}{c}%
    {{\normalfont \tablename\ \thetable{} -- continuación de la página anterior}}                                                                                                                                                                                                                                                                                                             \\
    \hline \multicolumn{1}{|c|}{\textbf{Criterio}} & \multicolumn{1}{|c|}{\textbf{Metodología tradicional}}                                                                             & \multicolumn{1}{|c|}{\textbf{Metodología ágil}}                                                                                                                                                     \\ \hline
    \endhead

    \hline \multicolumn{3}{|r|}{{Continua en la siguiente página}}                                                                                                                                                                                                                                                                                                                            \\ \hline
    \endfoot

    \hline \hline
    \endlastfoot
    Hipótesis fundamental                          & Los sistemas pueden ser completamente definidos, predecibles y se construyen a través de una planificación exhaustiva y detallada. & Pequeños equipos emplean el principio de mejorar constantemente el diseño y realizar pruebas basadas en una retroalimentación rápida y cambios para desarrollar software adaptativo de alta calidad \\\hline
    Estilo de gestión                              & Mando y control                                                                                                                    & Liderazgo y colaboración                                                                                                                                                                            \\\hline
    Gestión del conocimiento                       & Explicito                                                                                                                          & Tácito                                                                                                                                                                                              \\\hline
    Comunicación                                   & Formal                                                                                                                             & Informal                                                                                                                                                                                            \\\hline
    Modelo de desarrollo                           & Modelo de ciclo de vida (cascada, espiral o modelos modificados)                                                                   & Modelo de entrega evolutivo                                                                                                                                                                         \\\hline
    Estructura organizativa                        & Mecánica (burocrática, alta formalización), dirigida a grandes organizaciones                                                      & Orgánica (flexible y participativa, fomenta la cooperación social), dirigida a pequeñas y medianas organizaciones                                                                                   \\\hline
    Control de calidad                             & Planificación difícil y control estricto. Pruebas difíciles y tardías                                                              & Control permanente o requisitos, diseño y soluciones. Pruebas permanentes                                                                                                                           \\\hline
    Requisitos de los usuarios                     & Detallado y definido antes de la codificación/implantación                                                                         & Entrada interactiva                                                                                                                                                                                 \\\hline
    Coste del reinicio                             & Alto                                                                                                                               & Bajo                                                                                                                                                                                                \\\hline
    Dirección del desarrollo                       & Fijo                                                                                                                               & Fácil de cambiar                                                                                                                                                                                    \\\hline
    Pruebas                                        & Una vez finalizada la codificación                                                                                                 & Cada iteración                                                                                                                                                                                      \\\hline
    Participación del cliente                      & Baja                                                                                                                               & Alta                                                                                                                                                                                                \\\hline
    Requisitos                                     & Muy estable, conocido de antemano                                                                                                  & Emergente, con cambios rápidos                                                                                                                                                                      \\\hline
    Arquitectura                                   & Diseño para necesidades actuales y previsibles                                                                                     & Diseño para las necesidades actuales                                                                                                                                                                \\\hline
    Remodelación                                   & Caro                                                                                                                               & No es caro                                                                                                                                                                                          \\\hline
    Tamaño                                         & Grandes equipos y proyectos                                                                                                        & Pequeños equipos y proyectos                                                                                                                                                                        \\
    Objetivos principales                          & Alta seguridad                                                                                                                     & Valor rápido                                                                                                                                                                                        \\
\end{longtable}

Con el objetivo de seleccionar la metodología ágil más adecuada para el desarrollo del proyecto, se realizó un
análisis comparativo entre las metodologías ágiles tomando en cuenta criterios tales como, los cuales se detallan en
la Tabla \ref{tab:metodologias-agiles}.

\begin{longtable}{|p{2.5cm}|p{3cm}|p{3cm}|p{3cm}|}
    \caption{Análisis y comparación entre metodologías de desarrollo ágil} \label{tab:metodologias-agiles}                                                                                                                                                                                                                                                                                                                                                                                                                                                                                                \\

    \hline \multicolumn{1}{|c|}{\textbf{Criterio}} & \multicolumn{1}{|c|}{\textbf{XP}}                                                                                                                                   & \multicolumn{1}{|c|}{\textbf{Kanban}}                                                                                                                                                   & \multicolumn{1}{|c|}{\textbf{Scrum}}                                                                                                                                                 \\ \hline
    \endfirsthead

    \multicolumn{4}{c}%
    {{\normalfont \tablename\ \thetable{} -- continuación de la página anterior}}                                                                                                                                                                                                                                                                                                                                                                                                                                                                                                                         \\
    \hline \multicolumn{1}{|c|}{\textbf{Criterio}} & \multicolumn{1}{|c|}{\textbf{XP}}                                                                                                                                   & \multicolumn{1}{|c|}{\textbf{Kanban}}                                                                                                                                                   & \multicolumn{1}{|c|}{\textbf{Scrum}}                                                                                                                                                 \\ \hline
    \endhead

    \hline \multicolumn{4}{|r|}{{Continua en la siguiente página}}                                                                                                                                                                                                                                                                                                                                                                                                                                                                                                                                        \\ \hline
    \endfoot

    \hline \hline
    \endlastfoot
    Tamaño del proyecto                            & Pequeños y medianos.                                                                                                                                                & Pequeños, medianos y grandes.                                                                                                                                                           & Proyectos de toda magnitud y dificultad.                                                                                                                                             \\ \hline
    Tamaño del equipo                              & 5 o menos integrantes.                                                                                                                                              & De 5 a 14 integrantes.                                                                                                                                                                  & De 5 a 9 integrantes.                                                                                                                                                                \\ \hline
    Enfoque                                        & Se centra en ofrecer un producto de alta calidad y en garantizar la satisfacción del cliente.                                                                       & Se centra en la mejora continua del proyecto.                                                                                                                                           & Se centra en la colaboración en equipo.                                                                                                                                              \\ \hline
    Roles                                          & Programador, cliente, tester, coach, encargado del seguimiento.                                                                                                     & No existen roles específicos, aunque algunos equipos pueden recibir apoyo de un coach.                                                                                                  & Product owner, scrum master, equipo de desarrollo.                                                                                                                                   \\ \hline
    Ciclo de vida                                  & Exploración, planificación, iteraciones, producción, monitoreo.                                                                                                     & Inicio, del inglés To-Do, Trabajo en progreso, del inglés Doing, Listo para revisión, del inglés Done.                                                                                  & Sprint, sprint panning, daily scrum, sprint review, sprint retrospective.                                                                                                            \\ \hline
    Ciclo de entrega                               & Se centra en hacer entregas regulares y de menor escala conforme se desarrollan nuevas funcionalidades.                                                             & Las tareas progresan a lo largo del tablero Kanban a medida que avanzan en el proceso de desarrollo, y una vez terminadas, se entregan directamente al cliente.                         & Al final de cada iteración o sprint, se realiza una revisión del trabajo y se entrega el avance del producto al cliente o usuario.                                                   \\ \hline
    Gestión de cambios                             & Se caracteriza por su capacidad de adaptarse a los cambios. En XP, los cambios son recibidos en todo momento debido a su enfoque en la retroalimentación constante. & Los cambios pueden ser gestionados a medida que son detectados y priorizados.                                                                                                           & Si se detecta un cambio que requiere atención, se cancela el Sprint en curso y se inicia uno nuevo con las adaptaciones necesarias.                                                  \\ \hline
    Enfoque en la calidad                          & Las pruebas unitarias, la integración continua y la refactorización son pilares esenciales en XP que contribuyen a mantener la calidad del software.                & La identificación temprana de problemas y su resolución inmediata son fundamentales para mantener la calidad. Cualquier defecto identificado es priorizado y se corregido de inmediato. & Durante un Sprint, el equipo se dedica a implementar y probar las funcionalidades según lo planificado. Cada funcionalidad debe satisfacer los criterios de aceptación establecidos. \\                                                                                                                                           & Planificación y modelado                                                                                                                                        & Rápida y flexible                                                                                                                  \\
\end{longtable}


\subsection{Análisis y selección de herramientas de desarrollo}

Mediante el análisis realizo en la Tabla \ref{tab:metodologias-agiles}, se optó por emplear la metodología de
Desarrollo Rápido de Aplicaciones (RAD) debido a su enfoque centrado en la rápida entrega de soluciones y
prototipos funcionales a los clientes. La selección de RAD se basa en su capacidad para proporcionar resultados
en plazos cortos gracias a su naturaleza iterativa e incremental.

\subsection{Análisis y selección del framework de desarrollo para el servidor web (Backend)}

Como lenguaje de desarrollo para el servidor web (Backend) se optó por utilizar Typescript junto al entorno de
ejecución de Node.js, debido que este permite el desarrollo de aplicaciones escalables y de alto rendimiento,
impulsado por eventos asíncronos, ademas del amplio soporte por parte de la comunidad y documentación disponible,
así como a la facilidad de uso y la gran cantidad de librerías y frameworks disponibles para el desarrollo de aplicaciones
web \cite{haroDesarrolloBackendPara}.
\bigbreak
Como framework de desarrollo para el servidor web (backend), se optó por el uso de NestJS debido a que proporciona
una arquitectura modular y escalable basada en el patrón de diseño de inyección de dependencias. Además, combina
elementos de la programación orientada a objetos (POO), programación funcional (PF) y programación funcional
reactiva (PFR). NestJS permite utilizar como base dos de los frameworks más populares en el desarrollo web, Express y
Fastify, mediante un nivel de abstracción superior que permite exponer las APIs de ambos frameworks de forma
directa al desarrollador, lo que proporciona una mayor flexibilidad al momento de incluir paquetes de terceros.
Además, destaca por la gran cantidad de módulos y librerías que posee \cite{phamDEVELOPINGBACKENDWEB2020}.
\bigbreak
En la tabla \ref{tab:frameworks-backend} se presenta una comparación entre Express y Fastify tomando en cuenta
criterios como el soporte para Typescript, rendimiento, velocidad, documentación, soporte por la comunidad y
paquetes/librerías.

\begin{longtable}{|p{5cm}|p{5cm}|p{5cm}|}
    \caption[]{Análisis y comparación entre los frameworks de NodeJS Express y Fastify \cite{ExpressInfraestructuraAplicaciones}\cite{FastLowOverhead}} \label{tab:frameworks-backend}                                                                                                              \\

    \hline \multicolumn{1}{|c|}{\textbf{Criterio}} & \multicolumn{1}{|c|}{\textbf{Express.js}}                                                                                      & \multicolumn{1}{|c|}{\textbf{Fastify.js}}                                                                   \\ \hline
    \endfirsthead

    \multicolumn{3}{c}%
    {{\normalfont \tablename\ \thetable{} -- continuación de la página anterior}}                                                                                                                                                                                                                 \\
    \hline \multicolumn{1}{|c|}{\textbf{Criterio}} & \multicolumn{1}{|c|}{\textbf{Express.js}}                                                                                      & \multicolumn{1}{|c|}{\textbf{Fastify.js}}                                                                   \\ \hline
    \endhead

    \hline \multicolumn{3}{|r|}{{Continua en la siguiente página}}                                                                                                                                                                                                                                \\ \hline
    \endfoot

    \hline \hline
    \endlastfoot
    Soporte para Typescript                        & Mediante un paquete externo                                                                                                    & Hecho en TypeScript con soporte directo                                                                     \\
    Rendimiento                                    & 11080 peticiones/s                                                                                                             & 45871 peticiones/s                                                                                          \\
    Velocidad                                      & Más lento debido a su mayor cantidad de middleware y flexibilidad                                                              & Significativamente más rápido debido a su enfoque en la velocidad y eficiencia                              \\
    Documentación                                  & Documentación detallada y ampliamente utilizada                                                                                & Documentación completa y fácil de entender                                                                  \\
    Soporte por la comunidad                       & Gran soporte por parte de la comunidad al ser el framework mas usado                                                           & Cuenta con un soporte amplio por parte de su comunidad en creciente aumento                                 \\
    Paquetes/librerías                             & Permite integrar fácilmente tanto librerías propias como de terceros gracias a su arquitectura flexible y su amplia comunidad. & Tine gran soporte para incorporar librerías personalizadas y de terceros gracias a su creciente popularidad \\
\end{longtable}

Considerando el análisis presentado en la Tabla \ref{tab:frameworks-backend}, se optó por utilizar Fastify como
el framework base para NestJS. Esto se debe a que Fastify es más rápido que Express, gracias a su enfoque en la
velocidad y la eficiencia. Fastify ha sido diseñado específicamente con el objetivo de ser rápido y eficiente,
lo que lo convierte en una excelente elección para aplicaciones que necesitan un alto rendimiento.

\subsection{Análisis y selección del framework de desarrollo para el cliente web (Frontend)}

Dado que en el lado del servidor se ha optado por utilizar TypeScript, para el cliente web (frontend) se ha optado
por emplear el mismo lenguaje de programación. Esto se debe a que permite una integración más fluida entre el
cliente y el servidor, además de facilitar la comunicación entre ambos. Para el desarrollo del cliente web
(frontend), se han considerado tres de los frameworks más populares en la actualidad: Angular, ReactJS y VueJS.
A continuación, se presenta una comparación de estos frameworks en la Tabla \ref{tab:frameworks-web}.

\begin{longtable}{|p{3cm}|p{3cm}|p{3cm}|p{3cm}|}
    \caption[Análisis y comparación entre los frameworks Angular, ReactJS, VueJS]{Análisis y comparación entre los frameworks Angular, ReactJS, VueJS  \cite{cincovicComparisonAngularVs2020}} \label{tab:frameworks-web} \\

    \hline \multicolumn{1}{|c|}{\textbf{Criterio}} & \multicolumn{1}{|c|}{\textbf{Angular}} & \multicolumn{1}{|c|}{\textbf{ReactJS}} & \multicolumn{1}{|c|}{\textbf{VueJS}}                                               \\ \hline
    \endfirsthead

    \multicolumn{4}{c}%
    {{\normalfont \tablename\ \thetable{} -- continuación de la página anterior}}                                                                                                                                         \\
    \hline \multicolumn{1}{|c|}{\textbf{Criterio}} & \multicolumn{1}{|c|}{\textbf{Angular}} & \multicolumn{1}{|c|}{\textbf{ReactJS}} & \multicolumn{1}{|c|}{\textbf{VueJS}}                                               \\ \hline
    \endhead

    \hline \multicolumn{4}{|r|}{{Continua en la siguiente página}}                                                                                                                                                        \\ \hline
    \endfoot

    \hline \hline
    \endlastfoot
    Popularidad                                    & Estancada                              & Creciente                              & Creciente                                                                          \\\hline
    Rendimiento                                    & Mayor sobrecarga                       & Ligero                                 & Ligero                                                                             \\\hline
    Soporte de la Comunidad                        & Mediano                                & Grande                                 & Grande                                                                             \\\hline
    Curva de Aprendizaje                           & Mediana                                & Mediana                                & Baja                                                                               \\\hline
    Conocimientos necesarios                       & TypeScript                             & JSX, TSX, Hooks                        & Ninguno                                                                            \\\hline
    Migraciones                                    & Frecuentes                             & Raras                                  & Fácilmente adaptable                                                               \\\hline
    Flexibilidad                                   & Limitada                               & Grande                                 & Grande                                                                             \\
\end{longtable}

\begin{longtable}{|p{0.5cm}|p{6cm}|p{6cm}|}
    \caption[Ventajas y desventajas de Angular, ReactJS y VueJS]{Ventajas y desventajas de Angular, ReactJS y VueJS \cite{xingResearchAnalysisFrontend2019a}} \label{tab:ventajas-desventajas-frameworks-web}                                                                                                                                                                                                                                                                        \\

    \hline \multicolumn{1}{|c|}{\textbf{Framework}} & \multicolumn{1}{|c|}{\textbf{Ventajas}}                                                                                                                                                                           & \multicolumn{1}{|c|}{\textbf{Desventajas}}                                                                                                                                                                 \\ \hline
    \endfirsthead

    \multicolumn{3}{c}%
    {{\normalfont \tablename\ \thetable{} -- continuación de la página anterior}}                                                                                                                                                                                                                                                                                                                                                                                                    \\
    \hline \multicolumn{1}{|c|}{\textbf{Framework}} & \multicolumn{1}{|c|}{\textbf{Ventajas}}                                                                                                                                                                           & \multicolumn{1}{|c|}{\textbf{Desventajas}}                                                                                                                                                                 \\ \hline
    \endhead

    \hline \multicolumn{3}{|r|}{{Continua en la siguiente página}}                                                                                                                                                                                                                                                                                                                                                                                                                   \\ \hline
    \endfoot

    \hline \hline
    \endlastfoot
    ReactJS                                         & \tabitem{Emplea un DOM virtual para lograr una eficiencia máxima al actualizar nodos según sea necesario.}                                                                                                        & \tabitem{Es necesario importar bibliotecas adicionales para manejar el estado y el modelo, ya que React no incluye la arquitectura MVC de forma nativa.}                                                   \\\hline
                                                    & \tabitem{La capacidad de renderizar en el servidor es otra ventaja que este framework ofrece, especialmente adecuada para ciertos tipos de implementaciones, como las aplicaciones enfocadas en el contenido.}    & \tabitem{Aunque React permite su uso, se distancia de los enfoques basados en clases y puede presentar dificultades para aquellos que prefieren la Programación Orientada a Objetos (POO).}                \\\hline
                                                    & \tabitem{Reduce la carga de recursos del usuario mediante el respaldo de bundling y tree shaking.}                                                                                                                &                                                                                                                                                                                                            \\\hline
                                                    & \tabitem{La programación funcional facilita la creación de código que puede ser reutilizado. }                                                                                                                    &                                                                                                                                                                                                            \\\hline
                                                    & \tabitem{Ofrece ventajas en términos de SEO en comparación con Angular y Vue.js. }                                                                                                                                &                                                                                                                                                                                                            \\\hline
    Angular                                         & \tabitem{Utiliza el patrón MVVM (Modelo-Vista-Modelo de Vista), el cual permite manipular la misma colección de datos de forma independiente dentro de una misma aplicación.}                                     & \tabitem{Posee múltiples estructuras como Inyectables, Componentes, Tuberías, Módulos, entre otros, que suelen presentar un mayor nivel de complejidad para su comprensión.}                               \\\hline
                                                    & \tabitem{Su estructura y arquitectura están diseñadas específicamente para mejorar la escalabilidad de los proyectos.}                                                                                            & \tabitem{Experimenta actualizaciones continuas, incorporando mejoras nuevas y significativas de manera constante. Sin embargo, estas actualizaciones pueden plantear desafíos al adaptarse a los cambios.} \\\hline
                                                    & \tabitem{La inyección de dependencias en los componentes ayuda a mejorar la modularidad de la aplicación.}                                                                                                        &                                                                                                                                                                                                            \\\hline
                                                    & \tabitem{La programación funcional facilita la creación de código que puede ser reutilizado. }                                                                                                                    &                                                                                                                                                                                                            \\\hline
    VueJS                                           & \tabitem{Facilita la creación de modelos modulares de gran alcance que pueden renderizarse eficientemente gracias a su estructura fundamental, sin requerir esfuerzos adicionales.}                               & \tabitem{Su participación en el mercado es moderada, lo que indica que el intercambio de información en esta plataforma se encuentra en sus primeras fases de desarrollo.}                                 \\\hline
                                                    & \tabitem{Posee una alta capacidad de respuesta y ofrece una sencilla vinculación de datos entre el código HTML y JavaScript. }                                                                                    & \tabitem{Existe el riesgo de que su flexibilidad pueda ser un problema al integrarse en proyectos extensos debido a la carencia de recursos disponibles.}                                                  \\\hline
                                                    & \tabitem{Vue gestiona de forma sobresaliente la vinculación de datos bidireccional dinámica. Además, lleva a cabo la manipulación del DOM de manera coherente, lo que lo hace ideal para diversas aplicaciones. } &                                                                                                                                                                                                            \\
\end{longtable}


Tomando en cuenta el análisis de las características de cada framework, así como sus ventajas y desventajas, se
optó por el uso de ReactJS para el desarrollo del cliente web (frontend). Esto se debe a que ReactJS es una de
las librerías más populares y ampliamente utilizadas en la actualidad, lo que significa que cuenta con una gran
comunidad de desarrolladores y una amplia variedad de recursos disponibles. Además, ReactJS posee un gran
rendimiento y eficiencia gracias a su manejo eficiente del DOM virtual, lo que lo convierte en una excelente
elección para aplicaciones web que requieren una interfaz de usuario rápida y receptiva, as���� como su capacidad
de crear componentes reutilizables, lo que facilita el desarrollo de aplicaciones web complejas y escalables.

\subsection{Análisis y selección del framework de desarrollo para el cliente móvil (Frontend)}

Para el desarrollo del cliente móvil (frontend), se han considerado tres de los frameworks más populares en la
actualidad: React Native, Flutter y Ionic. A continuación, se presenta una comparación de estos frameworks en la
Tabla \ref{tab:frameworks-movil}.

\begin{longtable}{|p{5cm}|p{3cm}|p{3cm}|p{3cm}|}
    \caption[Análisis y comparación entre los frameworks React Native, Flutter y Xamarin]{Análisis y comparación entre los frameworks React Native, Flutter y Xamarin \cite{alferezzamoraEstudioComparativoFrameworks2018}\cite{lazoANALISISDISENOAPLICATIVO}} \label{tab:frameworks-movil}                                                                    \\

    \hline \multicolumn{1}{|c|}{\textbf{Criterio}} & \multicolumn{1}{|c|}{\textbf{React Native}}               & \multicolumn{1}{|c|}{\textbf{Flutter}}                                                                         & \multicolumn{1}{|c|}{\textbf{Xamarin}}                                                                                       \\ \hline
    \endfirsthead

    \multicolumn{4}{c}%
    {{\normalfont \tablename\ \thetable{} -- continuación de la página anterior}}                                                                                                                                                                                                                                                                              \\
    \hline \multicolumn{1}{|c|}{\textbf{Criterio}} & \multicolumn{1}{|c|}{\textbf{React Native}}               & \multicolumn{1}{|c|}{\textbf{Flutter}}                                                                         & \multicolumn{1}{|c|}{\textbf{Xamarin}}                                                                                       \\ \hline
    \endhead

    \hline \multicolumn{4}{|r|}{{Continua en la siguiente página}}                                                                                                                                                                                                                                                                                             \\ \hline
    \endfoot

    \hline \hline
    \endlastfoot
    Lanzamiento                                    & 2015                                                      & 2017                                                                                                           & 2013                                                                                                                         \\\hline
    Licencia                                       & MIT Licensed                                              & BSD                                                                                                            & MIT Licensed                                                                                                                 \\\hline
    Lenguaje                                       & JavaScript, TypeScript                                    & Dart                                                                                                           & C\#                                                                                                                          \\\hline
    Plataformas soportadas                         & Android, IOS                                              & Android, IOS                                                                                                   & Android, IOS                                                                                                                 \\\hline
    Open source                                    & Si                                                        & Si                                                                                                             & Si                                                                                                                           \\\hline
    Paradigma                                      & Declarativo                                               & Declarativo                                                                                                    & Imperativo                                                                                                                   \\\hline
    Recarga en tiempo real                         & Si                                                        & Si                                                                                                             & No                                                                                                                           \\\hline
    Recarga en vivo                                & Si                                                        & Si                                                                                                             & No                                                                                                                           \\\hline
    Administrador de paquetes                      & NPM, Yarn                                                 & Pub                                                                                                            & NuGet                                                                                                                        \\\hline
    Enfoque multiplataforma                        & Interpretado                                              & Compilado a nativo                                                                                             & Compilado a nativo                                                                                                           \\\hline
    Compatibilidad con 64bits                      & No en android                                             & Si                                                                                                             & Si                                                                                                                           \\\hline
    Portabilidad                                   & Si                                                        & No                                                                                                             & No                                                                                                                           \\\hline
    Geolocalización                                & Incluido                                                  & Mediante paquete de la comunidad                                                                               & Incluido                                                                                                                     \\\hline
    Notificaciones                                 & Mediante paquete de la comunidad                          & Mediante paquete de la comunidad                                                                               & Mediante paquete de la comunidad                                                                                             \\\hline
    Rendimiento                                    & Alto desempeño al ser un framework ligero                 & Elevada eficiencia gracias a su propio motor de renderizado                                                    & Un desempeño sólido, aunque podría beneficiarse de una capa extra de abstracción.                                            \\\hline
    Integración con APIs y Bibliotecas             & Gran cantidad de paquetes de terceros                     & Una amplia disponibilidad de bibliotecas y paquetes de terceros gracias a su comunidad en constante desarrollo & Gran cantidad de bibliotecas y recursos disponibles, especialmente diseñados para integrarse con los servicios de Microsoft. \\\hline
    Documentación                                  & Amplia documentación                                      & Documentación amplia y detallada, además  mas de gran cantidad de recursos de terceros                         & Documentación completa y abundante                                                                                           \\\hline
    Tiempo de desarrollo                           & Rápida gracias a su hot reload con cambios en tiempo real & Rápido gracias al hot reload que permite visualizar cambios en tiempo real                                     & Tiempo de desarrollo mas lento debido a la compilación                                                                       \\
\end{longtable}

Considerando el análisis realizado en la Tabla \ref{tab:frameworks-movil}, se optó por utilizar Flutter para el
desarrollo del cliente móvil (frontend). Esto se debe a que Flutter es un framework de desarrollo de aplicaciones
móviles multiplataforma creado por Google, que permite el desarrollo de aplicaciones móviles nativas de alta
calidad para Android e iOS desde una sola base de código. Flutter destaca por su rendimiento y eficiencia, gracias
a su propio motor de renderizado, lo que lo convierte en una excelente elección para aplicaciones móviles que
requieren un alto rendimiento y una interfaz de usuario rápida y receptiva. Además, Flutter cuenta con una amplia
variedad de widgets y herramientas que facilitan el desarrollo de aplicaciones móviles complejas y escalables.

\subsection{Análisis y selección de la base de datos}

Pra la base de datos del proyecto se han considerado dos opciones: MySQL y PostgreSQL. A continuación, se presenta
una comparación de estas bases de datos en la Tabla \ref{tab:bases-datos}.

\label{app:analisis-bases-datos}
\begin{longtable}{|p{5cm}|p{5cm}|p{5cm}|}
    \caption[Análisis y comparación entre las bases de datos MySQL y PostgreSQL]{Análisis y comparación entre las bases de datos MySQL y PostgreSQL \cite{lazoANALISISDISENOAPLICATIVO}} \label{tab:bases-datos} \\

    \hline \multicolumn{1}{|c|}{\textbf{Criterio}} & \multicolumn{1}{|c|}{\textbf{MySQL}}                                              & \multicolumn{1}{|c|}{\textbf{PostgreSQL}}                               \\ \hline
    \endfirsthead

    \multicolumn{3}{c}%
    {{\normalfont \tablename\ \thetable{} -- continuación de la página anterior}}                                                                                                                                \\
    \hline \multicolumn{1}{|c|}{\textbf{Criterio}} & \multicolumn{1}{|c|}{\textbf{MySQL}}                                              & \multicolumn{1}{|c|}{\textbf{PostgreSQL}}                               \\ \hline
    \endhead

    \hline \multicolumn{3}{|r|}{{Continua en la siguiente página}}                                                                                                                                               \\ \hline
    \endfoot

    \hline \hline
    \endlastfoot
    GUI                                            & MySQL Workbench                                                                   & pgAdmin                                                                 \\\hline
    Consumo de recursos                            & Consumo mayor de CPU y memoria                                                    & Consumo mayor de CPU y memoria                                          \\\hline
    Tiempo de respuesta de CRUD                    & Bajo                                                                              & Alto                                                                    \\\hline
    Lenguaje de ejecución                          & C/C++                                                                             & C                                                                       \\\hline
    herencia de tablas                             & No                                                                                & Si                                                                      \\\hline
    Tipos de datos                                 & Solo tipos estándar                                                               & Estándar, store, arreglos, geográficos, definido por el usuario, etc.   \\\hline
    APIs y otros métodos de acceso                 & ADO.NET, JDBC, biblioteca C nativa, ODBC, API de transmisión para objetos grandes & ADO.NET, JDBC, ODBC, API nativa patentada                               \\\hline
    Tipos de conexión                              & Las conexiones son subprocesos del sistema operativo                              & Las conexiones son subprocesos del sistema operativo                    \\\hline
    Respaldo                                       & En MySQL, Mysqldump y XtraBackup, las herramientas proporcionan respaldo          & PostgreSQL proporciona una copia de seguridad completa en línea.        \\\hline
    Lenguajes soportados                           & C/C++, PHP, Java, Go, Delphi, Lisp, Erlang, Node.js, R, Perl, PHP                 & Go, C/C++, Java, Delphi, Javascript, Erlang, Lisp, R, .Net, Tcl, Python \\\hline
    Sistemas Operativos Soportados                 & Windows, Linux, macOS, Oracle Solaris, Fedora, FreeBSD                            & Windows, macOS, BSD, Linux, Solaris                                     \\
\end{longtable}

Tomando en cuenta el análisis presentado en la Tabla \ref{tab:bases-datos}, se optó por utilizar PostgreSQL como
la base de datos del proyecto. Esto se debe a que PostgreSQL es un sistema de gestión de bases de datos relacional
de código abierto y de alto rendimiento, que ofrece una amplia variedad de características y funcionalidades
avanzadas, como soporte para tipos de datos geoespaciales, herencia de tablas, tipos de datos definidos por el
usuario, entre otros. Además, PostgreSQL cuenta con una arquitectura robusta y escalable, que permite gestionar
grandes volúmenes de datos de forma eficiente y fiable, lo que lo convierte en una excelente elección para
aplicaciones que requieren una base de datos potente y confiable.