\subsection{Marco teórico}
Esta sección establecerá las bases, fundamentos y teorías pertinentes a través de una exhaustiva revisión de
antecedentes. Esto permitirá respaldar y contextualizar el desarrollo de la presente investigación, ofreciendo
un marco sólido que sustente cada paso del estudio.

\subsubsection{Metodología ágil}
Una metodología ágil es un conjunto de técnicas aplicadas utilizadas con el propósito de facilitar los tiempos de
desarrollo y obtener un producto final de calidad. Los autores de \cite{lesanoperezAplicativoMovilGeoubicacion2022} y
\cite{chicaizavillegasAplicacionWebPara2023}
señalan el uso de la
metodología ágil Extreme Programming (XP). Los autores
mencionan la flexibilidad y facilidad de modelado que ofrece la metodología ágil XP, además, de mejorar los tiempos
de desarrollo, calidad del producto y ayuda a minimizar los errores durante el desarrollo. También
se indica que la metodología XP está orienta a grupos de
trabajo pequeño, además, de adaptarse mejor a las necesidades de los proyectos ya que es versátil al incorporar
cambios durante el desarrollo. Así mismo los autores señalan que una de las
principales ventajas de la metodología XP es la capacidad de dividir el trabajo en pequeñas historias de usuario
las cuales son completamente funcionales, lo cual garantiza la entrega
del producto final en etapas manejables.

\bigbreak
En \cite{chasichangoAplicacionMovilApoyo2022} se utilizó la metodología de desarrollo Mobile-D, ya que combina
beneficios de metodologías como XP, Crystal y RUP. El autor menciona que la metodología Mobile-D se enfoca en el
desarrollo de aplicaciones móviles, especialmente útil en aplicaciones pequeñas con pocos módulos y vistas. Además,
ofrece un estilo de desarrollo por iteraciones, el cual es rápido y útil para grupos pequeños.

\subsubsection{Framework Web}
Un framework web constituye un conjunto de herramientas y librerías esenciales para el desarrollo de aplicaciones web.
En \cite{chicaizavillegasAplicacionWebPara2023}, el autor destaca Angular como un framework que posibilita el diseño
de sistemas web estables y escalables. Esto se debe a las capacidades que ofrece para crear aplicaciones SPA eficientes.
Asimismo, menciona que Angular proporciona una arquitectura basada en componentes y módulos organizados, los cuales se
integran fácilmente con otras tecnologías, como el caso de ASP.NET Core, utilizado como plantilla de interfaz de usuario.
% Por otro lado, en \cite{moralesgutamaSistemaWebUsando2023}, el autor describe a Laravel como un framework web que ofrece
% una arquitectura limpia y cuenta con una amplia comunidad de desarrolladores. Esto asegura un soporte extenso y una fuerte
% presencia en el mercado. Además, señala que Laravel dispone de una gran cantidad de herramientas y bibliotecas que simplifican
% la implementación de diversas funcionalidades.

\subsubsection{Framework Backend}
Un framework backend consiste en un conjunto de herramientas y librerías predefinidas que permiten desarrollar y mantener
la lógica de una aplicación en el lado del servidor. En \cite{chicaizavillegasAplicacionWebPara2023}, el autor menciona
que ASP.NET es un framework backend gratuito, seguro, de alto rendimiento y de código abierto. El autor destaca que ASP.NET
permite ejecutar aplicaciones web en sistemas operativos como Windows, macOS y Linux utilizando C\# en el lado del servidor,
% además de otras tecnologías en el lado del cliente. Por otra parte, en \cite{moralesgutamaSistemaWebUsando2023}, el autor indica que Laravel,
% como framework backend, es robusto y completo. Se mencionan también las medidas de seguridad incluidas en Laravel para
% proteger las aplicaciones contra ataques de inyección SQL y XSS. El autor explica que Laravel cuenta con un Object-Relational
% Mapping (ORM) llamado Eloquent, que permite manipular la base de datos de manera intuitiva y eficiente. En cambio, 
el autor de
\cite{chasichangoAplicacionMovilApoyo2022} destaca a Node.js como tecnología backend  la cual permitiendo el manejo eficiente
de múltiples procesos simultáneamente. Además, se menciona que la curva de aprendizaje de Node.js es baja, al igual que su
costo de desarrollo. También se resalta que Node.js ofrece una manera sencilla de crear APIs REST utilizando Express,
las API REST permiten comunicar diferentes sistemas mediante los protocolos HTTP, lo que facilita la creación rápida y
flexible de un backend completo.

\subsubsection{Aplicación Movil}
Una aplicación móvil se refiere a un tipo de aplicación diseñada para ejecutarse en teléfonos inteligentes o tabletas.
En \cite{chasichangoAplicacionMovilApoyo2022}, el autor destaca que IONIC, utilizado en el desarrollo de aplicaciones
móviles, presenta una curva de aprendizaje reducida, ya que puede combinarse con frameworks como Angular y librerías de
componentes como React y Svelte. Además, menciona que IONIC permite la creación de componentes reutilizables, lo que conlleva
a un código más legible y fácil de mantener. Asimismo, resalta su extensa documentación respaldada por una gran comunidad,
lo que facilita la solución a errores de manera más sencilla. Por otra parte, en \cite{lesanoperezAplicativoMovilGeoubicacion2022},
el autor enfatiza el uso de Xamarin Forms para el diseño de aplicaciones móviles debido a su capacidad para crear aplicaciones
multiplataforma. Expone que Xamarin permite el desarrollo de aplicaciones utilizando el lenguaje C\#, el cual se compila en
código nativo para Android e iOS. Además, al ser una tecnología de Microsoft, posibilita la integración sencilla de paquetes
externos mediante NuGet.

\subsubsection{Bases de datos}
Una base de datos es un sistema electrónico que permite el almacenamiento de datos de forma organizada.
En \cite{chicaizavillegasAplicacionWebPara2023}, el autor destaca que MySQL, como Sistema Gestor de Bases de Datos Relacionales
(SGBDR), ofrece un alto rendimiento y fiabilidad, lo que permite almacenar, buscar, ordenar y obtener información de forma eficiente.
% Por otro lado, en \cite{moralesgutamaSistemaWebUsando2023}, 
También se indica que MySQL es uno de los SGBDR más populares, siendo de
código abierto y respaldado por Oracle. En el mismo sentido, en \cite{lesanoperezAplicativoMovilGeoubicacion2022}, el autor
resalta a PostgreSQL como SGBDR por ser uno de los gestores más rápidos y seguros del mundo. Expone su gran escalabilidad,
facilidad de uso y capacidad para manejar no solo variables primitivas, sino también variables tipo objeto. Además, señala que
la base de datos de PostgreSQL es ilimitada, permitiendo un número ilimitado de campos por tabla.

\bigbreak
En otro enfoque, el autor de \cite{chasichangoAplicacionMovilApoyo2022} menciona a MongoDB como una base de datos NoSQL capaz
de manejar una gran cantidad de información. Se destaca su carácter de código abierto y su flexibilidad para trabajar con
diversas tecnologías de backend. Además, se resalta su escalabilidad al estar vinculada a servidores de Amazon, Google y
Microsoft, ofreciendo también planes gratuitos convenientes para aplicaciones pequeñas.

\subsubsection{Mapa web}
Un mapa web es una visualización interactiva de información geográfica que se puede abrir en navegadores y dispositivos móviles.
En \cite{botto-tobarAppliedTechnologiesSecond2021}, el autor destaca que el API proporcionado por Google Maps para renderizar
mapas ofrece varias características y prestaciones, siendo uno de los productos más importantes de Google. También menciona que
una de las principales ventajas de esta API es su capacidad para integrarse con distintos lenguajes de programación y, por ende,
con diferentes frameworks. Esto hace que sea altamente flexible para su uso en diferentes sistemas. Por otro lado, en
\cite{gomezcantilloAplicativoMovilPara}, el autor señala que Leaflet permite crear mapas interactivos y aplicaciones de
mapas en línea, además de ser de código abierto. También destaca la amplia gama de herramientas disponibles en Leaflet que
facilitan la visualización de datos georreferenciados. Incluso, ofrece a los desarrolladores la capacidad de crear experiencias
de usuario altamente interactivas y personalizables, lo cual resulta ideal para aplicaciones tanto móviles como web.

\subsubsection{Geo Business Intelligence (GEOBI)}
En \cite{herreraGeoBIBigVGI2015} se empleó GEOBI dado que esta mejora la toma de decisiones en organizaciones. El
GEOBI proporciona información más precisa y oportuna incorporando la dimensión espacial en la información. También
permite que sus componentes se vuelvan geográficos, es decir, en lugar de OLAP se utiliza SOLAP una combinación de un
Sistema de información geográfica (SIG) y un cubo OLAP, además, sus bases de datos se convierten en GeoDB, bases de
datos que contienen información geográfica.

\subsubsection{Patron MVC}
En \cite{lesanoperezAplicativoMovilGeoubicacion2022}, el autor menciona la importancia del patrón Model View Controller
(MVC), el cual permite separar una aplicación en tres capas de componentes: Modelo, Vista y Controlador. El autor señala que
la separación de funciones de este patrón de arquitectura mejora la mantenibilidad del código y hace más sencillo depurar
errores. El autor indica que el uso del patrón MVC permite disminuir los errores, dado que las dependencias no están
distribuidas en diferentes secciones del código y, por lo tanto, evita mezclar la lógica de negocio con las interfaces
de usuario.

\subsubsection{Diagrama UML}
En los artículos revisados no se señala el uso de diagramas UML en el desarrollo de sistemas de software. Sin embargo, se
considera necesario mencionarlo para el desarrollo de este proyecto. El lenguaje de modelado unificado (UML) es una
representación visual y semántica de la arquitectura, diseño e implementación de sistemas de software. Los diagramas
UML permiten describir los límites, la estructura, el comportamiento y los objetos que contiene un sistema. Aunque este
no es un lenguaje de programación, mediante herramientas externas puede ser utilizado para la generación de código para
diferentes lenguajes. Según el Object Management Group (OMG), el propósito de UML es proveer a los ingenieros,
desarrolladores y arquitectos de software un medio por el cual puedan realizar análisis, diseño e implementación de
sistemas de software. Aunque UML no está enfocado en bases de datos, este permite crear modelos de datos conceptuales
de alto nivel.

\subsubsection{NextJS}
En los artículos revisados no se menciona el uso de NextJS como framework frontend para el desarrollo de aplicaciones web.
Sin embargo, se considera necesario mencionarlo para el desarrollo de este proyecto. NextJS es un framework web el cual
utiliza la librería de componentes React la cual es desarrollada por Meta. NextJS permite crear aplicaciones web
completas utilizando React para crear interfaces de usuario dinámicas y rápidas basadas en componentes y añadiendo
características adicionales y optimizaciones tales como enrutador basado en archivos, compilación y empaquetado de la
aplicación, optimización de images, Renderizado del lado del servidor (SSR), Renderizado del lado del cliente (CSR),
Generación de sitios estáticos (SSG). NextJS permite desarrollar un backend basado en NodeJS con el cual crear un API
REST en la misma aplicación.

\subsubsection{NestJS}
En los artículos revisados no se menciona el uso de NestJS como framework backend para el desarrollo de API REST.
Sin embargo, se considera necesario mencionarlo para el desarrollo de este proyecto. NestJS es un framework backend
basado en NodeJS el cual permite diseñar aplicaciones en el lado del servidor como. NestJS combina elementos de la
programación orientada a objetos (POO), programación funcional (PF) y programación reactiva funcional junto con con
TypeScript para diseñar aplicaciones eficientes y escalables. NestJS brinda un nivel de abstracción a framework de
servidor HTTP sólidos como Express y Fastify, además de exponer su API directamente con lo cual permite a los
desarrolladores implementar modules externos disponibles para la plataforma.

\subsubsection{Flutter}
En los artículos revisados, no se menciona el uso de Flutter como framework para el desarrollo de aplicaciones móviles;
sin embargo, se considera necesario mencionarlo para el desarrollo de este proyecto. Flutter es un framework desarrollado
por Google para el desarrollo de aplicaciones móviles, web y de escritorio de forma nativa con solo código base. Permite
crear interfaces de usuario visualmente atractivas al manejar los elementos de la interfaz en componentes más pequeños
llamados widgets. Flutter tiene un alto rendimiento, dado que utiliza Dart como lenguaje de programación, el cual compila
código a máquina, permitiendo que las aplicaciones desarrolladas sean rápidas y eficaces. Además, Flutter ofrece una buena
experiencia de desarrollo al incluir herramientas que permiten realizar actualizaciones en caliente y revisar el estado de
los componentes de la aplicación.