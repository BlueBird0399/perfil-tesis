\begin{longtable}{|p{2.5cm}|p{3cm}|p{3cm}|p{3cm}|}
    \caption{Análisis y comparación entre metodologías de desarrollo ágil} \label{tab:metodologias-agiles}                                                                                                                                                                                                                                                                                                                                                                                                                                                                                                \\

    \hline \multicolumn{1}{|c|}{\textbf{Criterio}} & \multicolumn{1}{|c|}{\textbf{XP}}                                                                                                                                   & \multicolumn{1}{|c|}{\textbf{Kanban}}                                                                                                                                                   & \multicolumn{1}{|c|}{\textbf{Scrum}}                                                                                                                                                 \\ \hline
    \endfirsthead

    \multicolumn{4}{c}%
    {{\normalfont \tablename\ \thetable{} -- continuación de la página anterior}}                                                                                                                                                                                                                                                                                                                                                                                                                                                                                                                         \\
    \hline \multicolumn{1}{|c|}{\textbf{Criterio}} & \multicolumn{1}{|c|}{\textbf{XP}}                                                                                                                                   & \multicolumn{1}{|c|}{\textbf{Kanban}}                                                                                                                                                   & \multicolumn{1}{|c|}{\textbf{Scrum}}                                                                                                                                                 \\ \hline
    \endhead

    \hline \multicolumn{4}{|r|}{{Continua en la siguiente página}}                                                                                                                                                                                                                                                                                                                                                                                                                                                                                                                                        \\ \hline
    \endfoot

    \hline \hline
    \endlastfoot
    Tamaño del proyecto                            & Pequeños y medianos.                                                                                                                                                & Pequeños, medianos y grandes.                                                                                                                                                           & Proyectos de toda magnitud y dificultad.                                                                                                                                             \\ \hline
    Tamaño del equipo                              & 5 o menos integrantes.                                                                                                                                              & De 5 a 14 integrantes.                                                                                                                                                                  & De 5 a 9 integrantes.                                                                                                                                                                \\ \hline
    Enfoque                                        & Se centra en ofrecer un producto de alta calidad y en garantizar la satisfacción del cliente.                                                                       & Se centra en la mejora continua del proyecto.                                                                                                                                           & Se centra en la colaboración en equipo.                                                                                                                                              \\ \hline
    Roles                                          & Programador, cliente, tester, coach, encargado del seguimiento.                                                                                                     & No existen roles específicos, aunque algunos equipos pueden recibir apoyo de un coach.                                                                                                  & Product owner, scrum master, equipo de desarrollo.                                                                                                                                   \\ \hline
    Ciclo de vida                                  & Exploración, planificación, iteraciones, producción, monitoreo.                                                                                                     & Inicio, del inglés To-Do, Trabajo en progreso, del inglés Doing, Listo para revisión, del inglés Done.                                                                                  & Sprint, sprint panning, daily scrum, sprint review, sprint retrospective.                                                                                                            \\ \hline
    Ciclo de entrega                               & Se centra en hacer entregas regulares y de menor escala conforme se desarrollan nuevas funcionalidades.                                                             & Las tareas progresan a lo largo del tablero Kanban a medida que avanzan en el proceso de desarrollo, y una vez terminadas, se entregan directamente al cliente.                         & Al final de cada iteración o sprint, se realiza una revisión del trabajo y se entrega el avance del producto al cliente o usuario.                                                   \\ \hline
    Gestión de cambios                             & Se caracteriza por su capacidad de adaptarse a los cambios. En XP, los cambios son recibidos en todo momento debido a su enfoque en la retroalimentación constante. & Los cambios pueden ser gestionados a medida que son detectados y priorizados.                                                                                                           & Si se detecta un cambio que requiere atención, se cancela el Sprint en curso y se inicia uno nuevo con las adaptaciones necesarias.                                                  \\ \hline
    Enfoque en la calidad                          & Las pruebas unitarias, la integración continua y la refactorización son pilares esenciales en XP que contribuyen a mantener la calidad del software.                & La identificación temprana de problemas y su resolución inmediata son fundamentales para mantener la calidad. Cualquier defecto identificado es priorizado y se corregido de inmediato. & Durante un Sprint, el equipo se dedica a implementar y probar las funcionalidades según lo planificado. Cada funcionalidad debe satisfacer los criterios de aceptación establecidos. \\                                                                                                                                           & Planificación y modelado                                                                                                                                        & Rápida y flexible                                                                                                                  \\
\end{longtable}