% \section{Análisis y discusión de los resultados}
% En esta sección se presentan los resultados obtenidos durante la investigación para el desarrollo del
% presente proyecto. Aquí se describen los procesos realizados para el envío de alertas de emergencia con
% ubicación en tiempo real (datos georreferenciados), basados en las respuestas obtenidas de las entrevistas
% realizadas en el capítulo anterior, así como la medición de la precisión de dichos datos georreferenciados.


% \section{Análisis del proceso actual}
% A continuación se describe el proceso para reportar incidentes por parte de los ciudadanos a través del servicio
% de emergencia Ecu 911. Este proceso se basa en la información obtenida de la entrevista realizada al teniente
% coronel de estado mayor Christian Iván Quintana Guerra, En la Figura 1 se detalla el siguiente proceso, donde:

% \begin{enumerate}
%     \item El usuario llama al número de emergencia del ECU 911.
%     \item El operador recibe la llamada y solicita la información necesaria al usuario.
%     \item En base al tipo de incidente, el operador asigna la emergencia a la entidad correspondiente.
%     \item La entidad correspondiente recibe la emergencia y envía una unidad al lugar del incidente.
%     \item La unidad llega al lugar del incidente, donde:
%           \begin{enumerate}
%               \item La unidad verifica la información con la que cuenta:
%                     \begin{enumerate}
%                         \item Si cuenta con la información necesaria, la emergencia es atendida.
%                         \item Si no cuenta con la información necesaria, se solicita a la víctima o testigo la información faltante.
%                     \end{enumerate}
%           \end{enumerate}
%     \item Se reporta el incidente la atención de la emergencia y la información es almacenada en una matriz de excel.
% \end{enumerate}

% \begin{figure}[H]
%     \centering
%     \includegraphics[width=0.8\textwidth]{chapters/III-resultados-y-discusion/resources/images/proceso-actual.png}
%     \caption{Proceso actual de reporte de incidentes al ECU 911.}
%     \label{fig:proceso-actual}
% \end{figure}


% \section{Análisis de herramientas de desarrollo}
% En este apartado se presenta el análisis y selección de las herramientas de desarrollo que se utilizarán para la
% implementación del proyecto.

% \section{Análisis y selección de la metodología de desarrollo}

% En la Tabla \ref{tab:metodologias} se presenta una comparación entre las metodologías de desarrollo de software ágil
% y tradicional. Se evalúan aspectos como el estilo de gestión, la gestión del conocimiento, el modelo de desarrollo,
% la estructura organizativa y el control de calidad, entre otros. En este análisis, se destaca que la metodología ágil
% es la más adecuada para el desarrollo del proyecto actual. Esta metodología permite realizar cambios en el desarrollo
% del software de manera rápida y eficiente, además de adaptarse fácilmente a equipos de desarrollo pequeños y medianos,
% manteniendo una planificación y control de calidad permanente en iteraciones a corto plazo. En contraposición, la
% metodología tradicional requiere una planificación exhaustiva y detallada, lo cual no es apropiado para el desarrollo
% de software adaptativo de alta calidad en equipos pequeños. Esto se debe a que la metodología tradicional presupone
% que los sistemas sean completamente definidos y predecibles, y cualquier cambio en el desarrollo puede resultar en
% un alto costo de reinicio.

% \bigbreak
% Por consiguiente, se optó por la metodología ágil para el desarrollo del presente proyecto.

% \begin{longtable}{|p{5cm}|p{5cm}|p{5cm}|}
    \caption[]{Análisis y comparación entre metodologías de desarrollo de software ágil y tradicional \cite{stoicaSoftwareDevelopmentAgile2013}.} \label{tab:metodologias}                                                                                                                                                                                                                      \\

    \hline \multicolumn{1}{|c|}{\textbf{Criterio}} & \multicolumn{1}{|c|}{\textbf{Metodología tradicional}}                                                                             & \multicolumn{1}{|c|}{\textbf{Metodología ágil}}                                                                                                                                                     \\ \hline
    \endfirsthead

    \multicolumn{3}{c}%
    {{\normalfont \tablename\ \thetable{} -- continuación de la página anterior}}                                                                                                                                                                                                                                                                                                             \\
    \hline \multicolumn{1}{|c|}{\textbf{Criterio}} & \multicolumn{1}{|c|}{\textbf{Metodología tradicional}}                                                                             & \multicolumn{1}{|c|}{\textbf{Metodología ágil}}                                                                                                                                                     \\ \hline
    \endhead

    \hline \multicolumn{3}{|r|}{{Continua en la siguiente página}}                                                                                                                                                                                                                                                                                                                            \\ \hline
    \endfoot

    \hline \hline
    \endlastfoot
    Hipótesis fundamental                          & Los sistemas pueden ser completamente definidos, predecibles y se construyen a través de una planificación exhaustiva y detallada. & Pequeños equipos emplean el principio de mejorar constantemente el diseño y realizar pruebas basadas en una retroalimentación rápida y cambios para desarrollar software adaptativo de alta calidad \\
    Estilo de gestión                              & Mando y control                                                                                                                    & Liderazgo y colaboración                                                                                                                                                                            \\
    Gestión del conocimiento                       & Explicito                                                                                                                          & Tácito                                                                                                                                                                                              \\
    Comunicación                                   & Formal                                                                                                                             & Informal                                                                                                                                                                                            \\
    Modelo de desarrollo                           & Modelo de ciclo de vida (cascada, espiral o modelos modificados)                                                                   & Modelo de entrega evolutivo                                                                                                                                                                         \\
    Estructura organizativa                        & Mecánica (burocrática, alta formalización), dirigida a grandes organizaciones                                                      & Orgánica (flexible y participativa, fomenta la cooperación social), dirigida a pequeñas y medianas organizaciones                                                                                   \\
    Control de calidad                             & Planificación difícil y control estricto. Pruebas difíciles y tardías                                                              & Control permanente o requisitos, diseño y soluciones. Pruebas permanentes                                                                                                                           \\
    Requisitos de los usuarios                     & Detallado y definido antes de la codificación/implantación                                                                         & Entrada interactiva                                                                                                                                                                                 \\
    Coste del reinicio                             & Alto                                                                                                                               & Bajo                                                                                                                                                                                                \\
    Dirección del desarrollo                       & Fijo                                                                                                                               & Fácil de cambiar                                                                                                                                                                                    \\
    Pruebas                                        & Una vez finalizada la codificación                                                                                                 & Cada iteración                                                                                                                                                                                      \\
    Participación del cliente                      & Baja                                                                                                                               & Alta                                                                                                                                                                                                \\
    Requisitos                                     & Muy estable, conocido de antemano                                                                                                  & Emergente, con cambios rápidos                                                                                                                                                                      \\
    Arquitectura                                   & Diseño para necesidades actuales y previsibles                                                                                     & Diseño para las necesidades actuales                                                                                                                                                                \\
    Remodelación                                   & Caro                                                                                                                               & No es caro                                                                                                                                                                                          \\
    Tamaño                                         & Grandes equipos y proyectos                                                                                                        & Pequeños equipos y proyectos                                                                                                                                                                        \\
    Objetivos principales                          & Alta seguridad                                                                                                                     & Valor rápido
\end{longtable}

% Con el objetivo de seleccionar la metodología ágil más adecuada para el desarrollo del proyecto, se realizó un
% análisis comparativo entre las metodologías ágiles tomando en cuenta criterios tales como, los cuales se detallan en
% la Tabla \ref{tab:metodologias-agiles}.

% \begin{longtable}{|p{3cm}|p{2.5cm}|p{2.5cm}|p{2.5cm}|p{2.5cm}|}
    \caption[]{Análisis y comparación entre metodologías de desarrollo ágil \cite{carrionvalarezoAPLICACIONWEBPARA2024}} \label{tab:metodologias-agiles}                                                                                                                                                                                                                                                                                            \\

    \hline \multicolumn{1}{|c|}{\textbf{Criterio}} & \multicolumn{1}{|c|}{\textbf{XP}}                                                                      & \multicolumn{1}{|c|}{\textbf{Lean}}                                                            & \multicolumn{1}{|c|}{\textbf{RAD}}                                                         & \multicolumn{1}{|c|}{\textbf{Kanban}}                                                   \\ \hline
    \endfirsthead

    \multicolumn{5}{c}%
    {{\normalfont \tablename\ \thetable{} -- continuación de la página anterior}}                                                                                                                                                                                                                                                                                                                                                                   \\
    \hline \multicolumn{1}{|c|}{\textbf{Criterio}} & \multicolumn{1}{|c|}{\textbf{XP}}                                                                      & \multicolumn{1}{|c|}{\textbf{Lean}}                                                            & \multicolumn{1}{|c|}{\textbf{RAD}}                                                         & \multicolumn{1}{|c|}{\textbf{Kanban}}                                                   \\ \hline
    \endhead

    \hline \multicolumn{5}{|r|}{{Continua en la siguiente página}}                                                                                                                                                                                                                                                                                                                                                                                  \\ \hline
    \endfoot

    \hline \hline
    \endlastfoot
    Enfoque                                        & Iterativo e incremental                                                                                & Iterativo e incremental                                                                        & Prototipado                                                                                & Continuo                                                                                \\
    Principios                                     & Integración continua, programación en pares, desarrollo basado en pruebas, comentarios de los clientes & Centrarse en el valor, eliminar desperdicios, flujo, mejora continua, respeto por las personas & Desarrollo rápido, participación del usuario, desarrollo iterativo, creación de prototipos & Visualización del flujo de trabajo, limitación del trabajo en progreso, mejora continua \\
    Tamaño del equipo                              & 3 a 5                                                                                                  & 2 a 3                                                                                          & 2 a 3                                                                                      & Indefinido                                                                              \\
    Tamaño del proyecto                            & Pequeños y medianos                                                                                    & Grandes                                                                                        & Pequeños y medianos                                                                        & Pequeños, medianos y grandes                                                            \\
    Ventajas                                       & Calidad y comunicación                                                                                 & Velocidad y flexibilidad                                                                       & Costo y tiempo                                                                             & Mejor flujo de trabajo                                                                  \\
    Simplicidad                                    & Simplicidad en el código, diseño y solución de problemas                                               & Se busca eliminar complejidad innecesaria con el fin de mejorar la eficiencia                  & Soluciones simples y rápidas, evitando excesos en el diseño                                & Simplifica procesos y aporta claridad en el flujo de trabajo                            \\
    Entrega de software                            & Frecuente y regular                                                                                    & Incremental                                                                                    & Entrega rápida                                                                             & Simplifica procesos y aporta claridad en el flujo de trabajo                            \\
    Planificación                                  & Continua                                                                                               & Planificación y modelado                                                                       & Rápida y flexible                                                                          & Continua y visual                                                                       \\
\end{longtable}


% \section{Análisis y selección de herramientas de desarrollo}

% Mediante el análisis realizo en la Tabla \ref{tab:metodologias-agiles}, se optó por emplear la metodología de
% Desarrollo Rápido de Aplicaciones (RAD) debido a su enfoque centrado en la rápida entrega de soluciones y
% prototipos funcionales a los clientes. La selección de RAD se basa en su capacidad para proporcionar resultados
% en plazos cortos gracias a su naturaleza iterativa e incremental.

% \section{Análisis y selección del framework de desarrollo para el servidor web (Backend)}

% Como lenguaje de desarrollo para el servidor web (Backend) se optó por utilizar Typescript junto al entorno de
% ejecución de Node.js, debido que este permite el desarrollo de aplicaciones escalables y de alto rendimiento,
% impulsado por eventos asíncronos, ademas del amplio soporte por parte de la comunidad y documentación disponible,
% así como a la facilidad de uso y la gran cantidad de librerías y frameworks disponibles para el desarrollo de aplicaciones
% web \cite{haroDesarrolloBackendPara}.
% \bigbreak
% Como framework de desarrollo para el servidor web (backend), se optó por el uso de NestJS debido a que proporciona
% una arquitectura modular y escalable basada en el patrón de diseño de inyección de dependencias. Además, combina
% elementos de la programación orientada a objetos (POO), programación funcional (PF) y programación funcional
% reactiva (PFR). NestJS permite utilizar como base dos de los frameworks más populares en el desarrollo web, Express y
% Fastify, mediante un nivel de abstracción superior que permite exponer las APIs de ambos frameworks de forma
% directa al desarrollador, lo que proporciona una mayor flexibilidad al momento de incluir paquetes de terceros.
% Además, destaca por la gran cantidad de módulos y librerías que posee \cite{phamDEVELOPINGBACKENDWEB2020}.
% \bigbreak
% En la tabla \ref{tab:frameworks-backend} se presenta una comparación entre Express y Fastify tomando en cuenta
% criterios como el soporte para Typescript, rendimiento, velocidad, documentación, soporte por la comunidad y
% paquetes/librerías.

% \begin{longtable}{|p{5cm}|p{5cm}|p{5cm}|}
    \caption[]{Análisis y comparación entre los frameworks de NodeJS Express y Fastify \cite{ExpressInfraestructuraAplicaciones}\cite{FastLowOverhead}} \label{tab:frameworks-backend}                                                                                                              \\

    \hline \multicolumn{1}{|c|}{\textbf{Criterio}} & \multicolumn{1}{|c|}{\textbf{Express.js}}                                                                                      & \multicolumn{1}{|c|}{\textbf{Fastify.js}}                                                                   \\ \hline
    \endfirsthead

    \multicolumn{3}{c}%
    {{\normalfont \tablename\ \thetable{} -- continuación de la página anterior}}                                                                                                                                                                                                                 \\
    \hline \multicolumn{1}{|c|}{\textbf{Criterio}} & \multicolumn{1}{|c|}{\textbf{Express.js}}                                                                                      & \multicolumn{1}{|c|}{\textbf{Fastify.js}}                                                                   \\ \hline
    \endhead

    \hline \multicolumn{3}{|r|}{{Continua en la siguiente página}}                                                                                                                                                                                                                                \\ \hline
    \endfoot

    \hline \hline
    \endlastfoot
    Soporte para Typescript                        & Mediante un paquete externo                                                                                                    & Hecho en TypeScript con soporte directo                                                                     \\
    Rendimiento                                    & 11080 peticiones/s                                                                                                             & 45871 peticiones/s                                                                                          \\
    Velocidad                                      & Más lento debido a su mayor cantidad de middleware y flexibilidad                                                              & Significativamente más rápido debido a su enfoque en la velocidad y eficiencia                              \\
    Documentación                                  & Documentación detallada y ampliamente utilizada                                                                                & Documentación completa y fácil de entender                                                                  \\
    Soporte por la comunidad                       & Gran soporte por parte de la comunidad al ser el framework mas usado                                                           & Cuenta con un soporte amplio por parte de su comunidad en creciente aumento                                 \\
    Paquetes/librerías                             & Permite integrar fácilmente tanto librerías propias como de terceros gracias a su arquitectura flexible y su amplia comunidad. & Tine gran soporte para incorporar librerías personalizadas y de terceros gracias a su creciente popularidad \\
\end{longtable}

% Considerando el análisis presentado en la Tabla \ref{tab:frameworks-backend}, se optó por utilizar Fastify como
% el framework base para NestJS. Esto se debe a que Fastify es más rápido que Express, gracias a su enfoque en la
% velocidad y la eficiencia. Fastify ha sido diseñado específicamente con el objetivo de ser rápido y eficiente,
% lo que lo convierte en una excelente elección para aplicaciones que necesitan un alto rendimiento.

% \section{Análisis y selección del framework de desarrollo para el cliente web (Frontend)}

% Dado que en el lado del servidor se ha optado por utilizar TypeScript, para el cliente web (frontend) se ha optado
% por emplear el mismo lenguaje de programación. Esto se debe a que permite una integración más fluida entre el
% cliente y el servidor, además de facilitar la comunicación entre ambos. Para el desarrollo del cliente web
% (frontend), se han considerado tres de los frameworks más populares en la actualidad: Angular, ReactJS y VueJS.
% A continuación, se presenta una comparación de estos frameworks en la Tabla \ref{tab:frameworks-web}.

% \begin{longtable}{|p{5cm}|p{3cm}|p{3cm}|p{3cm}|}
    \caption[]{Análisis y comparación entre los frameworks Angular, ReactJS, VueJS  \cite{cincovicComparisonAngularVs2020}} \label{tab:frameworks-web}                      \\

    \hline \multicolumn{1}{|c|}{\textbf{Criterio}} & \multicolumn{1}{|c|}{\textbf{Angular}} & \multicolumn{1}{|c|}{\textbf{ReactJS}} & \multicolumn{1}{|c|}{\textbf{VueJS}} \\ \hline
    \endfirsthead

    \multicolumn{4}{c}%
    {{\normalfont \tablename\ \thetable{} -- continuación de la página anterior}}                                                                                           \\
    \hline \multicolumn{1}{|c|}{\textbf{Criterio}} & \multicolumn{1}{|c|}{\textbf{Angular}} & \multicolumn{1}{|c|}{\textbf{ReactJS}} & \multicolumn{1}{|c|}{\textbf{VueJS}} \\ \hline
    \endhead

    \hline \multicolumn{4}{|r|}{{Continua en la siguiente página}}                                                                                                          \\ \hline
    \endfoot

    \hline \hline
    \endlastfoot
    Popularidad                                    & Estancada                              & Creciente                              & Creciente                            \\
    Rendimiento                                    & Mayor sobrecarga                       & Ligero                                 & Ligero                               \\
    Soporte de la Comunidad                        & Mediano                                & Grande                                 & Grande                               \\
    Curva de Aprendizaje                           & Mediana                                & Mediana                                & Baja                                 \\
    Conocimientos necesarios                       & TypeScript                             & JSX, TSX, Hooks                        & Ninguno                              \\
    Migraciones                                    & Frecuentes                             & Raras                                  & Fácilmente adaptable                 \\
    Flexibilidad                                   & Limitada                               & Grande                                 & Grande                               \\
\end{longtable}

% \begin{longtable}{|p{0.5cm}|p{6cm}|p{6cm}|}
    \caption[]{Ventajas y desventajas de Angular, ReactJS y VueJS \cite{xingResearchAnalysisFrontend2019a}} \label{tab:ventajas-desventajas-frameworks-web}                                                                                                                                                                                                                                                                                                                          \\

    \hline \multicolumn{1}{|c|}{\textbf{Framework}} & \multicolumn{1}{|c|}{\textbf{Ventajas}}                                                                                                                                                                           & \multicolumn{1}{|c|}{\textbf{Desventajas}}                                                                                                                                                                 \\ \hline
    \endfirsthead

    \multicolumn{3}{c}%
    {{\normalfont \tablename\ \thetable{} -- continuación de la página anterior}}                                                                                                                                                                                                                                                                                                                                                                                                    \\
    \hline \multicolumn{1}{|c|}{\textbf{Framework}} & \multicolumn{1}{|c|}{\textbf{Ventajas}}                                                                                                                                                                           & \multicolumn{1}{|c|}{\textbf{Desventajas}}                                                                                                                                                                 \\ \hline
    \endhead

    \hline \multicolumn{3}{|r|}{{Continua en la siguiente página}}                                                                                                                                                                                                                                                                                                                                                                                                                   \\ \hline
    \endfoot

    \hline \hline
    \endlastfoot
    ReactJS                                         & \tabitem{Emplea un DOM virtual para lograr una eficiencia máxima al actualizar nodos según sea necesario.}                                                                                                        & \tabitem{Es necesario importar bibliotecas adicionales para manejar el estado y el modelo, ya que React no incluye la arquitectura MVC de forma nativa.}                                                   \\
                                                    & \tabitem{La capacidad de renderizar en el servidor es otra ventaja que este framework ofrece, especialmente adecuada para ciertos tipos de implementaciones, como las aplicaciones enfocadas en el contenido.}    & \tabitem{Aunque React permite su uso, se distancia de los enfoques basados en clases y puede presentar dificultades para aquellos que prefieren la Programación Orientada a Objetos (POO).}                \\
                                                    & \tabitem{Reduce la carga de recursos del usuario mediante el respaldo de bundling y tree shaking.}                                                                                                                &                                                                                                                                                                                                            \\
                                                    & \tabitem{La programación funcional facilita la creación de código que puede ser reutilizado. }                                                                                                                    &                                                                                                                                                                                                            \\
                                                    & \tabitem{Ofrece ventajas en términos de SEO en comparación con Angular y Vue.js. }                                                                                                                                &                                                                                                                                                                                                            \\
    Angular                                         & \tabitem{Utiliza el patrón MVVM (Modelo-Vista-Modelo de Vista), el cual permite manipular la misma colección de datos de forma independiente dentro de una misma aplicación.}                                     & \tabitem{Posee múltiples estructuras como Inyectables, Componentes, Tuberías, Módulos, entre otros, que suelen presentar un mayor nivel de complejidad para su comprensión.}                               \\
                                                    & \tabitem{Su estructura y arquitectura están diseñadas específicamente para mejorar la escalabilidad de los proyectos.}                                                                                            & \tabitem{Experimenta actualizaciones continuas, incorporando mejoras nuevas y significativas de manera constante. Sin embargo, estas actualizaciones pueden plantear desafíos al adaptarse a los cambios.} \\
                                                    & \tabitem{La inyección de dependencias en los componentes ayuda a mejorar la modularidad de la aplicación.}                                                                                                        &                                                                                                                                                                                                            \\
                                                    & \tabitem{La programación funcional facilita la creación de código que puede ser reutilizado. }                                                                                                                    &                                                                                                                                                                                                            \\
    VueJS                                           & \tabitem{Facilita la creación de modelos modulares de gran alcance que pueden renderizarse eficientemente gracias a su estructura fundamental, sin requerir esfuerzos adicionales.}                               & \tabitem{Su participación en el mercado es moderada, lo que indica que el intercambio de información en esta plataforma se encuentra en sus primeras fases de desarrollo.}                                 \\
                                                    & \tabitem{Posee una alta capacidad de respuesta y ofrece una sencilla vinculación de datos entre el código HTML y JavaScript. }                                                                                    & \tabitem{Existe el riesgo de que su flexibilidad pueda ser un problema al integrarse en proyectos extensos debido a la carencia de recursos disponibles.}                                                  \\
                                                    & \tabitem{Vue gestiona de forma sobresaliente la vinculación de datos bidireccional dinámica. Además, lleva a cabo la manipulación del DOM de manera coherente, lo que lo hace ideal para diversas aplicaciones. } &                                                                                                                                                                                                            \\
\end{longtable}


% Tomando en cuenta el análisis de las características de cada framework, así como sus ventajas y desventajas, se
% optó por el uso de ReactJS para el desarrollo del cliente web (frontend). Esto se debe a que ReactJS es una de
% las librerías más populares y ampliamente utilizadas en la actualidad, lo que significa que cuenta con una gran
% comunidad de desarrolladores y una amplia variedad de recursos disponibles. Además, ReactJS posee un gran
% rendimiento y eficiencia gracias a su manejo eficiente del DOM virtual, lo que lo convierte en una excelente
% elección para aplicaciones web que requieren una interfaz de usuario rápida y receptiva, as���� como su capacidad
% de crear componentes reutilizables, lo que facilita el desarrollo de aplicaciones web complejas y escalables.

% \section{Análisis y selección del framework de desarrollo para el cliente móvil (Frontend)}

% Para el desarrollo del cliente móvil (frontend), se han considerado tres de los frameworks más populares en la
% actualidad: React Native, Flutter y Ionic. A continuación, se presenta una comparación de estos frameworks en la
% Tabla \ref{tab:frameworks-movil}.

% \begin{longtable}{|p{5cm}|p{3cm}|p{3cm}|p{3cm}|}
    \caption[Análisis y comparación entre los frameworks React Native, Flutter y Xamarin]{Análisis y comparación entre los frameworks React Native, Flutter y Xamarin \cite{alferezzamoraEstudioComparativoFrameworks2018}\cite{lazoANALISISDISENOAPLICATIVO}} \label{tab:frameworks-movil}                                                                    \\

    \hline \multicolumn{1}{|c|}{\textbf{Criterio}} & \multicolumn{1}{|c|}{\textbf{React Native}}               & \multicolumn{1}{|c|}{\textbf{Flutter}}                                                                         & \multicolumn{1}{|c|}{\textbf{Xamarin}}                                                                                       \\ \hline
    \endfirsthead

    \multicolumn{4}{c}%
    {{\normalfont \tablename\ \thetable{} -- continuación de la página anterior}}                                                                                                                                                                                                                                                                              \\
    \hline \multicolumn{1}{|c|}{\textbf{Criterio}} & \multicolumn{1}{|c|}{\textbf{React Native}}               & \multicolumn{1}{|c|}{\textbf{Flutter}}                                                                         & \multicolumn{1}{|c|}{\textbf{Xamarin}}                                                                                       \\ \hline
    \endhead

    \hline \multicolumn{4}{|r|}{{Continua en la siguiente página}}                                                                                                                                                                                                                                                                                             \\ \hline
    \endfoot

    \hline \hline
    \endlastfoot
    Lanzamiento                                    & 2015                                                      & 2017                                                                                                           & 2013                                                                                                                         \\\hline
    Licencia                                       & MIT Licensed                                              & BSD                                                                                                            & MIT Licensed                                                                                                                 \\\hline
    Lenguaje                                       & JavaScript, TypeScript                                    & Dart                                                                                                           & C\#                                                                                                                          \\\hline
    Plataformas soportadas                         & Android, IOS                                              & Android, IOS                                                                                                   & Android, IOS                                                                                                                 \\\hline
    Open source                                    & Si                                                        & Si                                                                                                             & Si                                                                                                                           \\\hline
    Paradigma                                      & Declarativo                                               & Declarativo                                                                                                    & Imperativo                                                                                                                   \\\hline
    Recarga en tiempo real                         & Si                                                        & Si                                                                                                             & No                                                                                                                           \\\hline
    Recarga en vivo                                & Si                                                        & Si                                                                                                             & No                                                                                                                           \\\hline
    Administrador de paquetes                      & NPM, Yarn                                                 & Pub                                                                                                            & NuGet                                                                                                                        \\\hline
    Enfoque multiplataforma                        & Interpretado                                              & Compilado a nativo                                                                                             & Compilado a nativo                                                                                                           \\\hline
    Compatibilidad con 64bits                      & No en android                                             & Si                                                                                                             & Si                                                                                                                           \\\hline
    Portabilidad                                   & Si                                                        & No                                                                                                             & No                                                                                                                           \\\hline
    Geolocalización                                & Incluido                                                  & Mediante paquete de la comunidad                                                                               & Incluido                                                                                                                     \\\hline
    Notificaciones                                 & Mediante paquete de la comunidad                          & Mediante paquete de la comunidad                                                                               & Mediante paquete de la comunidad                                                                                             \\\hline
    Rendimiento                                    & Alto desempeño al ser un framework ligero                 & Elevada eficiencia gracias a su propio motor de renderizado                                                    & Un desempeño sólido, aunque podría beneficiarse de una capa extra de abstracción.                                            \\\hline
    Integración con APIs y Bibliotecas             & Gran cantidad de paquetes de terceros                     & Una amplia disponibilidad de bibliotecas y paquetes de terceros gracias a su comunidad en constante desarrollo & Gran cantidad de bibliotecas y recursos disponibles, especialmente diseñados para integrarse con los servicios de Microsoft. \\\hline
    Documentación                                  & Amplia documentación                                      & Documentación amplia y detallada, además  mas de gran cantidad de recursos de terceros                         & Documentación completa y abundante                                                                                           \\\hline
    Tiempo de desarrollo                           & Rápida gracias a su hot reload con cambios en tiempo real & Rápido gracias al hot reload que permite visualizar cambios en tiempo real                                     & Tiempo de desarrollo mas lento debido a la compilación                                                                       \\
\end{longtable}

% Considerando el análisis realizado en la Tabla \ref{tab:frameworks-movil}, se optó por utilizar Flutter para el
% desarrollo del cliente móvil (frontend). Esto se debe a que Flutter es un framework de desarrollo de aplicaciones
% móviles multiplataforma creado por Google, que permite el desarrollo de aplicaciones móviles nativas de alta
% calidad para Android e iOS desde una sola base de código. Flutter destaca por su rendimiento y eficiencia, gracias
% a su propio motor de renderizado, lo que lo convierte en una excelente elección para aplicaciones móviles que
% requieren un alto rendimiento y una interfaz de usuario rápida y receptiva. Además, Flutter cuenta con una amplia
% variedad de widgets y herramientas que facilitan el desarrollo de aplicaciones móviles complejas y escalables.

% \section{Análisis y selección de la base de datos}

% Pra la base de datos del proyecto se han considerado dos opciones: MySQL y PostgreSQL. A continuación, se presenta
% una comparación de estas bases de datos en la Tabla \ref{tab:bases-datos}.

% \begin{longtable}{|p{5cm}|p{5cm}|p{5cm}|}
    \caption[]{Análisis y comparación entre las bases de datos MySQL y PostgreSQL\cite{lazoANALISISDISENOAPLICATIVO}} \label{tab:bases-datos}                                                                    \\

    \hline \multicolumn{1}{|c|}{\textbf{Criterio}} & \multicolumn{1}{|c|}{\textbf{MySQL}}                                              & \multicolumn{1}{|c|}{\textbf{PostgreSQL}}                               \\ \hline
    \endfirsthead

    \multicolumn{3}{c}%
    {{\normalfont \tablename\ \thetable{} -- continuación de la página anterior}}                                                                                                                                \\
    \hline \multicolumn{1}{|c|}{\textbf{Criterio}} & \multicolumn{1}{|c|}{\textbf{MySQL}}                                              & \multicolumn{1}{|c|}{\textbf{PostgreSQL}}                               \\ \hline
    \endhead

    \hline \multicolumn{3}{|r|}{{Continua en la siguiente página}}                                                                                                                                               \\ \hline
    \endfoot

    \hline \hline
    \endlastfoot
    GUI                                            & MySQL Workbench                                                                   & pgAdmin                                                                 \\
    Consumo de recursos                            & Consumo mayor de CPU y memoria                                                    & Consumo mayor de CPU y memoria                                          \\
    Tiempo de respuesta de CRUD                    & Bajo                                                                              & Alto                                                                    \\
    Lenguaje de ejecución                          & C/C++                                                                             & C                                                                       \\
    herencia de tablas                             & No                                                                                & Si                                                                      \\
    Tipos de datos                                 & Solo tipos estándar                                                               & Estándar, store, arreglos, geográficos, definido por el usuario, etc.   \\
    APIs y otros métodos de acceso                 & ADO.NET, JDBC, biblioteca C nativa, ODBC, API de transmisión para objetos grandes & ADO.NET, JDBC, ODBC, API nativa patentada                               \\
    Tipos de conexión                              & Las conexiones son subprocesos del sistema operativo                              & Las conexiones son subprocesos del sistema operativo                    \\
    Respaldo                                       & En MySQL, Mysqldump y XtraBackup, las herramientas proporcionan respaldo          & PostgreSQL proporciona una copia de seguridad completa en línea.        \\
    Lenguajes soportados                           & C/C++, PHP, Java, Go, Delphi, Lisp, Erlang, Node.js, R, Perl, PHP                 & Go, C/C++, Java, Delphi, Javascript, Erlang, Lisp, R, .Net, Tcl, Python \\
    Sistemas Operativos Soportados                 & Windows, Linux, macOS, Oracle Solaris, Fedora, FreeBSD                            & Windows, macOS, BSD, Linux, Solaris                                     \\
\end{longtable}

% Tomando en cuenta el análisis presentado en la Tabla \ref{tab:bases-datos}, se optó por utilizar PostgreSQL como
% la base de datos del proyecto. Esto se debe a que PostgreSQL es un sistema de gestión de bases de datos relacional
% de código abierto y de alto rendimiento, que ofrece una amplia variedad de características y funcionalidades
% avanzadas, como soporte para tipos de datos geoespaciales, herencia de tablas, tipos de datos definidos por el
% usuario, entre otros. Además, PostgreSQL cuenta con una arquitectura robusta y escalable, que permite gestionar
% grandes volúmenes de datos de forma eficiente y fiable, lo que lo convierte en una excelente elección para
% aplicaciones que requieren una base de datos potente y confiable.

% \section{Análisis de la precisión de los datos georreferenciados}

% Con la finalidad de evaluar la precisión de los datos georreferenciados obtenidos a través de la aplicación móvil, se
% desarrolló un prototipo de aplicación que permite obtener la ubicación del dispositivo móvil y compararla con un punto
% de referencia conocido, el cual será seleccionado en la aplicación de prueba. Para ello, se utilizó el GPS del dispositivo
% móvil para obtener la ubicación, el framework de desarrollo Flutter para la creación de la aplicación móvil y Google Maps
% para la visualización de la ubicación en un mapa interactivo.
% \bigbreak

% En la Figura  \ref{fig:prototipo-georreferenciacion} se muestra el prototipo de la aplicación móvil desarrollada para la
% evaluación de la precisión de los datos georreferenciados. Esta aplicación cuenta con un botón que permite obtener la ubicación
% actual del dispositivo móvil. Con esto, el usuario puede seleccionar un punto de referencia en el mapa y guardar ambos puntos
% para su posterior comparación.

% \begin{figure}[H]
%     \centering
%     \includegraphics[width=0.3\textwidth]{chapters/III-resultados-y-discusion/resources/images/prototipo-georreferenciacion.png}
%     \caption{Prototipo de la aplicación móvil para la evaluación de la precisión de los datos georreferenciados.}
%     \label{fig:prototipo-georreferenciacion}
% \end{figure}

% Para establecer la cantidad de datos necesarios para la evaluación de la precisión de los datos georreferenciados, se optó por
% utilizar una muestra infinita con población desconocida, aplicando la Ecuación \ref{eq:ecuacion-muestra-datos-georreferenciados}, donde:

% \begin{itemize}
%     \item n = tamaño de la muestra
%     \item Z = nivel de confianza
%     \item p = probabilidad de éxito o proporción esperada
%     \item q = probabilidad de fracaso
%     \item pq = varianza de la población
%     \item e = error de estimación máximo aceptable
% \end{itemize}

% \begin{equation}
%     n=\frac{Z^2 \cdot p \cdot q}{e^2}
%     \label{eq:ecuacion-muestra-datos-georreferenciados}
% \end{equation}

% Sustituyendo los valores en la Ecuación \ref{eq:ecuacion-muestra-datos-georreferenciados}, se obtiene:

% \begin{itemize}
%     \item Z = 1.96, con un nivel de confianza del 95\% y un error de estimación máximo aceptable del 5\%
%     \item p = 0.50
%     \item q = 0.50
%     \item e = 0.05
% \end{itemize}

% \begin{equation}
%     n=\frac{(1.96)^2 \cdot 0.50 \cdot 0.50}{(0.05)^2}
%     \label{eq:ecuacion-valores-muestra-datos-georreferenciados}
% \end{equation}

% \begin{equation}
%     n=384.16 \approx 385
%     \label{eq:ecuacion-resultado-muestra-datos-georreferenciados}
% \end{equation}

% Utilizando un nivel de confianza del 95\% y un error de estimación máximo aceptable del 5\%, se obtiene una muestra aproximada
% de 385 puntos de referencia para la evaluación de la precisión de los datos georreferenciados.
% \bigbreak

% Los datos obtenidos de la aplicación móvil se almacenaron en una tabla de una base de datos PostgreSQL, la cual se muestra en
% la Figura \ref{fig:tabla-georreferenciacion}. Esta tabla contiene la información de los puntos de referencia seleccionados en
% la aplicación móvil, los cuales son: latitud y longitud real (Punto de referencia) y latitud y longitud obtenida mediante
% GPS (Punto de ubicación).

% \begin{figure}[H]
%     \centering
%     \includegraphics[width=0.3\textwidth]{chapters/III-resultados-y-discusion/resources/images/tabla-georreferenciacion.png}
%     \caption{Tabla de datos georreferenciados almacenados en la base de datos PostgreSQL.}
%     \label{fig:tabla-georreferenciacion}
% \end{figure}

% Estos datos se traspasaron a un archivo de Excel para su posterior análisis. Para ello, se utilizó un script con
% Node.js y la librería de manejo de archivos ExcelJS, como se muestra en el Anexo \ref{apendix:script-exceljs}. La distancia entre
% los puntos de referencia y los puntos de ubicación se calculó utilizando la fórmula de Haversine, dado que esta toma en cuenta
% la curvatura de la Tierra y proporciona una mejor precisión \cite{basyirDeterminationNearestEmergency2017}. La función de
% Haversine utilizada se muestra en el Anexo \ref{apendix:script-haversine}.

% % \begin{figure}[H]
% %     \begin{minted}[fontsize=\footnotesize, linenos, breaklines, frame=single]{js}
import Exceljs from "exceljs";
import distances from "./distances.json" with { type: "json" };
import { haversineDistance } from "./haversineDistance.js";
import { roundDecimal } from "./lib/roundDecimal.js";
const distancesArray = [];
distances.forEach((d) => {
    const center = { lat: d.realLatitude, lng: d.realLongitude };
    const location = { lat: d.latitude, lng: d.longitude };
    const distance = haversineDistance(center, location);
    distancesArray.push({ distance: roundDecimal(distance) });
});
const newWorkbook = new Exceljs.Workbook();
const sheet = newWorkbook.addWorksheet("geo-distances");
sheet.addRow(["#", "Latitude", "Longitude", "Haversine distance from center (m)"]);
distancesArray.forEach((distance, index) => {
    sheet.addRow([index + 1, distances[index].latitude, distances[index].longitude, distance.distance]);
});
newWorkbook.xlsx.writeFile("./geo-locations.xlsx");
\end{minted}


% %     \caption{Script en Node.js para la comparación de los datos georreferenciados.}
% %     \label{apendix:script-exceljs}
% % \end{figure}

% % \begin{figure}[H]
% %     \begin{minted}[fontsize=\footnotesize, linenos, breaklines, frame=single]{js}
function haversineDistance(point1, point2) {
  const R = 6378137; // Radius of the Earth in meters
  const rlat1 = point1.lat * (Math.PI / 180); // Convert degrees to radians
  const rlat2 = point2.lat * (Math.PI / 180); // Convert degrees to radians
  const difflat = rlat2 - rlat1; // Radian difference (latitudes)
  const difflon = (point2.lng - point1.lng) * (Math.PI / 180); // Radian difference (longitudes)
  const d = 2 * R * Math.asin( Math.sqrt(
        Math.sin(difflat / 2) * Math.sin(difflat / 2) +
        Math.cos(rlat1) *
        Math.cos(rlat2) *
        Math.sin(difflon / 2) *
        Math.sin(difflon / 2)
      )
    );
  return d;
}
\end{minted}
% %     \caption{Función de Haversine para el cálculo de la distancia entre dos puntos georreferenciados.}
% %     \label{apendix:script-haversine}
% % \end{figure}

% El archivo de Excel generado con los datos georreferenciados se muestra en la Figura \ref{fig:archivo-excel-georreferenciacion}. El cual
% contiene la longitud obtenida mediante GPS y la distancia entre la ubicación del dispositivo y el punto de referencia seleccionado.

% \begin{figure}[H]
%     \centering
%     \includegraphics[width=0.5\textwidth]{chapters/III-resultados-y-discusion/resources/images/archivo-excel-georreferenciacion.png}
%     \caption{Archivo de Excel con los datos georreferenciados para la comparación.}
%     \label{fig:archivo-excel-georreferenciacion}
% \end{figure}

% Utilizando estos datos se creo un cluster de puntos georreferenciados en python, para ello se utilizó el algoritmo de K-Means, el cual
% permite agrupar los puntos en clusters en función de su distancia. El código utilizado para la creación del cluster se muestra en el
% Anexo \ref{apendix:script-cluster-python}. Utilizando el método del codo se determinó el número óptimo de clusters, el cual fue de 2 como
% se muestra en la Figura \ref{fig:metodo-del-codo} en done se puede observar que el codo se encuentra en el punto 2.

% % \begin{figure}[H]
% %     \begin{minted}[fontsize=\footnotesize, linenos, breaklines, frame=single]{python}
import pandas as pd
import matplotlib.pyplot as plt
from sklearn.cluster import KMeans
import pygments as pg

df = pd.read_excel('geo-locations.xlsx', usecols='B:D', skiprows=0)
df.columns = ['latitude', 'longitude', 'distance']

# Elbow Method
def elbow_method(data):
    wcss = []
    for i in range(1, 11):
        kmeans = KMeans(n_clusters=i, max_iter=300)
        kmeans.fit(df[['latitude', 'longitude']])
        wcss.append(kmeans.inertia_)
    plt.plot(range(1, 11), wcss)
    plt.title('Método del Codo')
    plt.xlabel('Número de clusters')
    plt.ylabel('WCSS (Within Cluster Sum of Squares)')
    plt.show()

elbow_method(df)

# Cluster optimal number
numero_optimo_clusters = 2

# apply K-Means
kmeans = KMeans(n_clusters=numero_optimo_clusters, max_iter=300, random_state=42)
kmeans.fit(df[['latitude', 'longitude']])
df['cluster'] = kmeans.labels_

# Calc the mean distance of each cluster
distancias_medias_clusters = df.groupby('cluster')['distance'].mean()
print("Distancias medias de cada cluster:")
print(distancias_medias_clusters)

# Calc the mean of the mean distances of the clusters
media_de_las_medias = round(distancias_medias_clusters.mean(), 2)
print("Media de las distancias medias de los clusters:")
print(media_de_las_medias)

# Graph
plt.scatter(df['latitude'], df['longitude'], c=df['cluster'], cmap='viridis')
plt.title('Clusters de Puntos GPS')
plt.xlabel('Latitud')
plt.ylabel('Longitud')
plt.colorbar(label='Cluster')
plt.show()

\end{minted}


% %     \caption{Script en Python para la creación de clusters de puntos georreferenciados.}
% %     \label{apendix:script-cluster-python}
% % \end{figure}

% \begin{figure}[H]
%     \centering
%     \includegraphics[width=0.5\textwidth]{chapters/III-resultados-y-discusion/resources/images/metodo-del-codo.png}
%     \caption{Método del codo para determinar el número óptimo de clusters.}
%     \label{fig:metodo-del-codo}
% \end{figure}

% El cluster resultante aplicando el algoritmo de K-Means se muestra en la Figura \ref{fig:cluster-georreferenciacion}.

% \begin{figure}[H]
%     \centering
%     \includegraphics[width=0.5\textwidth]{chapters/III-resultados-y-discusion/resources/images/cluster-georreferenciacion.png}
%     \caption{Cluster de puntos georreferenciados obtenidos mediante el algoritmo de K-Means.}
%     \label{fig:cluster-georreferenciacion}
% \end{figure}

% Posteriormente se calculó la distancia media de cada cluster, así como la media de las distancias medias de los clusters. Los resultados
% se muestran a continuación:

% \begin{itemize}
%     \item Distancias medias de cada cluster:
%           \begin{itemize}
%               \item Cluster 1: 6.32 m
%               \item Cluster 2: 8.78 m
%           \end{itemize}
%     \item Media de las distancias medias de los clusters: 7.55 m
% \end{itemize}

% Estos resultados indican que la precisión media de los cluster es de 7.55 metros, Con la finalidad de obtener el intervalo de
% confianza para esta media, se utilizó la Ecuación \ref{eq:ecuacion-intervalo-confianza}, donde:

% \begin{itemize}
%     \item IC = Intervalo de confianza
%     \item M = Media de las distancias medias de los clusters
%     \item ME = Margen de error
% \end{itemize}

% \begin{equation}
%     IC = M \pm ME
%     \label{eq:ecuacion-intervalo-confianza}
% \end{equation}

% El margen de error se calculó utilizando la Ecuación \ref{eq:ecuacion-margen-error}, donde:

% \begin{itemize}
%     \item ME = Margen de error
%     \item Z = Valor crítico de la distribución normal estándar
%     \item SE = Error estándar
% \end{itemize}

% \begin{equation}
%     ME = Z \times SE
%     \label{eq:ecuacion-margen-error}
% \end{equation}

% El error estándar se calculó utilizando la Ecuación \ref{eq:ecuacion-error-estandar}, donde:

% \begin{itemize}
%     \item SE = Error estándar
%     \item $\sigma$ = Desviación estándar
%     \item k = Tamaño de la muestra
% \end{itemize}

% \begin{equation}
%     SE = \frac{\sigma}{\sqrt{k}}
%     \label{eq:ecuacion-error-estandar}
% \end{equation}


% Sustituyendo los valores en las Ecuaciones \ref{eq:ecuacion-error-estandar} y \ref{eq:ecuacion-margen-error}, se obtiene:

% \begin{itemize}
%     \item Z = 1.96, con un nivel de confianza del 95\%
%     \item $\sigma$ = 4.77
%     \item k = 385
% \end{itemize}

% \begin{equation}
%     SE = \frac{4.77}{\sqrt{385}} = 0.24
%     \label{eq:ecuacion-resultado-error-estandar}
% \end{equation}

% \begin{equation}
%     ME = 1.96 \times 0.24 = 0.47
%     \label{eq:ecuacion-resultado-margen-error}
% \end{equation}

% Sustituyendo los valores en la Ecuación \ref{eq:ecuacion-intervalo-confianza} para el intervalo de confianza, se obtiene:

% \begin{itemize}
%     \item ME = 0.47
%     \item M = 7.55
% \end{itemize}

% \begin{equation}
%     IC = 7.55 \pm 0.47 = [7.08, 8.02]
%     \label{eq:ecuacion-resultado-intervalo-confianza}
% \end{equation}

% Por lo tanto, se tiene que para un nivel de confianza del 95\%, el error de estimación de la precisión de los datos georreferenciados
% se encuentra entre 7.08 y 8.02 metros con respecto al punto real de la ubicación.

% \subsection{Desarrollo de la propuesta}
% Para desarrollar la propuesta, se utilizó la metodología RAD, una metodología ágil para el desarrollo de software
% creada por James Martin en 1991 \cite{agrawalUSINGRAPIDAPPLICATION2019}. Esta metodología se compone de las
% siguientes cuatro fases:

% \begin{enumerate}
%     \item Planificación de requerimientos: En la fase de planificación de requerimientos, el usuario y el analista
%           se reúnen para definir el objetivo de la aplicación o sistema y determinar los requisitos de información
%           necesarios para alcanzar ese objetivo, el enfoque en esta etapa es resolver problemas empresariales \cite{maulanyDesignLearningApplications2021}.
%     \item Diseño de usuario: El diseño de la aplicación se basará en esta descripción y será beneficioso para
%           todos. La segunda fase de RAD implica la creación de diagramas ERD, UML y otros, así como el diseño de
%           la interfaz de usuario mediante prototipos \cite{maulanyDesignLearningApplications2021}.
%     \item Construcción: La etapa de construcción se enfoca en la programación y producción de código del sistema \cite{fauziSystematicLiteratureReviews2023}.
%     \item Cierre: La etapa de cierre consiste en la puesta a prueba de la aplicación \cite{fauziSystematicLiteratureReviews2023}.
% \end{enumerate}

% \subsection{Planificación de requerimientos}
% En esta fase, se identificaron los requerimientos del sistema a través de la recopilación de información de los
% usuarios y las entrevistas realizadas.
% \bigbreak

% A continuación, en la tabla \ref{tab:descripcion-usuarios}, se describen los usuarios y la forma en que interactúan con
% el sistema web y móvil.

% \label{app:descripcion-usuarios}
\begin{longtable}{|p{3cm}|p{10cm}|}
    \caption{Descripción de usuarios} \label{tab:descripcion-usuarios}                                                                                                                                                                                               \\

    \hline \multicolumn{1}{|c|}{\textbf{Usuario}} & \multicolumn{1}{|c|}{\textbf{Descripción}}                                                                                                                                                                       \\ \hline
    \endfirsthead

    \multicolumn{2}{c}%
    {{\normalfont \tablename\ \thetable{} -- continuación de la página anterior}}                                                                                                                                                                                    \\
    \hline \multicolumn{1}{|c|}{\textbf{Usuario}} & \multicolumn{1}{|c|}{\textbf{Descripción}}                                                                                                                                                                       \\ \hline
    \endhead

    \hline \multicolumn{2}{|r|}{{Continua en la siguiente página}}                                                                                                                                                                                                   \\ \hline
    \endfoot

    \hline \hline
    \endlastfoot
    Administrador del sistema                     &
    Usuario responsable de gestionar a otros usuarios, las zonas de vigilancia y los tipos de delitos, además de monitorear los incidentes delictivos, coordinar el despacho de las entidades correspondientes y tomar decisiones basadas en los reportes generados. \\\hline
    Usuario ciudadano                             & Usuario encargado de enviar alertas sobre incidentes delictivos.                                                                                                                                                 \\\hline
    Policía                                       & Usuario encargado de recibir y atender las alertas de incidentes delictivos.                                                                                                                                     \\
\end{longtable}

% En la tabla \ref{tab:requerimientos} se presentan los requerimientos definidos para el desarrollo del proyecto.

% \newcounter{reqcounter}
\setcounter{reqcounter}{1}

\begin{longtable}{|p{0.6cm}|p{2.5cm}|p{5.3cm}|c|c|}
    \caption{Definición de requerimientos} \label{tab:requerimientos}                                                                                                                                                                                                                                                                   \\

    \hline \multicolumn{1}{|c|}{\textbf{ID}}     & \multicolumn{1}{|c|}{\textbf{Requerimiento}}       & \multicolumn{1}{|c|}{\textbf{Descripción}}                                                                                                   & \multicolumn{1}{|c|}{\textbf{Prioridad}} & \multicolumn{1}{|c|}{\textbf{Riesgo}} \\ \hline
    \endfirsthead

    \multicolumn{5}{c}%
    {{\normalfont \tablename\ \thetable{} -- continuación de la página anterior}}                                                                                                                                                                                                                                                       \\
    \hline \multicolumn{1}{|c|}{\textbf{ID}}     & \multicolumn{1}{|c|}{\textbf{Requerimiento}}       & \multicolumn{1}{|c|}{\textbf{Descripción}}                                                                                                   & \multicolumn{1}{|c|}{\textbf{Prioridad}} & \multicolumn{1}{|c|}{\textbf{Riesgo}} \\ \hline
    \endhead

    \hline
    \multicolumn{5}{|c|}{{Continua en la siguiente página}}                                                                                                                                                                                                                                                                             \\
    \hline
    \endfoot

    \hline
    \endlastfoot

    \multicolumn{5}{|l|}{\textbf{Todos los usuarios de la aplicación web y móvil}}                                                                                                                                                                                                                                                      \\
    \hline
    R\arabic{reqcounter}\stepcounter{reqcounter} & Iniciar Sesión                                     & El inicio de sesión se realizará a través de la autenticación del usuario, quien deberá ingresar su correo y contraseña.                     & Alta                                     & Alto                                  \\
    \hline
    R\arabic{reqcounter}\stepcounter{reqcounter} & Cerrar Sesión                                      & El usuario podrá cerrar la sesión en cualquier momento.                                                                                      & Alta                                     & Bajo                                  \\
    \hline
    \multicolumn{5}{|l|}{\textbf{Administrador del sistema}}                                                                                                                                                                                                                                                                            \\
    \hline
    R\arabic{reqcounter}\stepcounter{reqcounter} & Gestionar usuarios                                 & El administrador podrá crear, visualizar, actualizar y deshabilitar usuarios                                                                 & Alta                                     & Alto                                  \\
    \hline
    R\arabic{reqcounter}\stepcounter{reqcounter} & Gestionar tipos de incidentes                      & El administrador podrá crear, visualizar, actualizar y deshabilitar tipos de incidentes                                                      & Alta                                     & Alto                                  \\
    \hline
    R\arabic{reqcounter}\stepcounter{reqcounter} & Gestionar zonas de vigilancia                      & El administrador podrá crear, visualizar, actualizar y deshabilitar tipos de incidentes                                                      & Alta                                     & Alto                                  \\
    \hline
    R\arabic{reqcounter}\stepcounter{reqcounter} & Asignar policías a las zonas de vigilancia         & El administrador podrá asignar y desasignar miembros de la policías a las zonas de vigilancia                                                & Alta                                     & Alto                                  \\
    \hline
    R\arabic{reqcounter}\stepcounter{reqcounter} & Gestionar alertas de incidentes                    & El administrador podrá visualizar mediante un mapa las alertas de emergencia enviadas por los ciudadanos así como su posición en tiempo real & Alta                                     & Alto                                  \\
    \hline
    R\arabic{reqcounter}\stepcounter{reqcounter} & Visualizar mapa de calor                           & El administrador podrá visualizar mediante un mapa de calor los incidentes delictivos suscitados en las diferentes zonas                     & Alta                                     & Bajo                                  \\
    \hline
    R\arabic{reqcounter}\stepcounter{reqcounter} & Visualizar reportes                                & El administrador podrá visualizar informes detallados sobre los incidentes ocurridos, generados mediante BI.                                 & Alta                                     & Alto                                  \\
    \hline
    \multicolumn{5}{|l|}{\textbf{Usuario ciudadano}}                                                                                                                                                                                                                                                                                    \\
    \hline
    R\arabic{reqcounter}\stepcounter{reqcounter} & Registro de usuario                                & Los usuarios deberán ingresar sus datos y fotografía mediante un formulario.                                                                 & Alta                                     & Alto                                  \\
    \hline
    R\arabic{reqcounter}\stepcounter{reqcounter} & Cambiar contraseña                                 & El usuario podrá cambiar su contraseña en cualquier momento.                                                                                 & Media                                    & Alto                                  \\
    \hline
    R\arabic{reqcounter}\stepcounter{reqcounter} & Recuperar contraseña                               & El usuario podrá recuperar su contraseña en caso de olvidarla mediante su correo electrónico.                                                & Baja                                     & Alto                                  \\
    \hline
    R\arabic{reqcounter}\stepcounter{reqcounter} & Asignar miembros al grupo familiar                 & El usuario podrá Asignar miembros al grupo familiar mediante la cédula de ciudadanía.                                                        & Alta                                     & Alto                                  \\
    \hline
    R\arabic{reqcounter}\stepcounter{reqcounter} & Enviar alertas de emergencia                       & El usuario podrá enviar alertas de emergencia seleccionando el tipo de incidente y oprimiendo un botón de pánico durante 3 segundos.         & Alta                                     & Alto                                  \\
    \hline
    R\arabic{reqcounter}\stepcounter{reqcounter} & Visualizar alertas de emergencia de familiares     & El usuario podrá visualizar mediante un mapa las alertas de emergencia enviadas por los sus familiares así como su posición en tiempo real.  & Alta                                     & Alto                                  \\
    \hline
    \multicolumn{5}{|l|}{\textbf{Policía}}                                                                                                                                                                                                                                                                                              \\
    \hline
    R\arabic{reqcounter}\stepcounter{reqcounter} & Visualizar alertas de emergencia de los ciudadanos & El policía podrá visualizar mediante un mapa las alertas de emergencia enviadas por los ciudadanos así como su posición en tiempo real       & Alta                                     & Alto                                  \\
    \hline
\end{longtable}

% Una vez definidos los requerimientos necesarios, se elaboró el siguiente plan de iteraciones para estructurar y gestionar
% el desarrollo del proyecto.

% \newcounter{numcounter}
\newcounter{itcounter}
\setcounter{numcounter}{1}
\setcounter{itcounter}{1}
\setcounter{reqcounter}{1}

\begin{longtable}{|p{0.6cm}|p{0.6cm}|p{0.6cm}|p{5cm}|c|c|}
    \caption{Planificación de iteraciones} \label{tab:planificacion-iteraciones}                                                                                                                                                                                                                                        \\

    \hline \multicolumn{1}{|c|}{\textbf{Iteración}}                      & \multicolumn{1}{|c|}{\textbf{N.}}           & \multicolumn{1}{|c|}{\textbf{ID}}            & \multicolumn{1}{|c|}{\textbf{Requerimientos}}      & \multicolumn{1}{|c|}{\textbf{Tiempo horas}} & \multicolumn{1}{|c|}{\textbf{Estimado días}} \\ \hline
    \endfirsthead

    \multicolumn{6}{c}%
    {{\normalfont \tablename\ \thetable{} -- continuación de la página anterior}}                                                                                                                                                                                                                                       \\
    \hline \multicolumn{1}{|c|}{\textbf{Iteración}}                      & \multicolumn{1}{|c|}{\textbf{N.}}           & \multicolumn{1}{|c|}{\textbf{ID}}            & \multicolumn{1}{|c|}{\textbf{Requerimientos}}      & \multicolumn{1}{|c|}{\textbf{Tiempo horas}} & \multicolumn{1}{|c|}{\textbf{Estimado días}} \\ \hline
    \endhead

    \hline \multicolumn{6}{|r|}{{Continua en la siguiente página}}                                                                                                                                                                                                                                                      \\ \hline
    \endfoot

    \hline \hline
    \endlastfoot
    \multirow{4}{*}{Iteración \arabic{itcounter}\stepcounter{itcounter}} & \arabic{numcounter}\stepcounter{numcounter} & R\arabic{reqcounter}\stepcounter{reqcounter} & Iniciar sesión                                     & 6                                           & 1                                            \\\cline{2-6}
                                                                         & \arabic{numcounter}\stepcounter{numcounter} & R\arabic{reqcounter}\stepcounter{reqcounter} & Cerrar sesión                                      & 1                                           & 1                                            \\\cline{2-6}
                                                                         & \arabic{numcounter}\stepcounter{numcounter} & R\arabic{reqcounter}\stepcounter{reqcounter} & Registro de usuario                                & 12                                          & 2                                            \\\cline{2-6}
                                                                         & \arabic{numcounter}\stepcounter{numcounter} & R\arabic{reqcounter}\stepcounter{reqcounter} & Cambiar contraseña                                 & 2                                           & 1                                            \\\cline{2-6}
                                                                         & \arabic{numcounter}\stepcounter{numcounter} & R\arabic{reqcounter}\stepcounter{reqcounter} & Recuperar contraseña                               & 6                                           & 1                                            \\\hline
    \multirow{4}{*}{Iteración \arabic{itcounter}\stepcounter{itcounter}} & \arabic{numcounter}\stepcounter{numcounter} & R\arabic{reqcounter}\stepcounter{reqcounter} & Asignar miembros al grupo familiar                 & 6                                           & 1                                            \\\cline{2-6}
                                                                         & \arabic{numcounter}\stepcounter{numcounter} & R\arabic{reqcounter}\stepcounter{reqcounter} & Enviar alertas de emergencia                       & 18                                          & 3                                            \\\cline{2-6}
                                                                         & \arabic{numcounter}\stepcounter{numcounter} & R\arabic{reqcounter}\stepcounter{reqcounter} & Visualizar alertas de emergencia de familiares     & 12                                          & 2                                            \\\cline{2-6}
                                                                         & \arabic{numcounter}\stepcounter{numcounter} & R\arabic{reqcounter}\stepcounter{reqcounter} & Administrar usuarios                               & 6                                           & 1                                            \\\cline{2-6}
                                                                         & \arabic{numcounter}\stepcounter{numcounter} & R\arabic{reqcounter}\stepcounter{reqcounter} & Gestionar tipos de incidentes                      & 4                                           & 1                                            \\\hline
    \multirow{6}{*}{Iteración \arabic{itcounter}\stepcounter{itcounter}} & \arabic{numcounter}\stepcounter{numcounter} & R\arabic{reqcounter}\stepcounter{reqcounter} & Gestionar zonas de vigilancia                      & 18                                          & 3                                            \\\cline{2-6}
                                                                         & \arabic{numcounter}\stepcounter{numcounter} & R\arabic{reqcounter}\stepcounter{reqcounter} & Asignar policías a las zonas de vigilancia         & 5                                           & 1                                            \\\cline{2-6}
                                                                         & \arabic{numcounter}\stepcounter{numcounter} & R\arabic{reqcounter}\stepcounter{reqcounter} & Visualizar alertas de incidentes                   & 18                                          & 3                                            \\\cline{2-6}
                                                                         & \arabic{numcounter}\stepcounter{numcounter} & R\arabic{reqcounter}\stepcounter{reqcounter} & Visualizar mapas de calor                          & 6                                           & 1                                            \\\cline{2-6}
                                                                         & \arabic{numcounter}\stepcounter{numcounter} & R\arabic{reqcounter}\stepcounter{reqcounter} & Visualizar reportes                                & 84                                          & 14                                           \\\cline{2-6}
                                                                         & \arabic{numcounter}\stepcounter{numcounter} & R\arabic{reqcounter}\stepcounter{reqcounter} & Visualizar alertas de emergencia de los ciudadanos & 12                                          & 2                                            \\
\end{longtable}

% \subsection{Diseño de usuario}

% \subsubsection{Análisis del proceso propuesto}

% En la Figura \ref{fig:proceso-propuesto} se presenta el proceso propuesto para el envió de alertas de emergencia con
% ubicación en tiempo real, en donde:

% \begin{enumerate}
%     \item Si el usuario no se encuentra registrado en el sistema:
%           \begin{enumerate}
%               \item El usuario se registra en el sistema.
%               \item El usuario inicia sesión en el sistema.
%           \end{enumerate}
%     \item El usuario ingresa miembros a su grupo familiar.
%     \item El usuario selecciona un tipo de incidente.
%           % \item El usuario envía la alerta de emergencia presionando el bot��������n de pánico durante 3 segundos.
%     \item El usuario envía la alerta de emergencia presionando el botón de pánico durante 3 segundos.
%     \item Se envía una notificación a los miembros del grupo familiar del usuario y los policías dentro de la zona de emergencia junto con la ubicación en tiempo real.
%     \item El Ecu 911 recibe la alerta de emergencia y la asigna a la entidad correspondiente.
%     \item La emergencia es atendida.
% \end{enumerate}

% \begin{figure}[H]
%     \centering
%     \includegraphics[width=0.8\textwidth]{chapters/III-resultados-y-discusion/resources/images/proceso-propuesto.png}
%     \caption{Proceso propuesto de reporte de incidentes al ECU 911.}
%     \label{fig:proceso-propuesto}
% \end{figure}

% \subsubsection{Arquitectura}

% El desarrollo del sistema se estructuró en 2 partes fundamentales: la API y La interfaz de usuario, tanto web como móvil.
% La interfaz de usuario en el sistema web permite a los administradores gestionar la información de los usuarios y los incidentes
% ademas de visualizar la ubicación en tiempo real de los incidentes en un mapa. La interfaz de usuario en el sistema móvil
% permite gestionar grupos familiares, enviar alertas de emergencia y visualizar la ubicación en tiempo real de los incidentes de sus
% miembros del grupo familiar. El API se encuentra alojada en un servidor web y se encarga de gestionar la conexión entre la base de
% datos, los servicios de almacenamiento de información, el servicio de hosting de imágenes, el servicio de web socket y las interfaces
% de usuario.

% \begin{figure}[H]
%     \centering
%     \includegraphics[width=1.1\textwidth]{chapters/III-resultados-y-discusion/resources/images/arquitectura.png}
%     \caption{Arquitectura del sistema}
%     \label{fig:arquitectura}
% \end{figure}


% \subsubsection{Prototipado}

% En esta etapa de la metodología, se desarrollaron prototipos con el objetivo de identificar y resolver problemas de diseño y funcionalidad.
% Esto permite detectar posibles fallos o áreas de mejora antes de la fase de construcción de la aplicación, optimizando así el proceso de
% desarrollo y asegurando que el producto final sea más robusto y alineado con los objetivos del proyecto.

% El prototipado para el sistema se divide en dos partes: el prototipo de la interfaz de usuario web y el prototipo de la interfaz de usuario móvil.

% \subsubsection{Prototipo de la interfaz de usuario web}

% En la Figura \ref{fig:prototipo-inicio-sesion-web} se presenta el prototipo de la interfaz de usuario web, donde se muestra la pantalla de inicio de sesión,
% el cual se realiza mediante correo y contraseña.

% \begin{figure}[H]
%     \centering
%     \includegraphics[width=0.6\textwidth]{chapters/III-resultados-y-discusion/resources/images/prototipo-inicio-sesion-web.png}
%     \caption{Prototipo de la interfaz de usuario web: Inicio de sesión.}
%     \label{fig:prototipo-inicio-sesion-web}
% \end{figure}

% En la Figura \ref{fig:prototipo-layout-web} se presenta el prototipo de la interfaz de usuario web, donde se muestra el layout de la aplicación web junto
% con el menú de navegación y el menú de opciones

% \begin{figure}[H]
%     \centering
%     \includegraphics[width=0.6\textwidth]{chapters/III-resultados-y-discusion/resources/images/prototipo-layout-web.png}
%     \caption{Prototipo de la interfaz de usuario web: Layout.}
%     \label{fig:prototipo-layout-web}
% \end{figure}

% La gestión de la información en el sistema web se realiza a través de una tabla de entradas, la cual permite al usuario administrador crear, visualizar,
% editar y eliminar los registros, así como también aplicar filtros de búsqueda, como se muestra en la Figura \ref{fig:prototipo-tabla-entradas-web}.

% \begin{figure}[H]
%     \centering
%     \includegraphics[width=0.6\textwidth]{chapters/III-resultados-y-discusion/resources/images/prototipo-tabla-entradas-web.png}
%     \caption{Prototipo de la interfaz de usuario web: Tabla de entradas.}
%     \label{fig:prototipo-tabla-entradas-web}
% \end{figure}

% En la Figura \ref{fig:prototipo-menu-tabla-entradas-web} se muestra el menú de opciones de la tabla de entradas, el cual permite al usuario administrador
% realizar acciones como editar y eliminar registros. Al eliminar un registro, se muestra un mensaje de confirmación para ejecutar la acción, como se puede
% observar en la Figura \ref{fig:prototipo-mensaje-eliminar-web}

% \begin{figure}[H]
%     \centering
%     \includegraphics[width=0.6\textwidth]{chapters/III-resultados-y-discusion/resources/images/prototipo-menu-tabla-entradas-web.png}
%     \caption{Prototipo de la interfaz de usuario web: Menú de opciones de la tabla de entradas.}
%     \label{fig:prototipo-menu-tabla-entradas-web}
% \end{figure}

% \begin{figure}[H]
%     \centering
%     \includegraphics[width=0.6\textwidth]{chapters/III-resultados-y-discusion/resources/images/prototipo-mensaje-eliminar-web.png}
%     \caption{Prototipo de la interfaz de usuario web: Mensaje de confirmación para eliminar un registro.}
%     \label{fig:prototipo-mensaje-eliminar-web}
% \end{figure}

% Para la gestión de usuarios en el sistema web, la informaci��n de dichos usuarios se visualizará en una tabla, como se puede observar en la Figura
% \ref{fig:prototipo-tabla-usuarios-web}. Se propone un formulario de registro en el cual el usuario administrador puede ingresar la información
% necesaria para crear un nuevo usuario, así como actualizar la información de un usuario existente y eliminar un usuario, como se muestra en la Figura
% \ref{fig:prototipo-formulario-usuario-web}.

% \begin{figure}[H]
%     \centering
%     \includegraphics[width=0.6\textwidth]{chapters/III-resultados-y-discusion/resources/images/prototipo-tabla-usuarios-web.png}
%     \caption{Prototipo de la interfaz de usuario web: Tabla de usuarios.}
%     \label{fig:prototipo-tabla-usuarios-web}
% \end{figure}

% \begin{figure}[H]
%     \centering
%     \includegraphics[width=0.6\textwidth]{chapters/III-resultados-y-discusion/resources/images/prototipo-formulario-usuario-web.png}
%     \caption{Prototipo de la interfaz de usuario web: Formulario de usuario.}
%     \label{fig:prototipo-formulario-usuario-web}
% \end{figure}

% Para la visualización de la información de los tipos de incidentes en el sistema web, se desarrolló una tabla de entradas, como se muestra en la Figura
% \ref{fig:prototipo-tabla-tipos-incidentes-web}. Se propone un formulario de registro en el cual el usuario administrador puede ingresar la información
% necesaria para crear un nuevo tipo de incidente, así como actualizar la información de un tipo de incidente existente y eliminar un tipo de incidente,
% como se muestra en la Figura \ref{fig:prototipo-formulario-tipo-incidente-web}.

% \begin{figure}[H]
%     \centering
%     \includegraphics[width=0.6\textwidth]{chapters/III-resultados-y-discusion/resources/images/prototipo-tabla-tipos-incidentes-web.png}
%     \caption{Prototipo de la interfaz de usuario web: Tabla de tipos de incidentes.}
%     \label{fig:prototipo-tabla-tipos-incidentes-web}
% \end{figure}

% \begin{figure}[H]
%     \centering
%     \includegraphics[width=0.6\textwidth]{chapters/III-resultados-y-discusion/resources/images/prototipo-formulario-tipo-incidente-web.png}
%     \caption{Prototipo de la interfaz de usuario web: Formulario de tipo de incidente.}
%     \label{fig:prototipo-formulario-tipo-incidente-web}
% \end{figure}

% En la Figura \ref{fig:prototipo-mapa-zonas-de-vigilancia-web} se muestra la pantalla de zonas de vigilancia, la cual permite al usuario administrador
% visualizar las zonas de vigilancia mediante polígonos en un mapa interactivo.

% \begin{figure}[H]
%     \centering
%     \includegraphics[width=0.6\textwidth]{chapters/III-resultados-y-discusion/resources/images/prototipo-mapa-zonas-de-vigilancia-web.png}
%     \caption{Prototipo de la interfaz de usuario web: Mapa de zonas de vigilancia.}
%     \label{fig:prototipo-mapa-zonas-de-vigilancia-web}
% \end{figure}

% Para la creación de zonas de vigilancia, se propone un formulario integrado con el mapa interactivo, el cual permite al usuario administrador
% dibujar un polígono en el mapa para definir una zona de vigilancia, como se muestra en la Figura \ref{fig:prototipo-formulario-zona-vigilancia-web}.

% \begin{figure}[H]
%     \centering
%     \includegraphics[width=0.6\textwidth]{chapters/III-resultados-y-discusion/resources/images/prototipo-formulario-zona-vigilancia-web.png}
%     \caption{Prototipo de la interfaz de usuario web: Formulario de zona de vigilancia.}
%     \label{fig:prototipo-formulario-zona-vigilancia-web}
% \end{figure}

% La pantalla de mapa de incidentes permite al usuario administrador visualizar los incidentes reportados en un mapa interactivo junto con la
% ubicación en tiempo real de la víctima, como se muestra en la Figura \ref{fig:prototipo-mapa-incidentes-web}.

% \begin{figure}[H]
%     \centering
%     \includegraphics[width=0.6\textwidth]{chapters/III-resultados-y-discusion/resources/images/prototipo-mapa-incidentes-web.png}
%     \caption{Prototipo de la interfaz de usuario web: Mapa de incidentes.}
%     \label{fig:prototipo-mapa-incidentes-web}
% \end{figure}

% En la Figura \ref{fig:prototipo-mapa-de-calor-web} se muestra la pantalla de mapa de calor, la cual permite al usuario administrador visualizar
% la densidad de incidentes reportados en un mapa interactivo mediante un gradiente de colores, así como filtrar los incidentes por tipo y fecha.

% \begin{figure}[H]
%     \centering
%     \includegraphics[width=0.6\textwidth]{chapters/III-resultados-y-discusion/resources/images/prototipo-mapa-de-calor-web.png}
%     \caption{Prototipo de la interfaz de usuario web: Mapa de calor.}
%     \label{fig:prototipo-mapa-de-calor-web}
% \end{figure}

% En la pantalla de reportería, el usuario administrador puede visualizar gráficos estadísticos de los incidentes reportados, tal como se muestra
% en la Figura \ref{fig:prototipo-reporteria-web}.

% \begin{figure}[H]
%     \centering
%     \includegraphics[width=0.6\textwidth]{chapters/III-resultados-y-discusion/resources/images/prototipo-reporteria-web.png}
%     \caption{Prototipo de la interfaz de usuario web: Reportería.}
%     \label{fig:prototipo-reporteria-web}
% \end{figure}

% \subsubsection{Prototipo de la interfaz de usuario móvil}
% % TODO: REVISAR ORTOGRAFÍA
% En la Figura \ref{fig:prototipo-inicio-sesion-mobile} se muestra la pantalla de inicio de sesión en la aplicación móvil, la cual se realiza mediante
% el correo y contraseña del usuario.

% \begin{figure}[H]
%     \centering
%     \includegraphics[width=0.3\textwidth]{chapters/III-resultados-y-discusion/resources/images/prototipo-inicio-sesion-mobile.png}
%     \caption{Prototipo de la interfaz de usuario móvil: Inicio de sesión.}
%     \label{fig:prototipo-inicio-sesion-mobile}
% \end{figure}

% Para el registro de usuarios se propone un formulario en el cual se ingresan una fotografía, nombres, apellidos, etnia, género, estado civil,
% discapacidad en caso de presentarla y dirección, tal como se muestra en la Figura \ref{fig:prototipo-registro-mobile}.

% \begin{figure}[H]
%     \centering
%     \includegraphics[width=0.3\textwidth]{chapters/III-resultados-y-discusion/resources/images/prototipo-registro-mobile.png}
%     \caption{Prototipo de la interfaz de usuario móvil: Registro.}
%     \label{fig:prototipo-registro-mobile}
% \end{figure}

% En la Figura \ref{fig:prototipo-recuperar-contrasena-mobile} se muestra la pantalla de recuperación de contraseña en la aplicación móvil, la cual se realiza
% mediante el correo electrónico del usuario.

% \begin{figure}[H]
%     \centering
%     \includegraphics[width=0.3\textwidth]{chapters/III-resultados-y-discusion/resources/images/prototipo-recuperar-contrasena-mobile.png}
%     \caption{Prototipo de la interfaz de usuario móvil: Recuperación de contraseña.}
%     \label{fig:prototipo-recuperar-contrasena-mobile}
% \end{figure}

% Para la pantalla de inicio, se propone un botón de pánico que permite al usuario enviar una alerta de emergencia seleccionando el tipo de incidente y
% presionando el botón durante 3 segundos para enviar la alerta. En la esquina superior derecha se encuentra el menú de opciones de la aplicación y en la
% esquina superior izquierda se encuentra el botón de notificaciones de alertas de emergencia, como se muestra en la Figura \ref{fig:prototipo-inicio-mobile}.

% \begin{figure}[H]
%     \centering
%     \includegraphics[width=0.3\textwidth]{chapters/III-resultados-y-discusion/resources/images/prototipo-inicio-mobile.png}
%     \caption{Prototipo de la interfaz de usuario móvil: Inicio.}
%     \label{fig:prototipo-inicio-mobile}
% \end{figure}

% En la Figura  \ref{fig:prototipo-alertas-mobile} se muestra la pantalla de alertas de emergencia, el usuario puede visualizar las alertas de emergencia
% enviadas por los miembros del grupo familiar mediante una lista de alertas con la información de la persona y un botón para visualizar la ubicación
% en tiempo real del incidente en un mapa interactivo como se puede observar en la Figura \ref{fig:prototipo-ubicacion-alerta-mobile}.

% \begin{figure}[H]
%     \centering
%     \includegraphics[width=0.3\textwidth]{chapters/III-resultados-y-discusion/resources/images/prototipo-alertas-mobile.png}
%     \caption{Prototipo de la interfaz de usuario móvil: Alertas de emergencia.}
%     \label{fig:prototipo-alertas-mobile}
% \end{figure}

% \begin{figure}[H]
%     \centering
%     \includegraphics[width=0.3\textwidth]{chapters/III-resultados-y-discusion/resources/images/prototipo-ubicacion-alerta-mobile.png}
%     \caption{Prototipo de la interfaz de usuario móvil: Ubicación de alerta.}
%     \label{fig:prototipo-ubicacion-alerta-mobile}
% \end{figure}

% El menú de opciones de la aplicación móvil permite al usuario gestionar su grupo familiar, cambiar su contraseña y cerrar sesión, como se muestra
% en la Figura \ref{fig:prototipo-menu-mobile}.

% \begin{figure}[H]
%     \centering
%     \includegraphics[width=0.3\textwidth]{chapters/III-resultados-y-discusion/resources/images/prototipo-menu-mobile.png}
%     \caption{Prototipo de la interfaz de usuario móvil: Menú.}
%     \label{fig:prototipo-menu-mobile}
% \end{figure}

% Para la gestión de grupos familiares, se propone una pantalla en la cual el usuario puede visualizar los miembros de su grupo familiar y agregar
% nuevos miembros, como se muestra en la Figura \ref{fig:prototipo-grupo-familiar-mobile}. Al agregar un nuevo miembro, se muestra un formulario
% en el cual se puede buscar un usuario por su cédula de identidad y agregarlo al grupo familiar, como se muestra en la Figura \ref{fig:prototipo-agregar-miembro-mobile}.

% \begin{figure}[H]
%     \centering
%     \includegraphics[width=0.3\textwidth]{chapters/III-resultados-y-discusion/resources/images/prototipo-grupo-familiar-mobile.png}
%     \caption{Prototipo de la interfaz de usuario móvil: Grupo familiar.}
%     \label{fig:prototipo-grupo-familiar-mobile}
% \end{figure}

% \begin{figure}[H]
%     \centering
%     \includegraphics[width=0.3\textwidth]{chapters/III-resultados-y-discusion/resources/images/prototipo-agregar-miembro-mobile.png}
%     \caption{Prototipo de la interfaz de usuario móvil: Agregar miembro.}
%     \label{fig:prototipo-agregar-miembro-mobile}
% \end{figure}

% En la Figura \ref{fig:prototipo-cambiar-contrasena-mobile} se muestra la pantalla para cambiar la contraseña en la aplicación móvil. El usuario
% debe ingresar su contraseña actual, la nueva contraseña y confirmar la nueva contraseña para realizar el cambio.

% \begin{figure}[H]
%     \centering
%     \includegraphics[width=0.3\textwidth]{chapters/III-resultados-y-discusion/resources/images/prototipo-cambiar-contrasena-mobile.png}
%     \caption{Prototipo de la interfaz de usuario móvil: Cambiar contraseña.}
%     \label{fig:prototipo-cambiar-contrasena-mobile}
% \end{figure}

% \subsection{Construcción}

% En esta sección se describe la construcción de la propuesta del sistema, la cual se divide en dos partes fundamentales: la API y la interfaz
% de usuario.

% \subsubsection{API (Backend)}

% \paragraph{Dependecias de la API}
% Las dependencias utilizadas en la API se gestionaron mediante npm (Node Package Manager), el cual es un gestor de paquetes para el lenguaje
% de programación JavaScript, el cual permite instalar, actualizar y eliminar paquetes mediante un archivo de configuración llamado package.json.
% En en Anexo \ref{apendix:dependencias-api} muestra el archivo package.json con las dependencias utilizadas en la API.

% \paragraph{Configurar variables de entorno de la API}
% Para la configuración de las variables de entorno de la API se utilizó un archivo .env, el cual contiene las propiedades de la
% aplicación, como la URL de conexión a la base de datos, la configuración para el envío de correos electrónicos, la configuración del servicio de
% almacenamiento de imágenes, la configuración del servicio de notificaciones push y la configuración del servicio de Google Maps y el secret key
% para la generación de tokens JWT. En el Anexo \ref{apendix:configuracion-env-api} se muestra el archivo .env con las variables de entorno de la API.

% \paragraph{Configurción de la seguridad de la API}
% Para la configuración de la seguridad de la API, se utilizó una estrategia de autenticación basada en JWT (Json Web Token) y el sistema
% de Guards que proporciona NestJS. La estrategia de autenticación JWT se implementó mediante un middleware que verifica la validez del
% token JWT en cada solicitud realizada a la API. Este middleware se encarga de extraer el token del encabezado de autorización y verificar
% la firma del token con la clave secreta, además de comprobar si el usuario tiene los permisos necesarios para acceder a los recursos
% protegidos. En el Anexo \ref{apendix:configuracion-seguridad-api} se muestra la configuración de la seguridad de la API.

% \paragraph{Accesso y persistencia de datos}
% Para administrar el acceso y la persistencia de datos en la API, se utilizó un ORM (Object-Relational Mapping) llamado PrismaJS, que
% emplea un lenguaje de modelado de datos propio para definir las entidades de la aplicación mediante un archivo "schema.prisma", el cual
% contiene los modelos y relaciones de la base de datos, como se muestra en el Anexo \ref{apendix:modelo-datos-api}. PrismaJS facilita
% la interacción con la base de datos PostgreSQL mediante consultas SQL generadas automáticamente a partir de las operaciones CRUD (Create,
% Read, Update, Delete) realizadas en la API, lo que permite mantener una sincronía entre TypeScript y la base de datos. De este modo,
% cualquier cambio en el esquema de la base de datos se reflejará automáticamente en el código de la API.

% \paragraph{Esquema de la base de datos}
% Una vez definidos los modelos en el esquema de prisma se procedió a realizar la migración de la base de datos, para ello se utilizó el
% comando \mintinline{text}{npx prisma migrate dev}, el cual crea las tablas en la base de datos PostgreSQL a partir del esquema definido
% en el archivo schema.prisma, como se muestra en la Figura \ref{fig:migracion-base-datos}.

% \begin{landscape}
%     \begin{figure}[H]
%         \centering
%         \includegraphics[width=1.7\textwidth]{chapters/III-resultados-y-discusion/resources/images/migracion-base-datos.png}
%         \caption{Migración de la base de datos con PrismaJS.}
%         \label{fig:migracion-base-datos}
%     \end{figure}
% \end{landscape}

% \paragraph{Validación de datos}
% Para la validación de datos en la API, se utilizó la librería class-validator, la cual permite definir reglas de validación mediante
% decoradores en las clases de los modelos de datos. Estos decoradores se encargan de validar los datos de entrada en las solicitudes
% realizadas a la API, como se muestra en el Anexo \ref{apendix:validacion-datos-api}.

% \paragraph{Envío de correos electrónicos}
% Para el envío de correos electrónicos en la API se utilizó la librería de nest mailer, la cual trabaja sobre la librería nodemailer
% para el envío de correos electrónicos. Para la configuración del servicio de envío de correos se utilizó el SMTP (Simple Mail Transfer
% Protocol) de Gmail, este permite enviar correos electrónicos utilizando la cuenta de Gmail del usuario administrador, como se muestra
% en el Anexo \ref{apendix:envio-correos-api}.
% \bigbreak

% Para plantilla de correo electrónico se utilizó el sistema de plantillas de handlebars, el cual permite generar HTML dinámico a partir
% de un archivo de plantilla y datos de contexto en formato JSON. En el Anexo \ref{apendix:plantilla-correo-api} se muestra la plantilla
% de correo electrónico utilizada para recuperar la contraseña de un usuario.

% \paragraph{Configuración y uso del servicio de almacenamiento de imágenes}
% Para el almacenamiento de imágenes en la API se utilizó el servicio de Cloudinary, el cual permite almacenar y gestionar imágenes en la nube.
% Cloudinary provee una librería para NodeJS que facilita la subida de imágenes, para la configuración se requiere obtener una api key, api secret
% y cloud name, como se muestra en el Anexo \ref{apendix:configuracion-cloudinary-api}.
% \bigbreak

% Una vez configurado el servicio de Cloudinary, se procedió a implementar la subida de imágenes en la API. Para ello, se creó un servicio
% que abstrae la lógica de la librería de Cloudinary y permite subir imágenes a Cloudinary y obtener la URL de la imagen subida, como se
% muestra en el Anexo \ref{apendix:subida-imagenes-api}. Este servicio se inyectó en el controlador de la API y se utilizó en el endpoint
% de creación de usuarios para subir la fotografía de perfil del usuario, como se muestra en el Anexo \ref{apendix:subida-imagenes-usuario-api}.

% \paragraph{Configuración y uso del servicio de notificaciones push}
% Para el envío de notificaciones push en la API, se utilizó el servicio de OneSignal, que permite enviar notificaciones a dispositivos
% móviles y navegadores web. OneSignal provee una librería para Node.js que facilita el envío de notificaciones push. Para la configuración,
% se requiere obtener una app ID y una API key, como se muestra en el Anexo \ref{apendix:configuracion-onesignal-api}.
% \bigbreak

% Con la configuración realizada, se procedió a implementar una abstracción del API de la librería de OneSignal en un servicio que proporciona
% una instancia del cliente para el envío de notificaciones, como se puede visualizar en el Anexo \ref{apendix:configuracion-onesignal-api}. Este servicio
% se inyectó en el controlador del módulo de notificaciones push, en el cual se implementó un endpoint para enviar notificaciones a los miembros
% del grupo familiar del usuario y a los policías dentro de la zona de emergencia, como se puede observar en el Anexo \ref{apendix:envio-notificaciones-api}.

% \paragraph{Modulos}
% NestJS permite organizar la aplicación en módulos, los cuales contienen controladores, servicios y dto (Data Transfer Object). Los módulos se encargan
% de importar y exportar los componentes de la aplicación, lo que facilita la reutilización de código y la separación de responsabilidades. En el Anexo
% \ref{apendix:modulos-api} se muestra la estructura de los módulos de la API.

% \paragraph{Controladores}
% Los controladores en NestJS son clases que se encargan de gestionar las solicitudes HTTP y devolver una respuesta al cliente. Cada controlador
% contiene una serie de métodos que se corresponden con los diferentes endpoints de la API además de los decoradores que definen los permisos de
% acceso a los recursos protegidos. En el Anexo \ref{apendix:controladores-api} se muestra la estructura de los controladores de la API.

% \paragraph{Servicios}
% Los servicios en NestJS son clases que contienen la lógica de negocio de la aplicación y se encargan de interactuar con la base de datos y otros
% servicios externos. Los servicios se inyectan en los controladores y otros servicios mediante la inyección de dependencias, lo que permite reutilizar
% la lógica de negocio en diferentes partes de la aplicación. En el Anexo \ref{apendix:servicios-api} se muestra la estructura de los servicios de la API.

% \paragraph{DTO (Data Transfer Object)}
% Los DTO en NestJS son clases que se utilizan para transferir datos entre los controladores y los servicios de la aplicación. Los DTO definen la
% estructura de los datos que se envían y reciben en las solicitudes HTTP, lo que facilita la validación de datos y la prevención de errores en la
% aplicación. En el Anexo \ref{apendix:dto-api} se muestra la estructura de los DTO de la API.

% \paragraph{Documentación de la API}
% Para la documentación de la API se utilizó la librería Swagger, la cual permite generar una documentación interactiva de la API a partir de los
% decoradores y comentarios en el código fuente. La documentación de la API se puede visualizar en la ruta /api-docs, como se muestra en la Figura
% \ref{fih:documentacion-api}.

% \begin{figure}[H]
%     \centering
%     \includegraphics[width=0.8\textwidth]{chapters/III-resultados-y-discusion/resources/images/documentacion-api.png}
%     \caption{Documentación de la API con Swagger.}
%     \label{fih:documentacion-api}
% \end{figure}

% \subsubsection{Interfaz de usuario (Frontend)}

% \textbf{Sistema web}
% \bigbreak

% \paragraph{Dependencias del sistema web}
% Para crear el proyecto de NextJS se utilizó el comando \mintinline{text}{npx create-next-app@latest}, el cual crea una aplicación de
% NextJS con una estructura de carpetas y archivos predefinida. Las dependencias utilizadas en el sistema web se gestionaron mediante
% npm y el archivo de configuración package.json. En el Anexo \ref{apendix:dependencias-web} se muestra dependencias utilizadas en el
% sistema web.

% \paragraph{Configuración de variables de entorno del sistema web}
% Para la configuración de las variables de entorno del sistema web se utilizó un archivo .env.local, el cual contiene las propiedades
% de la aplicación, como la URL de la API, el api key de Google Maps, la url y secret para la autenticación y la url
% para obtener las imágenes de Cloudinary. En el Anexo \ref{apendix:configuracion-env-web} se muestra el archivo .env.local con las
% variables de entorno del sistema web.

% \paragraph{Configuración de la aplicación}
% En Next.js, la configuración global de la aplicación se realiza mediante Providers y Contexts, los cuales permiten compartir datos y
% funcionalidades entre los componentes. En el Anexo \ref{apendix:configuracion-aplicacion-web} se muestra la configuración para los
% proveedores de sesión, componentes de UI, notificaciones y gestión de datos en el sistema web.

% Las rutas de la aplicación en Next.js se crean mediante el gestor de archivos, el cual permite crear rutas dinámicas y estáticas
% siguiendo una estructura "carpeta/page.ts", donde "carpeta" es el nombre de la ruta y "page.ts" es el archivo de la página. En el
% Anexo \ref{apendix:rutas-aplicacion-web} se muestra la estructura de las rutas de la aplicación web.

% \paragraph{Inicio de sesión}
% El inicio de sesión en el sistema web se realiza mediante un formulario en el cual el usuario ingresa su correo electrónico y
% contraseña, como se muestra en la Figura \ref{fig:inicio-sesion-web}. Estos campos son validados mediante la librería Zod, la
% cual permite definir esquemas de validación para los datos de entrada en los formularios, como se puede observar en el Anexo
% \ref{apendix:validacion-datos-web}. Una vez validados los datos, se envía una solicitud POST a la API para autenticar al usuario
% y obtener un token JWT, el cual se almacena en la sesión del navegador para mantener la sesión activa, en el Anexo
% \ref{apendix:guardar-token-web} se muestra el código para guardar el token en la sesión del navegador.

% \begin{figure}[H]
%     \centering
%     \includegraphics[width=0.8\textwidth]{chapters/III-resultados-y-discusion/resources/images/inicio-sesion-web.png}
%     \caption{Inicio de sesión en el sistema web.}
%     \label{fig:inicio-sesion-web}
% \end{figure}

% \paragraph{Menú de usuario}
% El menú de usuario en el sistema web permite al usuario administrador acceder a las opciones de la aplicación al hacer clic en
% el botón de configuración en la esquina superior derecha de la pantalla. Al hacer esto, se muestra un menú desplegable en el
% cual se encuentran las opciones de cambiar contraseña y cerrar sesión, como se muestra en la Figura \ref{fig:menu-usuario-web}.

% \begin{figure}[H]
%     \centering
%     \includegraphics[width=0.8\textwidth]{chapters/III-resultados-y-discusion/resources/images/menu-usuario-web.png}
%     \caption{Menú de usuario en el sistema web.}
%     \label{fig:menu-usuario-web}
% \end{figure}

% \paragraph{Gestión de usuarios}
% La gestión de usuarios en el sistema web se realiza mediante una tabla de entradas, la cual permite al usuario administrador
% visualizar los usuarios registrados en el sistema, como se muestra en la Figura \ref{fig:tabla-usuarios-web}. En esta tabla se
% muestran los campos de la información de los usuarios, como la fotografía, nombres, apellidos, correo electrónico, etnia, género,
% entre otros. El usuario administrador puede realizar acciones como crear, editar y deshabilitar usuarios, como se puede observar en la Figura
% \ref{fig:menu-tabla-usuarios-web}. El formulario de registro de usuarios permite al usuario administrador ingresar la información
% necesaria para crear un nuevo usuario, como se puede visualizar en las Figuras \ref{fig:formulario-usuario-web-1} y \ref{fig:formulario-usuario-web-2}.

% \begin{figure}[H]
%     \centering
%     \includegraphics[width=0.8\textwidth]{chapters/III-resultados-y-discusion/resources/images/tabla-usuarios-web.png}
%     \caption{Tabla de usuarios en el sistema web.}
%     \label{fig:tabla-usuarios-web}
% \end{figure}

% \begin{figure}[H]
%     \centering
%     \includegraphics[width=0.8\textwidth]{chapters/III-resultados-y-discusion/resources/images/menu-tabla-usuarios-web.png}
%     \caption{Menú de opciones de la tabla de usuarios en el sistema web.}
%     \label{fig:menu-tabla-usuarios-web}
% \end{figure}

% \begin{figure}[H]
%     \centering
%     \includegraphics[width=0.8\textwidth]{chapters/III-resultados-y-discusion/resources/images/formulario-usuario-web-1.png}
%     \caption{Formulario para crear/editar usuarios en el sistema web (Parte 1).}
%     \label{fig:formulario-usuario-web-1}
% \end{figure}

% \begin{figure}[H]
%     \centering
%     \includegraphics[width=0.8\textwidth]{chapters/III-resultados-y-discusion/resources/images/formulario-usuario-web-2.png}
%     \caption{Formulario para crear/editar usuarios en el sistema web (Parte 2).}
%     \label{fig:formulario-usuario-web-2}
% \end{figure}

% La tabla de gestión de usuarios cuenta con varios filtros de búsqueda, los cuales permiten al usuario administrador buscar usuarios por
% columnas especificas o por un filtro general, como se muestra en la Figura \ref{fig:filtros-tabla-usuarios-web}.

% \begin{figure}[H]
%     \centering
%     \includegraphics[width=0.8\textwidth]{chapters/III-resultados-y-discusion/resources/images/filtros-tabla-usuarios-web.png}
%     \caption{Filtros de búsqueda en la tabla de usuarios en el sistema web.}
%     \label{fig:filtros-tabla-usuarios-web}
% \end{figure}

% \paragraph{Gestión de tipos de incidentes}
% La gestión de tipos de incidentes en el sistema web se realiza mediante una tabla de entradas, en la cual el usuario administrador
% puede visualizar los tipos de incidentes registrados en el sistema, como se muestra en la Figura \ref{fig:tabla-tipos-incidentes-web}.
% En esta tabla se muestran los campos de la información de los tipos de incidentes, como el nombre, descripción y estado. El usuario
% administrador puede realizar acciones como crear, editar y deshabilitar tipos de incidentes, como se puede observar en la Figura
% \ref{fig:menu-tabla-tipos-incidentes-web}. El formulario de registro de tipos de incidentes permite al usuario administrador ingresar
% la información necesaria para crear un nuevo tipo de incidente, como se puede visualizar en la Figura \ref{fig:formulario-tipo-incidente-web}

% \begin{figure}[H]
%     \centering
%     \includegraphics[width=0.8\textwidth]{chapters/III-resultados-y-discusion/resources/images/tabla-tipos-incidentes-web.png}
%     \caption{Tabla de tipos de incidentes en el sistema web.}
%     \label{fig:tabla-tipos-incidentes-web}
% \end{figure}

% \begin{figure}[H]
%     \centering
%     \includegraphics[width=0.8\textwidth]{chapters/III-resultados-y-discusion/resources/images/menu-tabla-tipos-incidentes-web.png}
%     \caption{Menú de opciones de la tabla de tipos de incidentes en el sistema web.}
%     \label{fig:menu-tabla-tipos-incidentes-web}
% \end{figure}

% \begin{figure}[H]
%     \centering
%     \includegraphics[width=0.8\textwidth]{chapters/III-resultados-y-discusion/resources/images/formulario-tipo-incidente-web.png}
%     \caption{Formulario para crear/editar tipos de incidentes en el sistema web.}
%     \label{fig:formulario-tipo-incidente-web}
% \end{figure}

% La tabla de gestión de tipos de incidentes cuenta con varios filtros de búsqueda, los cuales permiten al usuario administrador buscar
% tipos de incidentes por columnas especificas o por un filtro general, como se muestra en la Figura \ref{fig:filtros-tabla-tipos-incidentes-web}.

% \begin{figure}[H]
%     \centering
%     \includegraphics[width=0.8\textwidth]{chapters/III-resultados-y-discusion/resources/images/filtros-tabla-tipos-incidentes-web.png}
%     \caption{Filtros de búsqueda en la tabla de tipos de incidentes en el sistema web.}
%     \label{fig:filtros-tabla-tipos-incidentes-web}
% \end{figure}

% \paragraph{Gestión de zonas de vigilancia}

% La gestión de zonas de vigilancia en el sistema web se realiza por medio de un mapa interactivo, en el cual el usuario administrador
% puede visualizar las zonas de vigilancia mediante polígonos, como se muestra en la Figura \ref{fig:mapa-zonas-vigilancia-web}.
% El usuario administrador puede realizar acciones como crear, editar y deshabilitar zonas de vigilancia, como se puede observar en la Figura
% \ref{fig:menu-mapa-zonas-vigilancia-web}. El formulario de registro de zonas de vigilancia permite al usuario administrador dibujar
% un polígono en el mapa para definir una zona de vigilancia, como se puede visualizar en la Figura \ref{fig:formulario-zona-vigilancia-web}.
% Además, el usuario administrador podrá asignar policías a las zonas de vigilancia, como se muestra en la Figura
% \ref{fig:asignar-policias-zona-vigilancia-web}.

% \begin{figure}[H]
%     \centering
%     \includegraphics[width=0.8\textwidth]{chapters/III-resultados-y-discusion/resources/images/mapa-zonas-vigilancia-web.png}
%     \caption{Mapa de zonas de vigilancia en el sistema web.}
%     \label{fig:mapa-zonas-vigilancia-web}
% \end{figure}

% \begin{figure}[H]
%     \centering
%     \includegraphics[width=0.8\textwidth]{chapters/III-resultados-y-discusion/resources/images/menu-mapa-zonas-vigilancia-web.png}
%     \caption{Menú de opciones del mapa de zonas de vigilancia en el sistema web.}
%     \label{fig:menu-mapa-zonas-vigilancia-web}
% \end{figure}

% \begin{figure}[H]
%     \centering
%     \includegraphics[width=0.8\textwidth]{chapters/III-resultados-y-discusion/resources/images/formulario-zona-vigilancia-web.png}
%     \caption{Formulario para crear/editar zonas de vigilancia en el sistema web.}
%     \label{fig:formulario-zona-vigilancia-web}
% \end{figure}

% \begin{figure}[H]
%     \centering
%     \includegraphics[width=0.8\textwidth]{chapters/III-resultados-y-discusion/resources/images/asignar-policias-zona-vigilancia-web.png}
%     \caption{Asignar policías a zonas de vigilancia en el sistema web.}
%     \label{fig:asignar-policias-zona-vigilancia-web}
% \end{figure}

% \paragraph{Gestión de incidentes}
% La gestión de incidentes en el sistema web se realiza empleando un mapa interactivo, en el cual el usuario administrador puede visualizar
% los incidentes reportados junto con la ubicación en tiempo real de la víctima, así como la ubicación de los policías en la zona de emergencia,
% como se muestra en la Figura \ref{fig:mapa-incidentes-web}. El usuario administrador podrá visualizar el tipo de incidente y modificarlo en
% caso de ser necesario. Además, podrá visualizar información de la víctima y de los policías asignados, como se puede observar en la Figura
% \ref{fig:detalles-incidente-web}.

% \begin{figure}[H]
%     \centering
%     \includegraphics[width=0.8\textwidth]{chapters/III-resultados-y-discusion/resources/images/mapa-incidentes-web.png}
%     \caption{Mapa de incidentes en el sistema web.}
%     \label{fig:mapa-incidentes-web}
% \end{figure}

% \begin{figure}[H]
%     \centering
%     \includegraphics[width=0.8\textwidth]{chapters/III-resultados-y-discusion/resources/images/detalles-incidente-web.png}
%     \caption{Detalles de incidente en el sistema web.}
%     \label{fig:detalles-incidente-web}
% \end{figure}

% El usuario administrador podrá revisar las alertas de incidentes pasados mediante una tabla de entradas, en la cual se muestran los
% campos de la información de los incidentes, como el tipo de incidente, la fecha y la ubicación, como se muestra en la Figura
% \ref{fig:tabla-incidentes-web}.

% \begin{figure}[H]
%     \centering
%     \includegraphics[width=0.8\textwidth]{chapters/III-resultados-y-discusion/resources/images/tabla-incidentes-web.png}
%     \caption{Tabla de incidentes en el sistema web.}
%     \label{fig:tabla-incidentes-web}
% \end{figure}

% \paragraph{Mapa de calor}
% El mapa de calor en el sistema web permite al usuario administrador visualizar la densidad de incidentes reportados en un mapa
% mediante un gradiente de colores, así como filtrar los incidentes por tipo y fecha, como se muestra en la Figura \ref{fig:mapa-de-calor-web}.

% \begin{figure}[H]
%     \centering
%     \includegraphics[width=0.8\textwidth]{chapters/III-resultados-y-discusion/resources/images/mapa-de-calor-web.png}
%     \caption{Mapa de calor en el sistema web.}
%     \label{fig:mapa-de-calor-web}
% \end{figure}

% \textbf{Aplicación móvil}
% \bigbreak

% \paragraph{Dependencias de la aplicación móvil}
% Para crear el proyecto de Flutter se utilizó el comando \mintinline{text}{flutter create}, el cual crea una aplicación de Flutter con
% una estructura de carpetas y archivos predefinida. Las dependencias utilizadas en la aplicación móvil se gestionaron mediante flutter pub
% y el archivo de configuración pubspec.yaml. En el Anexo \ref{apendix:dependencias-movil} se muestra las dependencias utilizadas en la
% aplicación móvil.

% \paragraph{Configuración de variables de entorno de la aplicación móvil}
% Para la configuración de las variables de entorno de la aplicación móvil se utilizó un archivo .env, el cual contiene las propiedades de
% de la aplicación, como la URL de la API, el api key de Google Maps, el api key de OneSignal y la URL para obtener las imágenes
% de Cloudinary. En el Anexo \ref{apendix:configuracion-env-movil} se muestra el archivo .env con las variables de entorno de la aplicación móvil.

% \paragraph{Configuración de la aplicación}
% En Flutter, la configuración global de la aplicación se realiza mediante Providers y ChangeNotifier, los cuales permiten compartir datos y
% funcionalidades entre los widgets. En el Anexo \ref{apendix:configuracion-aplicacion-movil} se muestra la configuración para los
% proveedores de sesión, sockets, localización y gestión de datos en la aplicación móvil.

% \paragraph{Inicio de sesión}
% El inicio de sesión en la aplicación móvil se realiza mediante un formulario en el cual el usuario ingresa su correo electrónico y
% contraseña, como se muestra en la Figura \ref{fig:inicio-sesion-movil}. Estas credenciales son enviadas a la API mediante una solicitud
% POST para autenticar al usuario y obtener un token JWT, el cual se almacena en el almacenamiento local del dispositivo para mantener
% la sesión activa, en el Anexo \ref{apendix:guardar-token-movil} se muestra el código para guardar el token en el almacenamiento local del
% dispositivo.

% \begin{figure}[H]
%     \centering
%     \includegraphics[width=0.3\textwidth]{chapters/III-resultados-y-discusion/resources/images/inicio-sesion-movil.png}
%     \caption{Inicio de sesión en la aplicación móvil.}
%     \label{fig:inicio-sesion-movil}
% \end{figure}

% \paragraph{Recuperar contraseña}
% La recuperación de contraseña en la aplicación móvil se realiza mediante un formulario en el cual el usuario ingresa su correo electrónico,
% como se muestra en la Figura \ref{fig:recuperar-contrasena-movil}. Una vez ingresado el correo electrónico, se envía una contraseña provisional
% al correo electrónico del usuario con la cual podrá iniciar sesión y cambiar su contraseña, en la Figura \ref{fig:recuperar-contrasena-email}
% se muestra el correo electrónico de recuperación de contraseña.

% \begin{figure}[H]
%     \centering
%     \includegraphics[width=0.3\textwidth]{chapters/III-resultados-y-discusion/resources/images/recuperar-contrasena-movil.png}
%     \caption{Recuperar contraseña en la aplicación móvil.}
%     \label{fig:recuperar-contrasena-movil}
% \end{figure}

% \begin{figure}[H]
%     \centering
%     \includegraphics[width=1\textwidth]{chapters/III-resultados-y-discusion/resources/images/recuperar-contrasena-email.png}
%     \caption{Correo electrónico de recuperación de contraseña.}
%     \label{fig:recuperar-contrasena-email}
% \end{figure}

% \paragraph{Registro de usuario}
% El registro de usuario en la aplicación móvil se realiza mediante un formulario en el cual el usuario ingresa su información personal,
% como nombres, apellidos, correo electrónico, contraseña, género, etnia, entre otros, como se muestra en las Figuras \ref{fig:registro-usuario-movil-1}
% y \ref{fig:registro-usuario-movil-2}. Para ingresar la dirección del usuario, se utilizó un campo de búsqueda de direcciones que permite
% al usuario buscar su dirección en un mapa interactivo y seleccionarla, como se muestra en la Figura \ref{fig:registro-usuario-movil-3}.

% \begin{figure}[H]
%     \centering
%     \includegraphics[width=0.3\textwidth]{chapters/III-resultados-y-discusion/resources/images/registro-usuario-movil-1.png}
%     \caption{Registro de usuario en la aplicación móvil (Parte 1).}
%     \label{fig:registro-usuario-movil-1}
% \end{figure}

% \begin{figure}[H]
%     \centering
%     \includegraphics[width=0.3\textwidth]{chapters/III-resultados-y-discusion/resources/images/registro-usuario-movil-2.png}
%     \caption{Registro de usuario en la aplicación móvil (Parte 2).}
%     \label{fig:registro-usuario-movil-2}
% \end{figure}

% \begin{figure}[H]
%     \centering
%     \includegraphics[width=0.3\textwidth]{chapters/III-resultados-y-discusion/resources/images/registro-usuario-movil-3.png}
%     \caption{Registro de usuario en la aplicación móvil (Parte 3).}
%     \label{fig:registro-usuario-movil-3}
% \end{figure}

% \paragraph{Pantalla principal}
% La pantalla principal de la aplicación móvil muestra el botón de pánico en la parte central de la pantalla, el cual permite al usuario
% enviar una alerta de emergencia a los miembros de su grupo familiar y a los policías en la zona de emergencia presionando el botón
% durante 3 segundos, también el usuario puede seleccionar el tipo de incidente mediante un check, como se muestra en la Figura
% \ref{fig:pantalla-principal-movil}.

% \begin{figure}[H]
%     \centering
%     \includegraphics[width=0.3\textwidth]{chapters/III-resultados-y-discusion/resources/images/pantalla-principal-movil.png}
%     \caption{Pantalla principal de la aplicación móvil.}
%     \label{fig:pantalla-principal-movil}
% \end{figure}

% \paragraph{Menú de usuario}
% El menú de usuario en la aplicación móvil permite al usuario acceder a las opciones tales como cambiar contraseña, cerrar sesión y
% gestionar el grupo familiar, como se muestra en la Figura \ref{fig:menu-usuario-movil}.

% \begin{figure}[H]
%     \centering
%     \includegraphics[width=0.3\textwidth]{chapters/III-resultados-y-discusion/resources/images/menu-usuario-movil.png}
%     \caption{Menú de usuario en la aplicación móvil.}
%     \label{fig:menu-usuario-movil}
% \end{figure}

% \paragraph{Gestión de grupo familiar}
% La gestión del grupo familiar en la aplicación móvil se realiza mediante una lista en la cual el usuario podrá visualizar a
% los miembros de su grupo familiar, como se muestra en la Figura \ref{fig:grupo-familiar-movil}. El usuario podrá agregar
% miembros a su grupo familiar mediante un formulario en el cual ingresará la cédula de identidad del usuario que desee
% agregar, como se muestra en la Figura \ref{fig:agregar-miembro-movil}.

% \begin{figure}[H]
%     \centering
%     \includegraphics[width=0.3\textwidth]{chapters/III-resultados-y-discusion/resources/images/grupo-familiar-movil.png}
%     \caption{Gestión de grupo familiar en la aplicación móvil.}
%     \label{fig:grupo-familiar-movil}
% \end{figure}

% \begin{figure}[H]
%     \centering
%     \includegraphics[width=0.3\textwidth]{chapters/III-resultados-y-discusion/resources/images/agregar-miembro-movil.png}
%     \caption{Agregar miembro al grupo familiar en la aplicación móvil.}
%     \label{fig:agregar-miembro-movil}
% \end{figure}

% \paragraph{Cambiar contraseña}
% La opción de cambiar contraseña en la aplicación móvil permite al usuario modificar su contraseña ingresando la contraseña actual,
% la nueva contraseña y la confirmación de la nueva contraseña, como se muestra en la Figura \ref{fig:cambiar-contrasena-movil}.

% \begin{figure}[H]
%     \centering
%     \includegraphics[width=0.3\textwidth]{chapters/III-resultados-y-discusion/resources/images/cambiar-contrasena-movil.png}
%     \caption{Cambiar contraseña en la aplicación móvil.}
%     \label{fig:cambiar-contrasena-movil}
% \end{figure}

% \paragraph{Alertas recibidas}
% Las alertas recibidas en la aplicación móvil se visualizan mediante una lista en la cual se muestra el estado de cada miembro del
% grupo familiar. Cuando un miembro del grupo familiar envía una alerta de emergencia, se muestra una notificación en el dispositivo
% del usuario, como se puede observar en la Figura \ref{fig:notificacion-recibida-movil} y el estado del miembro cambia a "En peligro", como se
% puede observar en la Figura \ref{fig:alerta-recibida-movil}. El usuario podrá visualizar la ubicación en tiempo real del miembro en
% peligro seleccionando el botón "Seguir ubicación", lo cual mostrará la en un mapa la ubicación del miembro en peligro y la ubicación
% del usuario, como se muestra en la Figura \ref{fig:seguir-ubicacion-movil}.

% \begin{figure}[H]
%     \centering
%     \includegraphics[width=0.3\textwidth]{chapters/III-resultados-y-discusion/resources/images/notificacion-recibida-movil.png}
%     \caption{Notificación de alerta recibida en la aplicación móvil.}
%     \label{fig:notificacion-recibida-movil}
% \end{figure}

% \begin{figure}[H]
%     \centering
%     \includegraphics[width=0.3\textwidth]{chapters/III-resultados-y-discusion/resources/images/alerta-recibida-movil.png}
%     \caption{Alerta recibida en la aplicación móvil.}
%     \label{fig:alerta-recibida-movil}
% \end{figure}

% \begin{figure}[H]
%     \centering
%     \includegraphics[width=0.3\textwidth]{chapters/III-resultados-y-discusion/resources/images/seguir-ubicacion-movil.png}
%     \caption{Seguir ubicación en la aplicación móvil.}
%     \label{fig:seguir-ubicacion-movil}
% \end{figure}

% \textbf{Modelo analítico de BI}
% \bigbreak

% Para el diseño del modelo analítico de BI se utilizó la metodología Hefesto, ya que esta facilita la construcción de un
% Data Warehouse y aporta información útil para mejorar el rendimiento \cite{darioDATAWAREHOUSINGMarco}.

% \paragraph{Carta de diseño para el modelo dimensional}

La carta de diseño para el modelo dimensional se utiliza para describir el proceso de diseño de un modelo dimensional
siguiendo los pasos de la metodología Hefesto. Estos pasos incluyen la selección del proceso de negocio, la declaración
del grano, la identificación de las dimensiones y sus atributos, la identificación de las tablas de hechos y sus métricas,
y el diseño del esquema dimensional. A continuación, se presenta cada uno de estos pasos.

\begin{itemize}
    \item Seleccionar el proceso de negocio:
          \begin{itemize}
              \item Identificar las pregustas del negocio.
              \item Identificar indicadores y perspectivas.
              \item Diseñar el modelo conceptual.
          \end{itemize}
    \item Declarar el grano.
    \item Identificar las dimensiones y sus atributos.
    \item Identificar las tablas de hechos y sus métricas.
    \item Diseñar el esquema dimensional
\end{itemize}

\paragraph{Seleccionar el proceso de negocio}

\paragraph{Preguntas del negocio}

El objetivo principal del sistema de reportería de incidentes delictivos es recolectar y analizar información
sobre los incidentes delictivos reportados a través de un sistema de alarma, así como detalles de los usuarios
y datos geográficos relevantes para apoyar la toma de decisiones de las autoridades policiales.

\paragraph{Identificar indicadores y perspectivas}

\begin{itemize}
    \item Determinar el número de incidentes por tipo de incidente y zona de vigilancia.
    \item Analizar la distribución de incidentes por género, discapacidad, etnia y edad de los usuarios.
    \item Identificar las zonas con mayor número de incidentes para optimizar la vigilancia policial.
\end{itemize}

\begin{longtable}{|p{5cm}|p{5cm}|}
    \caption{Indicadores y perspectivas en base las pregustas de negocio} \label{tab:indicadores-perspectivas} \\

    \hline \multicolumn{1}{|c|}{\textbf{Indicadores}} & \multicolumn{1}{|c|}{\textbf{Perspectivas}}            \\ \hline
    \endfirsthead

    \multicolumn{2}{c}%
    {{\normalfont \tablename\ \thetable{} -- continuación de la página anterior}}                              \\
    \hline \multicolumn{1}{|c|}{\textbf{Indicadores}} & \multicolumn{1}{|c|}{\textbf{Perspectivas}}            \\ \hline
    \endhead

    \hline \multicolumn{2}{|r|}{{Continua en la siguiente página}}                                             \\ \hline
    \endfoot

    \hline \hline
    \endlastfoot
    Número de incidentes                              & Tipo de Alarma                                         \\\hline
    Número de incidentes                              & Tipo de Incidente                                      \\\hline
    Número de incidentes                              & Zona de Vigilancia                                     \\\hline
    Número de incidentes                              & Género, Discapacidad, Etnia, Estado Civil              \\
\end{longtable}

\paragraph{Declarar el grano}

El grano del modelo dimensional es un único incidente delictivo reportado. Esto implica que cada registro en la
tabla de hechos representa un incidente específico.

\paragraph{Identificar las dimensiones y sus atributos}

En las Tablas \ref{tab:dimension-tiempo}, \ref{tab:dimension-usuarios}, \ref{tab:dimension-tipo-de-alarma},
\ref{tab:dimension-tipo-de-incidente} y \ref{tab:dimension-zona-de-vigilancia} se presentan las dimensiones y
sus atributos identificados para el modelo dimensional.

En la Tabla \ref{tab:dimension-tiempo} se presenta los atributos seleccionados para la dimensión de tiempo.

\begin{longtable}{|p{6cm}|p{6cm}|}
    \caption{Dimensión de tiempo con sus atributos} \label{tab:dimension-tiempo}             \\

    \hline \multicolumn{1}{|c|}{\textbf{Campo}} & \multicolumn{1}{|c|}{\textbf{Descripción}} \\ \hline
    \endfirsthead

    \multicolumn{2}{c}%
    {{\normalfont \tablename\ \thetable{} -- continuación de la página anterior}}            \\
    \hline \multicolumn{1}{|c|}{\textbf{Campo}} & \multicolumn{1}{|c|}{\textbf{Descripción}} \\ \hline
    \endhead

    \hline \multicolumn{2}{|r|}{{Continua en la siguiente página}}                           \\ \hline
    \endfoot

    \hline \hline
    \endlastfoot
    fechaID                                     & Identificador único para cada fecha        \\\hline
    fecha                                       & Fecha completa del incidente (YYYY-MM-DD)  \\\hline
    anio                                        & Año en que ocurrió el incidente            \\\hline
    mes                                         & Mes en que ocurrió el incidente            \\\hline
    día                                         & Día en que ocurrió el incidente            \\\hline
    trimestre                                   & Trimestre en que ocurrió el incidente      \\\hline
    semestre                                    & Semestre en que ocurrió el incidente       \\\hline
    hora                                        & Hora en que ocurrió el incidente           \\
\end{longtable}

En la Tabla \ref{tab:dimension-usuarios} se presenta los atributos seleccionados para la dimensión de usuarios.

\begin{longtable}{|p{6cm}|p{6cm}|}
    \caption{Dimensión de usuarios con sus atributos} \label{tab:dimension-usuarios}         \\

    \hline \multicolumn{1}{|c|}{\textbf{Campo}} & \multicolumn{1}{|c|}{\textbf{Descripción}} \\ \hline
    \endfirsthead

    \multicolumn{2}{c}%
    {{\normalfont \tablename\ \thetable{} -- continuación de la página anterior}}            \\
    \hline \multicolumn{1}{|c|}{\textbf{Campo}} & \multicolumn{1}{|c|}{\textbf{Descripción}} \\ \hline
    \endhead

    \hline \multicolumn{2}{|r|}{{Continua en la siguiente página}}                           \\ \hline
    \endfoot

    \hline \hline
    \endlastfoot
    usuarioID                                   & Identificador único del usuario            \\\hline
    género                                      & Género del usuario                         \\\hline
    discapacidad                                & Estado de discapacidad del usuario         \\\hline
    etnia                                       & Etnia del usuario                          \\\hline
    estadoCivil                                 & Estado civil del usuario                   \\\hline
    edad                                        & Edad del usuario                           \\
\end{longtable}

En la Tabla \ref{tab:dimension-tipo-de-alarma} se presenta los atributos seleccionados para la dimensión de tipo de alarma.

\begin{longtable}{|p{6cm}|p{6cm}|}
    \caption{Dimensión de tipo de alarma con sus atributos} \label{tab:dimension-tipo-de-alarma} \\

    \hline \multicolumn{1}{|c|}{\textbf{Campo}} & \multicolumn{1}{|c|}{\textbf{Descripción}}     \\ \hline
    \endfirsthead

    \multicolumn{2}{c}%
    {{\normalfont \tablename\ \thetable{} -- continuación de la página anterior}}                \\
    \hline \multicolumn{1}{|c|}{\textbf{Campo}} & \multicolumn{1}{|c|}{\textbf{Descripción}}     \\ \hline
    \endhead

    \hline \multicolumn{2}{|r|}{{Continua en la siguiente página}}                               \\ \hline
    \endfoot

    \hline \hline
    \endlastfoot
    tipoAlarmaID                                & Identificador único del tipo de alarma         \\\hline
    nombreTipoAlarma                            & Nombre del tipo de alarma                      \\
\end{longtable}

En la Tabla \ref{tab:dimension-tipo-de-incidente} se presenta los atributos seleccionados para la dimensión de tipo de incidente.

\begin{longtable}{|p{6cm}|p{6cm}|}
    \caption{Dimensión de tipo de incidente con sus atributos} \label{tab:dimension-tipo-de-incidente} \\

    \hline \multicolumn{1}{|c|}{\textbf{Campo}} & \multicolumn{1}{|c|}{\textbf{Descripción}}           \\ \hline
    \endfirsthead

    \multicolumn{2}{c}%
    {{\normalfont \tablename\ \thetable{} -- continuación de la página anterior}}                      \\
    \hline \multicolumn{1}{|c|}{\textbf{Campo}} & \multicolumn{1}{|c|}{\textbf{Descripción}}           \\ \hline
    \endhead

    \hline \multicolumn{2}{|r|}{{Continua en la siguiente página}}                                     \\ \hline
    \endfoot

    \hline \hline
    \endlastfoot
    tipoIncidenteID                             & Identificador único del tipo de incidente            \\\hline
    nombreTipoIncidente                         & Nombre del tipo de incidente                         \\
\end{longtable}

En la Tabla \ref{tab:dimension-zonas-vigilancia} se presenta los atributos seleccionados para la dimensión de zonas de vigilancia.

\begin{longtable}{|p{6cm}|p{6cm}|}
    \caption{Dimensión de zonas de vigilancia con sus atributos} \label{tab:dimension-zonas-vigilancia} \\

    \hline \multicolumn{1}{|c|}{\textbf{Campo}} & \multicolumn{1}{|c|}{\textbf{Descripción}}            \\ \hline
    \endfirsthead

    \multicolumn{2}{c}%
    {{\normalfont \tablename\ \thetable{} -- continuación de la página anterior}}                       \\
    \hline \multicolumn{1}{|c|}{\textbf{Campo}} & \multicolumn{1}{|c|}{\textbf{Descripción}}            \\ \hline
    \endhead

    \hline \multicolumn{2}{|r|}{{Continua en la siguiente página}}                                      \\ \hline
    \endfoot

    \hline \hline
    \endlastfoot
    zonaVigilanciaID                            & Identificador único de la zona de vigilancia          \\\hline
    polígono                                    & Representación geográfica de la zona de vigilancia    \\\hline
    nombreZonaVigilancia                        & Nombre de la zona de vigilancia                       \\
\end{longtable}

En la Tabla \ref{tab:dimension-ubicacion} se presenta los atributos seleccionados para la dimensión de ubicación.

\begin{longtable}{|p{6cm}|p{6cm}|}
    \caption{Dimensión de ubicación con sus atributos} \label{tab:dimension-ubicacion}       \\

    \hline \multicolumn{1}{|c|}{\textbf{Campo}} & \multicolumn{1}{|c|}{\textbf{Descripción}} \\ \hline
    \endfirsthead

    \multicolumn{2}{c}%
    {{\normalfont \tablename\ \thetable{} -- continuación de la página anterior}}            \\
    \hline \multicolumn{1}{|c|}{\textbf{Campo}} & \multicolumn{1}{|c|}{\textbf{Descripción}} \\ \hline
    \endhead

    \hline \multicolumn{2}{|r|}{{Continua en la siguiente página}}                           \\ \hline
    \endfoot

    \hline \hline
    \endlastfoot
    ubicacionID                                 & Identificador único de la ubicación        \\\hline
    canton                                      & Cantón del lugar del incidente             \\\hline
    ciudad                                      & Ciudad del lugar del incidente             \\
\end{longtable}

\paragraph{Identificar las tablas de hechos y sus métricas}

En la Tabla \ref{tab:hechos-incidentes-delictivos} se presentan los hechos de incidentes delictivos y sus atributos.

\begin{longtable}{|p{6cm}|p{6cm}|}
    \caption{Hechos de incidentes delictivos con sus atributos} \label{tab:hechos-incidentes-delictivos}     \\

    \hline \multicolumn{1}{|c|}{\textbf{Campo}} & \multicolumn{1}{|c|}{\textbf{Descripción}}                 \\ \hline
    \endfirsthead

    \multicolumn{2}{c}%
    {{\normalfont \tablename\ \thetable{} -- continuación de la página anterior}}                            \\
    \hline \multicolumn{1}{|c|}{\textbf{Campo}} & \multicolumn{1}{|c|}{\textbf{Descripción}}                 \\ \hline
    \endhead

    \hline \multicolumn{2}{|r|}{{Continua en la siguiente página}}                                           \\ \hline
    \endfoot

    \hline \hline
    \endlastfoot
    incidenteID                                 & Clave primaria de la tabla de hechos incidentes delictivos \\\hline
    fechaID                                     & Clave foránea a la dimensión de tiempo                     \\\hline
    usuarioID                                   & Clave foránea a la dimensión de usuarios                   \\\hline
    tipoAlarmaID                                & Clave foránea a la dimensión de tipo de alarma             \\\hline
    tipoIncidenteID                             & Clave foránea a la dimensión de tipo de incidente          \\\hline
    zonaVigilanciaID                            & Clave foránea a la dimensión de zonas de vigilancia        \\\hline
    ubicacionID                                 & Clave foránea a la dimensión de ubicación geográfica       \\\hline
    numeroIncidentes                            & Número de incidentes reportados                            \\\hline
\end{longtable}

\paragraph{Esquema dimensional}

En la Figura \ref{fig:esquema-modelo-dimensional} se muestra el esquema del modelo dimensional propuesto para el sistema de reportería de incidentes delictivos.

\begin{figure}[H]
    \centering
    \includegraphics[width=1\textwidth]{chapters/III-resultados-y-discusion/resources/images/esquema-modelo-dimensional.png}
    \caption{Esquema del modelo dimensional para el sistema de reportería de incidentes delictivos}
    \label{fig:esquema-modelo-dimensional}
\end{figure}

% \paragraph{Proceso ETL}

% Una vez definido el modelo dimensional, se procedió a diseñar el proceso de extracción, transformación y carga (ETL) de los datos.
% Para ello, se utilizó la herramienta de ETL de visual studio, la cual permite extraer datos de diferentes fuentes, transformarlos
% y cargarlos en el modelo dimensional. En la Figura \ref{fig:etl-bi} se muestra el diseño del proceso ETL para el modelo dimensional.

% \begin{figure}[H]
%     \centering
%     \includegraphics[width=0.8\textwidth]{chapters/III-resultados-y-discusion/resources/images/etl-bi.png}
%     \caption{Diseño del proceso ETL para el modelo dimensional.}
%     \label{fig:etl-bi}
% \end{figure}

% El proceso ETL inicia con la limpieza de las tablas que componen el modelo dimensional. Esto se realiza mediante la ejecución de un
% script SQL que elimina los registros de las tablas de hechos y dimensiones, como se puede observar en la Figura\ref{fig:limpieza-bi}.
% Una vez limpias las tablas, se procede a extraer los datos de las fuentes de datos. En este caso, se utilizó un origen de datos de
% ADO.NET para extraer los datos desde PostgreSQL. Además, se utilizó el componente de Data Conversion para convertir los datos a un
% formato compatible con el destino en una base de datos de SQL Server. Finalmente, se cargan los datos mediante el componente de OLE
% DB Destination, como se puede observar en la Figura \ref{fig:extraccion-bi}.

% \begin{figure}[H]
%     \centering
%     \includegraphics[width=0.8\textwidth]{chapters/III-resultados-y-discusion/resources/images/limpieza-bi.png}
%     \caption{Limpieza de las tablas del modelo dimensional.}
%     \label{fig:limpieza-bi}
% \end{figure}

% \begin{figure}[H]
%     \centering
%     \includegraphics[width=0.8\textwidth]{chapters/III-resultados-y-discusion/resources/images/extraccion-bi.png}
%     \caption{Extracción de datos para el modelo dimensional.}
%     \label{fig:extraccion-bi}
% \end{figure}

% Al ejecutar el proceso ETL, se comienza con la extracción siguiendo el flujo de datos definido en el proceso ETL. En caso de no
% existir ningún error, el sistema mostrará un mensaje de éxito, como se puede observar en la Figura \ref{fig:exito-bi}.

% \begin{figure}[H]
%     \centering
%     \includegraphics[width=0.8\textwidth]{chapters/III-resultados-y-discusion/resources/images/exito-bi.png}
%     \caption{Mensaje de éxito al ejecutar el proceso ETL.}
%     \label{fig:exito-bi}
% \end{figure}

% \pagebreak{Cubo OLAP}
% Para la construcción del cubo OLAP se empleó Visual Studio Community 2022, creando un proyecto de Analysis Services. En este
% proyecto, se definió la conexión a la base de datos de SQL Server, como se muestra en la Figura \ref{fig:conexion-olap},
% se creó un origen de datos, como se puede observar en la Figura \ref{fig:origen-datos-olap}, y se diseñó el cubo OLAP, como se
% puede visualizar en la Figura \ref{fig:cubo-olap}.

% \begin{figure}[H]
%     \centering
%     \includegraphics[width=0.8\textwidth]{chapters/III-resultados-y-discusion/resources/images/conexion-olap.png}
%     \caption{Conexión a la base de datos de SQL Server en el proyecto de Analysis Services.}
%     \label{fig:conexion-olap}
% \end{figure}

% \begin{figure}[H]
%     \centering
%     \includegraphics[width=0.8\textwidth]{chapters/III-resultados-y-discusion/resources/images/origen-datos-olap.png}
%     \caption{Origen de datos en el proyecto de Analysis Services.}
%     \label{fig:origen-datos-olap}
% \end{figure}

% \begin{figure}[H]
%     \centering
%     \includegraphics[width=0.8\textwidth]{chapters/III-resultados-y-discusion/resources/images/cubo-olap.png}
%     \caption{Diseño del cubo OLAP en el proyecto de Analysis Services.}
%     \label{fig:cubo-olap}
% \end{figure}

% En el origen de datos se definieron campos creados mediante cálculos con nombre que permiten agregar medidas e información
% adicional al cubo OLAP, como se puede observar en la Figura \ref{fig:campos-origen-datos-olap}.

% \begin{figure}[H]
%     \centering
%     \includegraphics[width=0.8\textwidth]{chapters/III-resultados-y-discusion/resources/images/campos-origen-datos-olap.png}
%     \caption{Campos en el origen de datos del proyecto de Analysis Services.}
%     \label{fig:campos-origen-datos-olap}
% \end{figure}

% En la dimension de tiempo se definieron jerarquías de tiempo que permiten visualizar los datos de forma secuencial en años, meses,
% días y horas, como se puede observar en la Figura \ref{fig:dimension-tiempo-olap}.

% \begin{figure}[H]
%     \centering
%     \includegraphics[width=0.8\textwidth]{chapters/III-resultados-y-discusion/resources/images/dimension-tiempo-olap.png}
%     \caption{Jerarquías de tiempo en la dimensión de tiempo del proyecto de Analysis Services.}
%     \label{fig:dimension-tiempo-olap}
% \end{figure}

% Una vez definido el cubo OLAP, se procedió a procesarlo para generar las medidas e indicadores definidos en el modelo dimensional, terminado
% el proceso de procesamiento se muestra un mensaje de éxito y se muestra el examinador de cubos en el cual se pueden realizar consultas a los
% datos del cubo OLAP, como se puede observar en la Figura \ref{fig:exito-olap}.

% \begin{figure}[H]
%     \centering
%     \includegraphics[width=0.8\textwidth]{chapters/III-resultados-y-discusion/resources/images/exito-olap.png}
%     \caption{Mensaje de éxito al procesar el cubo OLAP.}
%     \label{fig:exito-olap}
% \end{figure}

% \paragraph{Visualización de datos}
% Para la visualización de datos se utilizó Power BI, el cual permite conectar a diferentes fuentes de datos, como SQL Server, Analysis Services,
% Excel, entre otros. En este caso se lo conecto al cubo OLAP de Analysis Services, como se muestra en la Figura \ref{fig:conexion-bi}. Una vez
% realizada la conexión En Power BI se creó un informe en el cual se visualizan los datos del cubo  mediante gráficos, tablas y mapas de calor
% que permiten al usuario analizar la información de forma interactiva, como se muestra en la Figura \ref{fig:informe-bi}.

% \begin{figure}[H]
%     \centering
%     \includegraphics[width=0.8\textwidth]{chapters/III-resultados-y-discusion/resources/images/conexion-bi.png}
%     \caption{Conexión al cubo OLAP de Analysis Services en Power BI.}
%     \label{fig:conexion-bi}
% \end{figure}

% \begin{figure}[H]
%     \centering
%     \includegraphics[width=0.8\textwidth]{chapters/III-resultados-y-discusion/resources/images/informe-bi.png}
%     \caption{Informe en Power BI con los datos del cubo OLAP.}
%     \label{fig:informe-bi}
% \end{figure}

% El informe en Power BI cuenta con filtros interactivos que permiten al usuario filtrar los datos por tiempo (Año, Mes, Día, Hora), tipo de
% incidente, genero, etnia, estado civil, zona de vigilancia y edad de la víctima, como se puede observar en la Figura \ref{fig:filtros-bi}.

% \begin{figure}[H]
%     \centering
%     \includegraphics[width=0.8\textwidth]{chapters/III-resultados-y-discusion/resources/images/filtros-bi.png}
%     \caption{Filtros interactivos en el informe de Power BI.}
%     \label{fig:filtros-bi}
% \end{figure}

% \subsection{Cierre}
% El cierre del desarrollo del sistema se realizo mediante la puesta en producción de la aplicación web, móvil y el API junto al modelo analítico
% de BI. Para la puesta en producción de la aplicación web y el API se utilizó Railway, un servicio de alojamiento de aplicaciones web que permite
% desplegar aplicaciones de forma sencilla y rápida manejando el proceso de CI/CD al desplegar los sistemas desde su repositorio en GitHub, en las
% Figuras \ref{fig:despliegue-web} y \ref{fig:despliegue-api} se muestran los servicios desplegados en Railway.

% \begin{figure}[H]
%     \centering
%     \includegraphics[width=0.8\textwidth]{chapters/III-resultados-y-discusion/resources/images/despliegue-web.png}
%     \caption{Despliegue de la aplicación web en Railway.}
%     \label{fig:despliegue-web}
% \end{figure}

% \begin{figure}[H]
%     \centering
%     \includegraphics[width=0.8\textwidth]{chapters/III-resultados-y-discusion/resources/images/despliegue-api.png}
%     \caption{Despliegue del API en Railway.}
%     \label{fig:despliegue-api}
% \end{figure}

% Para la aplicación móvil se creo un archivo APK que permite instalar la aplicación en dispositivos Android, en la Figura \ref{fig:apk-movil}
% se muestra el archivo APK generado para la aplicación móvil.

% \begin{figure}[H]
%     \centering
%     \includegraphics[width=0.8\textwidth]{chapters/III-resultados-y-discusion/resources/images/apk-movil.png}
%     \caption{Archivo APK de la aplicación móvil.}
%     \label{fig:apk-movil}
% \end{figure}


% Para la puesta en producción del modelo analítico de BI se lo mantuvo en el servidor de SQL Server Analysis Services de forma local, esto
% debido a que no se contaba con un servidor de Analysis Services en la nube ni las licencias necesarias para obtener uno.



