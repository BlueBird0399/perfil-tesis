\subsection{Antecedentes investigativos}
En esta sección se muestran todos aquellos trabajos que proporcionan un contexto
precedente sobre el trabajo de investigación a realizarse.
\bigbreak
Con el propósito de obtener información relevante sobre el tema propuesto para este proyecto se realizó una
búsqueda exhaustiva en las revistas IEEE, SPRINGER LINK, así como en los repositorios de GOOGLE SCHOLAR
y LA UNIVERSIDAD TÉCNICA DE AMBATO en el periodo de 2015 a 2023, en la tabla \ref{table:articulosobtenidos}
se muestran los resultados
obtenidos:
% \begin{ThreePartTable}
%     \begin{TableNotes}[flushleft]
%         \centering
%         \item Elaborado por: el investigador
%     \end{TableNotes}
%     \begin{longtable}{|l|l|l|}
%         \caption{Artículos obtenidos}
%         \label{table:articulosobtenidos}                                                                  \\
%         \hline
%         \textbf{Cadena}                   & \textbf{Revista / Repositorio} & \textbf{Número de artículos} \\
%         \hline
%         (''georeferenced data''           & IEEE                           & 28                           \\ \cline{2-3}
%         OR ''spatiotemporal data''        & SPRINGER LINK                  & 417                          \\ \cline{2-3}
%         OR ''spatio-temporal              & GOOGLE SCHOLAR                 & 1.610                        \\ \cline{2-3}
%         data'' OR ''spatio                & UNIVERSIDAD                    & 0                            \\
%         temporal data'') AND              & TÉCNICA DE AMBATO              &                              \\ \cline{2-3}
%         (''crime analysis''               &                                &                              \\
%         OR crime OR felony OR             &                                &                              \\
%         (''business intelligence''        &                                &                              \\
%         OR BI OR ''analytical             &                                &                              \\
%         ''felony acts'') OR (date         &                                &                              \\
%         OR dates OR time OR  hour         &                                &                              \\
%         OR hours OR ''crime types'')      &                                &                              \\
%         AND  model'' OR ''analytical      &                                &                              \\
%         models'') AND (''crime quantity'' &                                &                              \\
%         OR ''crime count'' OR ''number    &                                &                              \\
%         of crimes'' OR ''total crimes'')  &                                &                              \\
%         \hline
%                                           & Total                          & 2055                         \\
%         \hline
%         \insertTableNotes
%     \end{longtable}
% \end{ThreePartTable}

\begin{longtable}{|l|l|l|}
    \caption{Artículos obtenidos} \label{tab:articulos-obtenidos}                                                                                            \\

    \hline \multicolumn{1}{|c|}{\textbf{Cadena}} & \multicolumn{1}{|c|}{\textbf{Revista / Repositorio}} & \multicolumn{1}{|c|}{\textbf{Número de artículos}} \\ \hline
    \endfirsthead

    \multicolumn{3}{c}%
    {{\normalfont \tablename\ \thetable{} -- continuación de la página anterior}}                                                                            \\
    \hline \multicolumn{1}{|c|}{\textbf{Cadena}} & \multicolumn{1}{|c|}{\textbf{Revista / Repositorio}} & \multicolumn{1}{|c|}{\textbf{Número de artículos}} \\ \hline
    \endhead

    \hline \multicolumn{3}{|r|}{{Continua en la siguiente página}}                                                                                           \\ \hline
    \endfoot

    \hline \hline
    \endlastfoot
    (''georeferenced data''                      & IEEE                                                 & 28                                                 \\ \cline{2-3}
    OR ''spatiotemporal data''                   & SPRINGER LINK                                        & 417                                                \\ \cline{2-3}
    OR ''spatio-temporal                         & GOOGLE SCHOLAR                                       & 1.610                                              \\ \cline{2-3}
    data'' OR ''spatio                           & UNIVERSIDAD                                          & 0                                                  \\ \cline{2-3}
    temporal data'') AND                         & TÉCNICA DE AMBATO                                    &                                                    \\ \cline{2-3}
    (''crime analysis''                          &                                                      &                                                    \\ \cline{2-3}
    OR crime OR felony OR                        &                                                      &                                                    \\ \cline{2-3}
    (''business intelligence''                   &                                                      &                                                    \\ \cline{2-3}
    OR BI OR ''analytical                        &                                                      &                                                    \\ \cline{2-3}
    ''felony acts'') OR (date                    &                                                      &                                                    \\ \cline{2-3}
    OR dates OR time OR  hour                    &                                                      &                                                    \\ \cline{2-3}
    OR hours OR ''crime types'')                 &                                                      &                                                    \\ \cline{2-3}
    AND  model'' OR ''analytical                 &                                                      &                                                    \\ \cline{2-3}
    models'') AND (''crime quantity''            &                                                      &                                                    \\ \cline{2-3}
    OR ''crime count'' OR ''number               &                                                      &                                                    \\ \cline{2-3}
    of crimes'' OR ''total crimes'')             &                                                      &                                                    \\
\end{longtable}
Como se muestra en la tabla \ref{table:articulosobtenidos}, mediante la revisión sistemática se obtuvo un total de 2055 artículos relacionados.
Sin embargo, Con la finalidad de obtener información relevante para el desarrollo de este proyecto se descartaron
aquellos artículos que no cumplan con los criterios de exclusión e inclusión que se muestran en la tabla \ref{table:criteriosinclusionexclusion}:
\begin{ThreePartTable}
    \begin{TableNotes}[flushleft]
        \centering
        \item Elaborado por: el investigador
    \end{TableNotes}
    \begin{longtable}{|l|l|}
        \caption{Criterios de exclusión e Inclusión}
        \label{table:criteriosinclusionexclusion}                                 \\
        \hline
        \textbf{Inclusión}          & \textbf{Exclusión}                          \\ \hline
        \tabitem Crimen             & \tabitem Solo artículos científicos y tesis \\
        \tabitem geointeligencia    & \tabitem Solo textos en español e inglés    \\
        \tabitem geodata            & \tabitem No contiene georreferenciación     \\
        \tabitem BI                 &                                             \\
        \tabitem Analítica          &                                             \\
        \tabitem Georreferenciación &                                             \\
        \hline
        \insertTableNotes
    \end{longtable}
\end{ThreePartTable}

Mediante la exclusión realizada se obtuvo un total de 20 artículos que contiene información sobre datos georreferenciados.
Aunque los artículos obtenidos no cumplen exactamente con el objetivo propuesto para esta investigación, se destaca a continuación
los trabajos más relevantes para el desarrollo de este proyecto.
% Mediante la revision sistemática de la bibliografía en revistas y repositorios digitales, se  no se encontraron trabajos que cumplan exactamente con el objetivo
% propuesto para esta investigación. Sin embargo, se destaca a continuación los trabajos con la información mas relevante para el desarrollo de este proyecto.
\bigbreak
Los autores de \cite{herreraGeoBIBigVGI2015} mencionan las ventajas del uso de los Sistemas de Información Geográfica (GIS).
Estas son herramientas masivas utilizadas para mostrar información de interés en mapas. Además, destacan las ventajas del empleo de
herramientas de Business Intelligence para el procesamiento de grandes cantidades de datos, especialmente en el contexto de
datos georreferenciados, denominado Geo Business Intelligence (GeoBI). Esta disciplina se encarga de abordar el análisis de datos
geoespaciales para mejorar la toma de decisiones en las organizaciones. En el trabajo se propone crear un sistema de Información
Geográfica Voluntaria, del inglés (VGI) el cual conste de un portal web. Este sistema habilitará el manejo de usuarios y visualización de mapas.
También se utilizará una aplicación cliente sea web o móvil, para ingresos de datos y una base de datos geográfica (GeoDB) con
la cual almacenar la información geográfica voluntaria. En el sistema se hará uso del framework "Monitor, Analyze and Drill to Detail"
(MAD), el cual permite separar las necesidades de los usuarios dependiendo al perfil que tengan asignado, como administradores,
analistas y entes de control de una organización. Se menciona que los sistemas GIS deberían proporcionar ciertos elementos básicos,
como portales web, aplicaciones clientes, servidores de mapas y bases de datos con capacidades de almacenamiento de datos geográficos.
Sin embargo, todo lo planteado en el trabajo es una propuesta por parte de los autores y no se muestra un prototipo funcional.
\bigbreak
En \cite{linUsingMachineLearning2017} el trabajo propuesto se centra en el uso de machine learning para la predicción de delitos,
específicamente en el uso de deep learning. En el trabajo se divide un mapa en cuadriculas, con el fin de verificar si en una cuadricula
se a detectado un crimen y tomarlo como un punto caliente. Para este modelo se tomaron en cuenta siete patrones espacio temporales,
las cuales son: mes, mes siguiente, año pasado, cuadrículas rodeadas en ocho direcciones en el mes actual, cuadrículas rodeadas en
ocho direcciones el año pasado, tendencia y proporción, más no se menciona de donde fueron extraídas tales características ni mediante
que medios. No obstante aunque no se trate el tema del software en la adquisición de los datos georreferenciados parad delitos, las
características que se muestran aportan información valiosa para ser utilizada en el contexto de modelos analíticos de BI.
\bigbreak
En \cite{CrimesPredictionUsing2019}, el autor propone un modelo de predicción de delitos basado en el uso de datos espacio-temporales
y el método Kernel Density Estimation (KDE), frecuentemente utilizado para estimar la densidad de crímenes. En el trabajo propuesto,
se detalla el flujo de trabajo para el preprocesamiento de los datos, que implica la eliminación de características irrelevantes y valores
geográficos fuera de la zona de estudio. El autor también menciona que el número de observaciones es de 1,762,311 y detalla las características
del dataset, como fecha, categoría del crimen, día de la semana, distrito policial, resolución del crimen, dirección, latitud y longitud.
A pesar de esto, en el trabajo no se menciona el uso de software para la adquisición de datos georreferenciados, ya que el dataset proviene de Kaggle.
No obstante, el artículo es importante para el desarrollo de este proyecto, ya que ofrece datos prácticos que pueden ser útiles en la creación
de modelos analíticos de Business Intelligence (BI).
\bigbreak
Los autores de \cite{winPCPDParallelCrime2019} proponen el algoritmo Criminal Activity Clustering (CAC) el cual está basado
en la clusterización difusa de puntos delictivos con la finalidad de agrupar grandes volúmenes de datos georreferenciados de delitos,
así como identificar patrones de comportamiento de los siniestros. En el trabajo se muestra una descripción de las características
utilizadas en el desarrollo del algoritmo, las cuales son: fecha, lugar del incidente, información del incidente, información del ataque,
información de la víctima, información del perpetrador, información del arma, causalidades y consecuencias, estas características son
obtenidas desde la base de Datos de Terrorismo Global, del inglés (GTD). Tales características propuestas son útiles para su uso en modelos
analíticos de BI. Sin embargo, el trabajo como tal no aborda el uso de software para la adquisición de datos georreferenciados o
su uso en BI.
\bigbreak
En el trabajo propuesto en \cite{amirkhanyanMethodsFrameworksGeoSpatioTemporal2019} no se aborda el uso de software para la
adquisición de datos georreferenciados de crímenes ni su utilización en modelos de BI. En su lugar, se tratan marcos para el
análisis de datos geoespaciales, los cuales incluyen la recopilación, normalización, geolocalización y almacenamiento de los datos.
Estos marcos permiten crear un algoritmo de clustering del lado del servidor para mapas, capaz de manejar una gran cantidad de datos
georreferenciados de forma masiva. En el trabajo se presentan los diagramas de entidad-relación con los tipos de datos utilizados
para desarrollar el algoritmo.
\bigbreak
En el trabajo presentado en \cite{kumarCrimePredictionUsing2020}, se propone un modelo de predicción de delitos basado en el uso
del algoritmo de K-Nearest Neighboring (KNN). Aunque en el trabajo propuesto no se considera el uso de software para la adquisición de datos
georreferenciados de delitos ni el desarrollo de un modelo analítico de BI, se muestra cómo los datos fueron preprocesados eliminando
valores nulos y datos innecesarios, obteniendo las siguientes características importantes: Hora, latitud, longitud, año, mes y semana del año.
Aun así, el autor menciona que los atributos obtenidos pueden no ser suficientes, por lo cual se deberían encontrar una mayor cantidad
de atributos importantes relacionados con los crímenes.
\bigbreak
En \cite{hossainCrimePredictionUsing2020} el autor se centra principalmente en el uso de algoritmos de aprendizaje supervisado
para la predicción de delitos. La finalidad del uso de estos algoritmos es proporcionar información temprana sobre la actividad
delictiva a entidades de control y agentes de la ley. El trabajo presenta los atributos propuestos para el dataset tales como:
fecha, categoría, día, distrito policial, resolución del crimen, dirección, latitud y longitud. Estos datos son extraídos del
sistema de reportes de incidentes del Departamento de Policía de San Francisco (SFPD). Sin embargo, el trabajo no aborda el uso
de software para la adquisición de datos georreferenciados de crímenes ni su uso para modelos analíticos de Business Intelligence
(BI). A pesar de ello, este trabajo resulta relevante para el desarrollo del proyecto actual, ya que proporciona datos útiles que
pueden ser empleados en la creación de un modelo analítico de BI.
\bigbreak
% El autor en \cite{moralesgutamaSistemaWebUsando2023} presenta el desarrollo de un sistema web el cual permita gestionar las
% calificaciones y asistencia en la unidad educativa Huachi Grande. Este sistema web esta desarrollado tanto su frontend y backend
% utilizando el framework Laravel con el lenguaje de programación PHP, junto a una base de datos en MySQL. Para el desarrollo de
% este proyecto de aplico la metodología de desarrollo ágil XP.
% \bigbreak
En \cite{botto-tobarAppliedTechnologiesSecond2021} se muestra un prototipo de una aplicación web desarrollada
utilizando el patrón de arquitectura cliente-servidor en Visual Studio 2017 Community, utilizando el lenguaje de programación C\#
junto al framework ASP.NET, también la aplicación utiliza el API de google map. Esta aplicación permite recolectar y mostrar datos
georreferenciados de delitos en la ciudad de Ambato. El trabajo propuesto exhibe el modelo de la base de datos, desarrollado en SQL
Server  y las interfaces propuestas para el prototipo. La aplicación desarrollada permite a los usuarios reportar delitos mediante
un formulario y muestra los puntos de siniestros mediante un mapa de calor de la ciudad. Además, permite visualizar reportes de los
delitos desde un punto de vista cualitativo y cuantitativo mediante el uso de la herramienta Power BI integrada en la aplicación web.
No obstante, el trabajo presentado no profundiza en el uso de los datos obtenidos para generar modelos analíticos de BI.
\bigbreak
En \cite{chasichangoAplicacionMovilApoyo2022} se destaca el uso de la georreferenciación en sistemas de seguridad en el cual
se propone una aplicación móvil desarrollada en IONIC usando el lenguaje de programación Javascript. La aplicación permite el
envío de geolocalización mediante notificaciones push utilizando el servicio de one signal. El backend de esta aplicación esta
desarrollado en NodeJS utilizando express y como base de datos utiliza MongoDB. El objetivo de esta aplicación es ayudar acometer
la problemática social de la inseguridad ciudadana en la parroquia Santa Rosa de la ciudad de Ambato.
\bigbreak
En \cite{lesanoperezAplicativoMovilGeoubicacion2022} se describe el desarrollo de una aplicación móvil que se encarga de enviar
la ubicación en tiempo real para el transporte privado de la empresa Plasticaucho S.A. de la ciudad de Ambato. Esta aplicación móvil
utiliza Xamarin Forms con el lenguaje de programación C\#. En el backend, se empleó el framework ASP.NET, también utilizando el lenguaje
de programación C\#, e implementando el patrón de arquitectura Model View Controller (MVC), junto a una base de datos en PostgreSQL. El proyecto se llevó a
cabo utilizando la metodología de desarrollo ágil Extreme programming (XP).
\bigbreak
En \cite{chicaizavillegasAplicacionWebPara2023}, se desarrolló una aplicación web que incorpora georreferenciación.
Esta aplicación utiliza el framework Angular junto con el lenguaje de programación JavaScript para el frontend. El backend fue
implementado mediante el framework ASP.NET utilizando el lenguaje de programación C\#, y se empleó una base de datos MySQL. Para
el desarrollo de este proyecto, se aplicó la metodología de desarrollo Extreme Programming (XP), orientada a grupos reducidos de
desarrollo. La aplicación creada permite gestionar servicios, notificaciones de pago y geolocalización de viviendas para la empresa
Optynet.
\bigbreak
Los autores en \cite{gomezcantilloAplicativoMovilPara} presentan el diseño y desarrollo de una aplicación móvil que aborda la
problemática de la predicción de crímenes. Esta aplicación se fundamenta en la integración de tecnologías modernas, la recopilación
de datos geoespaciales, el análisis de datos e inteligencia artificial. En el trabajo, se detallan las técnicas utilizadas para la
recopilación de información. Se menciona el empleo de la herramienta Taled Open Studio, la cual se utilizó para llevar a cabo el
proceso de Extracción, Transformación y Carga de datos (ETL), almacenándolos en una base de datos MySql. Además, el proyecto exhibe
las interfaces de la aplicación móvil, que constan de un módulo de autenticación, otro de reporte de delitos y un tercer módulo para
la visualización de estos. Dentro del módulo de visualización de delitos, se presentan mapas de calor que muestran los incidentes
ocurridos, agrupándolos por áreas mediante esferas de colores que representan el total de siniestros en cada zona. Por otro lado,
el módulo de reporte de delitos permite a los usuarios informar los incidentes de los que han sido víctimas mediante un formulario
y un mapa para señalar la ubicación del suceso. Además, la aplicación ofrece estadísticas sobre los incidentes reportados en distintos
periodos de tiempo para el conocimiento de la ciudadanía. Este trabajo proporciona un medio para la adquisición de datos georreferenciados
mediante el uso de software, así como una plataforma para mostrar información relevante a la ciudadanía. No obstante, no incluye detalles
sobre las características utilizadas ni los diagramas de la base de datos. Tampoco aborda el uso de modelos analíticos de Business
Intelligence (BI).
