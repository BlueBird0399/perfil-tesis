\begin{longtable}{|p{0.5cm}|p{6cm}|p{6cm}|}
    \caption[]{Ventajas y desventajas de Angular, ReactJS y VueJS \cite{xingResearchAnalysisFrontend2019a}} \label{tab:ventajas-desventajas-frameworks-web}                                                                                                                                                                                                                                                                                                                          \\

    \hline \multicolumn{1}{|c|}{\textbf{Framework}} & \multicolumn{1}{|c|}{\textbf{Ventajas}}                                                                                                                                                                           & \multicolumn{1}{|c|}{\textbf{Desventajas}}                                                                                                                                                                 \\ \hline
    \endfirsthead

    \multicolumn{3}{c}%
    {{\normalfont \tablename\ \thetable{} -- continuación de la página anterior}}                                                                                                                                                                                                                                                                                                                                                                                                    \\
    \hline \multicolumn{1}{|c|}{\textbf{Framework}} & \multicolumn{1}{|c|}{\textbf{Ventajas}}                                                                                                                                                                           & \multicolumn{1}{|c|}{\textbf{Desventajas}}                                                                                                                                                                 \\ \hline
    \endhead

    \hline \multicolumn{3}{|r|}{{Continua en la siguiente página}}                                                                                                                                                                                                                                                                                                                                                                                                                   \\ \hline
    \endfoot

    \hline \hline
    \endlastfoot
    ReactJS                                         & \tabitem{Emplea un DOM virtual para lograr una eficiencia máxima al actualizar nodos según sea necesario.}                                                                                                        & \tabitem{Es necesario importar bibliotecas adicionales para manejar el estado y el modelo, ya que React no incluye la arquitectura MVC de forma nativa.}                                                   \\
                                                    & \tabitem{La capacidad de renderizar en el servidor es otra ventaja que este framework ofrece, especialmente adecuada para ciertos tipos de implementaciones, como las aplicaciones enfocadas en el contenido.}    & \tabitem{Aunque React permite su uso, se distancia de los enfoques basados en clases y puede presentar dificultades para aquellos que prefieren la Programación Orientada a Objetos (POO).}                \\
                                                    & \tabitem{Reduce la carga de recursos del usuario mediante el respaldo de bundling y tree shaking.}                                                                                                                &                                                                                                                                                                                                            \\
                                                    & \tabitem{La programación funcional facilita la creación de código que puede ser reutilizado. }                                                                                                                    &                                                                                                                                                                                                            \\
                                                    & \tabitem{Ofrece ventajas en términos de SEO en comparación con Angular y Vue.js. }                                                                                                                                &                                                                                                                                                                                                            \\
    Angular                                         & \tabitem{Utiliza el patrón MVVM (Modelo-Vista-Modelo de Vista), el cual permite manipular la misma colección de datos de forma independiente dentro de una misma aplicación.}                                     & \tabitem{Posee múltiples estructuras como Inyectables, Componentes, Tuberías, Módulos, entre otros, que suelen presentar un mayor nivel de complejidad para su comprensión.}                               \\
                                                    & \tabitem{Su estructura y arquitectura están diseñadas específicamente para mejorar la escalabilidad de los proyectos.}                                                                                            & \tabitem{Experimenta actualizaciones continuas, incorporando mejoras nuevas y significativas de manera constante. Sin embargo, estas actualizaciones pueden plantear desafíos al adaptarse a los cambios.} \\
                                                    & \tabitem{La inyección de dependencias en los componentes ayuda a mejorar la modularidad de la aplicación.}                                                                                                        &                                                                                                                                                                                                            \\
                                                    & \tabitem{La programación funcional facilita la creación de código que puede ser reutilizado. }                                                                                                                    &                                                                                                                                                                                                            \\
    VueJS                                           & \tabitem{Facilita la creación de modelos modulares de gran alcance que pueden renderizarse eficientemente gracias a su estructura fundamental, sin requerir esfuerzos adicionales.}                               & \tabitem{Su participación en el mercado es moderada, lo que indica que el intercambio de información en esta plataforma se encuentra en sus primeras fases de desarrollo.}                                 \\
                                                    & \tabitem{Posee una alta capacidad de respuesta y ofrece una sencilla vinculación de datos entre el código HTML y JavaScript. }                                                                                    & \tabitem{Existe el riesgo de que su flexibilidad pueda ser un problema al integrarse en proyectos extensos debido a la carencia de recursos disponibles.}                                                  \\
                                                    & \tabitem{Vue gestiona de forma sobresaliente la vinculación de datos bidireccional dinámica. Además, lleva a cabo la manipulación del DOM de manera coherente, lo que lo hace ideal para diversas aplicaciones. } &                                                                                                                                                                                                            \\
\end{longtable}