\bigbreak

En el proyecto propuesto para las aplicaciones cliente, se empleará una aplicación móvil, dado que esta puede ser utilizada en una gran variedad de
dispositivos, tanto Android como iOS. La aplicación móvil empleará una arquitectura Model View Controller (MVC), con la cual se separará la lógica de
negocio de la interfaz de usuario y la interacción. También se utilizará un framework web, dado que este ofrece una estructura de trabajo establecida,
además de incluir características extras y optimizaciones que facilitan el desarrollo de aplicaciones web. El sistema web se desarrollará utilizando el
patrón de arquitectura MVC, dado que este permite dividir la lógica y responsabilidad en 3 capas y así mantener una mejor estructura en el código.
Para el desarrollo del backend, se emplearán frameworks que agilicen la creación de un API REST eficiente y escalable, el cual permita la comunicación
entre diferentes sistemas, como aplicaciones web y móviles, mediante el protocolo HTTP. Para el almacenamiento de información, se empleará una base de
datos relacional, dado que esta ofrece un gran rendimiento y fiabilidad al manejar grandes volúmenes de datos. Además, permite manejar relaciones entre
diferentes tablas, las cuales ayudan a recuperar datos de forma eficiente, garantizando la integridad referencial. Para el desarrollo del proyecto, se
utilizará una metodología de desarrollo ágil por la facilidad, flexibilidad y versatilidad que ofrece al realizar cambios durante el desarrollo. Además,
mejora los tiempos de desarrollo y la calidad del producto final.

\subsection{Modalidad de investigación}
En esta sección se muestran los enfoques a los que se encuentra orientada la investigación propuesta para este proyecto. A continuación se detalla
cada una de ellas.
\bigbreak
\textbf{Investigación cuantitativa}\\

En el presente proyecto se emplea una metodología de investigación cuantitativa. Esta se centra en la recopilación de
datos numéricos y estadísticos que permiten establecer las tendencias de siniestralidad a través del análisis del número
de crímenes. Esta información se utiliza como base fundamental para el diseño de un modelo analítico de BI.

\bigbreak
\textbf{Investigación aplicada}\\
Se emplea una investigación aplicada, ya que el desarrollo del proyecto propuesto utilizará la teoría, métodos,
técnicas, tecnologías y conocimientos adquiridos a lo largo de ciclos académicos previos. Este enfoque permitirá
aplicar de manera práctica y específica el conocimiento acumulado durante la trayectoria educativa.

% \bigbreak
% \textbf{Investigación bibliográfica}\\
% La Investigación es Bibliográfica y Documental porque es necesario recopilar información de
% documentos como artículos académicos, tesis y libros que sirvan de apoyo
% para la contextualización de la propuesta a desarrollar.

% \bigbreak
% \textbf{Investigación de campo}\\
% La Investigación es de Campo ya que la información y características del problema
% serán extraídas mediante el contacto directo en el lugar de los hechos, es decir, el
% Centro de Transferencia y Desarrollo de Tecnologías y la Dirección de Vinculación
% con la Sociedad.

% \bigbreak
% \textbf{Investigación aplicada}\\
% La Investigación es Aplicada debido a que se emplearán conocimientos
% adquiridos a lo largo de la carrera para el desarrollo de la propuesta.