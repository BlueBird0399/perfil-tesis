En la Tabla \ref{tab:dimension-tiempo} se presenta los atributos seleccionados para la dimensión de tiempo.

\begin{longtable}{|p{6cm}|p{6cm}|}
    \caption{Dimensión de tiempo con sus atributos} \label{tab:dimension-tiempo}             \\

    \hline \multicolumn{1}{|c|}{\textbf{Campo}} & \multicolumn{1}{|c|}{\textbf{Descripción}} \\ \hline
    \endfirsthead

    \multicolumn{2}{c}%
    {{\normalfont \tablename\ \thetable{} -- continuación de la página anterior}}            \\
    \hline \multicolumn{1}{|c|}{\textbf{Campo}} & \multicolumn{1}{|c|}{\textbf{Descripción}} \\ \hline
    \endhead

    \hline \multicolumn{2}{|r|}{{Continua en la siguiente página}}                           \\ \hline
    \endfoot

    \hline \hline
    \endlastfoot
    fechaID                                     & Identificador único para cada fecha        \\\hline
    fecha                                       & Fecha completa del incidente (YYYY-MM-DD)  \\\hline
    anio                                        & Año en que ocurrió el incidente            \\\hline
    mes                                         & Mes en que ocurrió el incidente            \\\hline
    día                                         & Día en que ocurrió el incidente            \\\hline
    trimestre                                   & Trimestre en que ocurrió el incidente      \\\hline
    semestre                                    & Semestre en que ocurrió el incidente       \\\hline
    hora                                        & Hora en que ocurrió el incidente           \\
\end{longtable}

En la Tabla \ref{tab:dimension-usuarios} se presenta los atributos seleccionados para la dimensión de usuarios.

\begin{longtable}{|p{6cm}|p{6cm}|}
    \caption{Dimensión de usuarios con sus atributos} \label{tab:dimension-usuarios}         \\

    \hline \multicolumn{1}{|c|}{\textbf{Campo}} & \multicolumn{1}{|c|}{\textbf{Descripción}} \\ \hline
    \endfirsthead

    \multicolumn{2}{c}%
    {{\normalfont \tablename\ \thetable{} -- continuación de la página anterior}}            \\
    \hline \multicolumn{1}{|c|}{\textbf{Campo}} & \multicolumn{1}{|c|}{\textbf{Descripción}} \\ \hline
    \endhead

    \hline \multicolumn{2}{|r|}{{Continua en la siguiente página}}                           \\ \hline
    \endfoot

    \hline \hline
    \endlastfoot
    usuarioID                                   & Identificador único del usuario            \\\hline
    género                                      & Género del usuario                         \\\hline
    discapacidad                                & Estado de discapacidad del usuario         \\\hline
    etnia                                       & Etnia del usuario                          \\\hline
    estadoCivil                                 & Estado civil del usuario                   \\\hline
    edad                                        & Edad del usuario                           \\
\end{longtable}

En la Tabla \ref{tab:dimension-tipo-de-alarma} se presenta los atributos seleccionados para la dimensión de tipo de alarma.

\begin{longtable}{|p{6cm}|p{6cm}|}
    \caption{Dimensión de tipo de alarma con sus atributos} \label{tab:dimension-tipo-de-alarma} \\

    \hline \multicolumn{1}{|c|}{\textbf{Campo}} & \multicolumn{1}{|c|}{\textbf{Descripción}}     \\ \hline
    \endfirsthead

    \multicolumn{2}{c}%
    {{\normalfont \tablename\ \thetable{} -- continuación de la página anterior}}                \\
    \hline \multicolumn{1}{|c|}{\textbf{Campo}} & \multicolumn{1}{|c|}{\textbf{Descripción}}     \\ \hline
    \endhead

    \hline \multicolumn{2}{|r|}{{Continua en la siguiente página}}                               \\ \hline
    \endfoot

    \hline \hline
    \endlastfoot
    tipoAlarmaID                                & Identificador único del tipo de alarma         \\\hline
    nombreTipoAlarma                            & Nombre del tipo de alarma                      \\
\end{longtable}

En la Tabla \ref{tab:dimension-tipo-de-incidente} se presenta los atributos seleccionados para la dimensión de tipo de incidente.

\begin{longtable}{|p{6cm}|p{6cm}|}
    \caption{Dimensión de tipo de incidente con sus atributos} \label{tab:dimension-tipo-de-incidente} \\

    \hline \multicolumn{1}{|c|}{\textbf{Campo}} & \multicolumn{1}{|c|}{\textbf{Descripción}}           \\ \hline
    \endfirsthead

    \multicolumn{2}{c}%
    {{\normalfont \tablename\ \thetable{} -- continuación de la página anterior}}                      \\
    \hline \multicolumn{1}{|c|}{\textbf{Campo}} & \multicolumn{1}{|c|}{\textbf{Descripción}}           \\ \hline
    \endhead

    \hline \multicolumn{2}{|r|}{{Continua en la siguiente página}}                                     \\ \hline
    \endfoot

    \hline \hline
    \endlastfoot
    tipoIncidenteID                             & Identificador único del tipo de incidente            \\\hline
    nombreTipoIncidente                         & Nombre del tipo de incidente                         \\
\end{longtable}

En la Tabla \ref{tab:dimension-zonas-vigilancia} se presenta los atributos seleccionados para la dimensión de zonas de vigilancia.

\begin{longtable}{|p{6cm}|p{6cm}|}
    \caption{Dimensión de zonas de vigilancia con sus atributos} \label{tab:dimension-zonas-vigilancia} \\

    \hline \multicolumn{1}{|c|}{\textbf{Campo}} & \multicolumn{1}{|c|}{\textbf{Descripción}}            \\ \hline
    \endfirsthead

    \multicolumn{2}{c}%
    {{\normalfont \tablename\ \thetable{} -- continuación de la página anterior}}                       \\
    \hline \multicolumn{1}{|c|}{\textbf{Campo}} & \multicolumn{1}{|c|}{\textbf{Descripción}}            \\ \hline
    \endhead

    \hline \multicolumn{2}{|r|}{{Continua en la siguiente página}}                                      \\ \hline
    \endfoot

    \hline \hline
    \endlastfoot
    zonaVigilanciaID                            & Identificador único de la zona de vigilancia          \\\hline
    polígono                                    & Representación geográfica de la zona de vigilancia    \\\hline
    nombreZonaVigilancia                        & Nombre de la zona de vigilancia                       \\
\end{longtable}

En la Tabla \ref{tab:dimension-ubicacion} se presenta los atributos seleccionados para la dimensión de ubicación.

\begin{longtable}{|p{6cm}|p{6cm}|}
    \caption{Dimensión de ubicación con sus atributos} \label{tab:dimension-ubicacion}       \\

    \hline \multicolumn{1}{|c|}{\textbf{Campo}} & \multicolumn{1}{|c|}{\textbf{Descripción}} \\ \hline
    \endfirsthead

    \multicolumn{2}{c}%
    {{\normalfont \tablename\ \thetable{} -- continuación de la página anterior}}            \\
    \hline \multicolumn{1}{|c|}{\textbf{Campo}} & \multicolumn{1}{|c|}{\textbf{Descripción}} \\ \hline
    \endhead

    \hline \multicolumn{2}{|r|}{{Continua en la siguiente página}}                           \\ \hline
    \endfoot

    \hline \hline
    \endlastfoot
    ubicacionID                                 & Identificador único de la ubicación        \\\hline
    canton                                      & Cantón del lugar del incidente             \\\hline
    ciudad                                      & Ciudad del lugar del incidente             \\
\end{longtable}