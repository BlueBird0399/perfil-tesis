En la tabla \ref{tab:crc-1} se presenta la tarjeta CRC de la historia de usuario 1, la cual corresponde a la funcionalidad de
iniciar sesión en el sistema web.

\begin{longtable}{|p{5cm}|p{5cm}|}
      \caption{Tarjeta CRC - Historia 1: Iniciar sesión sistema web} \label{tab:crc-1}                       \\

      \hline \multicolumn{2}{|c|}{\textbf{Iniciar sesión sistema web}}                                       \\ \hline
      \hline \multicolumn{1}{|c|}{\textbf{Responsabilidades}} & \multicolumn{1}{|c|}{\textbf{Colaboradores}} \\ \hline
      \endfirsthead

      \multicolumn{2}{c}%
      {{\normalfont \tablename\ \thetable{} -- continuación de la página anterior}}                          \\
      \hline \multicolumn{1}{|c|}{\textbf{Responsabilidades}} & \multicolumn{1}{|c|}{\textbf{Colaboradores}} \\ \hline
      \endhead

      \hline \multicolumn{2}{|r|}{{Continua en la siguiente página}}                                         \\ \hline
      \endfoot

      \hline \hline
      \endlastfoot
      Diseño de la base de datos                              & Usuario                                      \\\hline
      Desarrollo backend para inicio de sesión                & Roles                                        \\\hline
      Diseño de la interfaz para iniciar sesión               &                                              \\\hline
      \multicolumn{2}{|c|}{\textbf{Observaciones:} Ninguna}                                                  \\
\end{longtable}

En la tabla \ref{tab:crc-2} se presenta la tarjeta CRC de la historia de usuario 2, la cual corresponde a la funcionalidad de
cerrar sesión en el sistema web.

\begin{longtable}{|p{5cm}|p{5cm}|}
      \caption{Tarjeta CRC - Historia 2: Cerrar sesión sistema web} \label{tab:crc-2}                        \\

      \hline \multicolumn{2}{|c|}{\textbf{Cerrar sesión sistema web}}                                        \\ \hline
      \hline \multicolumn{1}{|c|}{\textbf{Responsabilidades}} & \multicolumn{1}{|c|}{\textbf{Colaboradores}} \\ \hline
      \endfirsthead

      \multicolumn{2}{c}%
      {{\normalfont \tablename\ \thetable{} -- continuación de la página anterior}}                          \\
      \hline \multicolumn{1}{|c|}{\textbf{Responsabilidades}} & \multicolumn{1}{|c|}{\textbf{Colaboradores}} \\ \hline
      \endhead

      \hline \multicolumn{2}{|r|}{{Continua en la siguiente página}}                                         \\ \hline
      \endfoot

      \hline \hline
      \endlastfoot
      Diseño de la interfaz para cerrar sesión                & Usuario                                      \\\hline
      \multicolumn{2}{|c|}{\textbf{Observaciones:} Ninguna}                                                  \\
\end{longtable}

En la tabla \ref{tab:crc-3} se presenta la tarjeta CRC de la historia de usuario 3, la cual corresponde a la funcionalidad de
gestionar usuarios.

\begin{longtable}{|p{5cm}|p{5cm}|}
      \caption{Tarjeta CRC - Historia 3: Gestionar usuarios} \label{tab:crc-3}                               \\

      \hline \multicolumn{2}{|c|}{\textbf{Gestionar usuarios}}                                               \\ \hline
      \hline \multicolumn{1}{|c|}{\textbf{Responsabilidades}} & \multicolumn{1}{|c|}{\textbf{Colaboradores}} \\ \hline
      \endfirsthead

      \multicolumn{2}{c}%
      {{\normalfont \tablename\ \thetable{} -- continuación de la página anterior}}                          \\
      \hline \multicolumn{1}{|c|}{\textbf{Responsabilidades}} & \multicolumn{1}{|c|}{\textbf{Colaboradores}} \\ \hline
      \endhead

      \hline \multicolumn{2}{|r|}{{Continua en la siguiente página}}                                         \\ \hline
      \endfoot

      \hline \hline
      \endlastfoot
      Diseño de la base de datos para la gestión de usuarios  & Usuario                                      \\\hline
      Desarrollo backend para la gestión de usuarios          & Roles                                        \\\hline
      Desarrollo de la interfaz para gestión de usuarios      &                                              \\\hline
      \multicolumn{2}{|c|}{\textbf{Observaciones:} Ninguna}                                                  \\
\end{longtable}

En la tabla \ref{tab:crc-4} se presenta la tarjeta CRC de la historia de usuario 4, la cual corresponde a la funcionalidad de
gestionar tipos de incidentes.

\begin{longtable}{|p{5cm}|p{5cm}|}
      \caption{Tarjeta CRC - Historia 4: Gestionar tipos de incidentes} \label{tab:crc-4}                     \\

      \hline \multicolumn{2}{|c|}{\textbf{Gestionar tipos de incidentes}}                                     \\ \hline
      \hline \multicolumn{1}{|c|}{\textbf{Responsabilidades}}  & \multicolumn{1}{|c|}{\textbf{Colaboradores}} \\ \hline
      \endfirsthead

      \multicolumn{2}{c}%
      {{\normalfont \tablename\ \thetable{} -- continuación de la página anterior}}                           \\
      \hline \multicolumn{1}{|c|}{\textbf{Responsabilidades}}  & \multicolumn{1}{|c|}{\textbf{Colaboradores}} \\ \hline
      \endhead

      \hline \multicolumn{2}{|r|}{{Continua en la siguiente página}}                                          \\ \hline
      \endfoot

      \hline \hline
      \endlastfoot
      Diseño de la base de datos para la gestión de incidentes & Tipo de incidente                            \\\hline
      Desarrollo backend para la gestión de incidentes         & Policía                                      \\\hline
      Desarrollo de la interfaz para la gestión de incidentes  &                                              \\\hline
      \multicolumn{2}{|c|}{\textbf{Observaciones:} Ninguna}                                                   \\
\end{longtable}

En la tabla \ref{tab:crc-5} se presenta la tarjeta CRC de la historia de usuario 5, la cual corresponde a la funcionalidad de
gestionar zonas de vigilancia.

\begin{longtable}{|p{5cm}|p{5cm}|}
      \caption{Tarjeta CRC - Historia 5: Gestionar zonas de vigilancia} \label{tab:crc-5}                              \\

      \hline \multicolumn{2}{|c|}{\textbf{Gestionar zonas de vigilancia}}                                              \\ \hline
      \hline \multicolumn{1}{|c|}{\textbf{Responsabilidades}}           & \multicolumn{1}{|c|}{\textbf{Colaboradores}} \\ \hline
      \endfirsthead

      \multicolumn{2}{c}%
      {{\normalfont \tablename\ \thetable{} -- continuación de la página anterior}}                                    \\
      \hline \multicolumn{1}{|c|}{\textbf{Responsabilidades}}           & \multicolumn{1}{|c|}{\textbf{Colaboradores}} \\ \hline
      \endhead

      \hline \multicolumn{2}{|r|}{{Continua en la siguiente página}}                                                   \\ \hline
      \endfoot

      \hline \hline
      \endlastfoot
      Diseño de la base de datos para la gestión de zonas de vigilancia & Zonas de vigilancia                          \\\hline
      Desarrollo backend para la gestión de zonas de vigilancia         & Policía                                      \\\hline
      Desarrollo de la interfaz para la gestión de zonas de vigilancia  &                                              \\\hline
      \multicolumn{2}{|c|}{\textbf{Observaciones:} Ninguna}                                                            \\
\end{longtable}

En la tabla \ref{tab:crc-6} se presenta la tarjeta CRC de la historia de usuario 6, la cual corresponde a la funcionalidad de
asignar policías a las zonas de vigilancia.

\begin{longtable}{|p{5cm}|p{5cm}|}
      \caption{Tarjeta CRC - Historia 6: Asignar policías a las zonas de vigilancia} \label{tab:crc-6}                          \\

      \hline \multicolumn{2}{|c|}{\textbf{Asignar policías a las zonas de vigilancia}}                                          \\ \hline
      \hline \multicolumn{1}{|c|}{\textbf{Responsabilidades}}                    & \multicolumn{1}{|c|}{\textbf{Colaboradores}} \\ \hline
      \endfirsthead

      \multicolumn{2}{c}%
      {{\normalfont \tablename\ \thetable{} -- continuación de la página anterior}}                                             \\
      \hline \multicolumn{1}{|c|}{\textbf{Responsabilidades}}                    & \multicolumn{1}{|c|}{\textbf{Colaboradores}} \\ \hline
      \endhead

      \hline \multicolumn{2}{|r|}{{Continua en la siguiente página}}                                                            \\ \hline
      \endfoot

      \hline \hline
      \endlastfoot
      Diseño de la base de datos para asignar policías a las zonas de vigilancia & Zonas de vigilancia                          \\\hline
      Desarrollo backend para asignar policías a las zonas de vigilancia         & Policía                                      \\\hline
      Desarrollo de la interfaz para asignar policías a las zonas de vigilancia  & Zonas de vigilancia                          \\\hline
      \multicolumn{2}{|c|}{\textbf{Observaciones:} Ninguna}                                                                     \\
\end{longtable}

En la tabla \ref{tab:crc-7} se presenta la tarjeta CRC de la historia de usuario 7, la cual corresponde a la funcionalidad de
gestionar alertas de incidentes.

\begin{longtable}{|p{5cm}|p{5cm}|}
      \caption{Tarjeta CRC - Historia 7: Gestionar alertas de incidentes} \label{tab:crc-7}                          \\

      \hline \multicolumn{2}{|c|}{\textbf{Gestionar alertas de incidentes}}                                          \\ \hline
      \hline \multicolumn{1}{|c|}{\textbf{Responsabilidades}}         & \multicolumn{1}{|c|}{\textbf{Colaboradores}} \\ \hline
      \endfirsthead

      \multicolumn{2}{c}%
      {{\normalfont \tablename\ \thetable{} -- continuación de la página anterior}}                                  \\
      \hline \multicolumn{1}{|c|}{\textbf{Responsabilidades}}         & \multicolumn{1}{|c|}{\textbf{Colaboradores}} \\ \hline
      \endhead

      \hline \multicolumn{2}{|r|}{{Continua en la siguiente página}}                                                 \\ \hline
      \endfoot

      \hline \hline
      \endlastfoot
      Diseño de la base de datos para Gestionar alertas de incidentes & Tipo de incidente                            \\\hline
      Desarrollo backend para Gestionar alertas de incidentes         & Policía                                      \\\hline
      Desarrollo de la interfaz para Gestionar alertas de incidentes  &                                              \\\hline
      \multicolumn{2}{|c|}{\textbf{Observaciones:} Ninguna}                                                          \\
\end{longtable}

En la tabla \ref{tab:crc-8} se presenta la tarjeta CRC de la historia de usuario 8, la cual corresponde a la funcionalidad de
visualizar mapa de calor.

\begin{longtable}{|p{5cm}|p{5cm}|}
      \caption{Tarjeta CRC - Historia 8: Visualizar mapa de calor} \label{tab:crc-8}                         \\

      \hline \multicolumn{2}{|c|}{\textbf{Visualizar mapa de calor}}                                         \\ \hline
      \hline \multicolumn{1}{|c|}{\textbf{Responsabilidades}} & \multicolumn{1}{|c|}{\textbf{Colaboradores}} \\ \hline
      \endfirsthead

      \multicolumn{2}{c}%
      {{\normalfont \tablename\ \thetable{} -- continuación de la página anterior}}                          \\
      \hline \multicolumn{1}{|c|}{\textbf{Responsabilidades}} & \multicolumn{1}{|c|}{\textbf{Colaboradores}} \\ \hline
      \endhead

      \hline \multicolumn{2}{|r|}{{Continua en la siguiente página}}                                         \\ \hline
      \endfoot

      \hline \hline
      \endlastfoot
      Desarrollo backend para visualizar mapa de calor        & Zonas de vigilancia                          \\\hline
      Desarrollo de la interfaz para visualizar mapa de calor &                                              \\\hline
      \multicolumn{2}{|c|}{\textbf{Observaciones:} Ninguna}                                                  \\
\end{longtable}

En la tabla \ref{tab:crc-9} se presenta la tarjeta CRC de la historia de usuario 9, la cual corresponde a la funcionalidad de
iniciar sesión en la aplicación móvil.

\begin{longtable}{|p{5cm}|p{5cm}|}
      \caption{Tarjeta CRC - Historia 9: Iniciar sesión aplicación móvil} \label{tab:crc-9}                               \\

      \hline \multicolumn{2}{|c|}{\textbf{Iniciar sesión aplicación móvil}}                                               \\ \hline
      \hline \multicolumn{1}{|c|}{\textbf{Responsabilidades}}              & \multicolumn{1}{|c|}{\textbf{Colaboradores}} \\ \hline
      \endfirsthead

      \multicolumn{2}{c}%
      {{\normalfont \tablename\ \thetable{} -- continuación de la página anterior}}                                       \\
      \hline \multicolumn{1}{|c|}{\textbf{Responsabilidades}}              & \multicolumn{1}{|c|}{\textbf{Colaboradores}} \\ \hline
      \endhead

      \hline \multicolumn{2}{|r|}{{Continua en la siguiente página}}                                                      \\ \hline
      \endfoot

      \hline \hline
      \endlastfoot
      Desarrollo de la interfaz para iniciar sesión en la aplicación móvil & Usuario                                      \\\hline
      \multicolumn{2}{|c|}{\textbf{Observaciones:} Ninguna}                                                               \\
\end{longtable}

En la tabla \ref{tab:crc-10} se presenta la tarjeta CRC de la historia de usuario 10, la cual corresponde a la funcionalidad de
cerrar sesión en la aplicación móvil.

\begin{longtable}{|p{5cm}|p{5cm}|}
      \caption{Tarjeta CRC - Historia 10: Cerrar sesión aplicación móvil} \label{tab:crc-10}                         \\

      \hline \multicolumn{2}{|c|}{\textbf{Cerrar sesión aplicación móvil}}                                           \\ \hline
      \hline \multicolumn{1}{|c|}{\textbf{Responsabilidades}}         & \multicolumn{1}{|c|}{\textbf{Colaboradores}} \\ \hline
      \endfirsthead

      \multicolumn{2}{c}%
      {{\normalfont \tablename\ \thetable{} -- continuación de la página anterior}}                                  \\
      \hline \multicolumn{1}{|c|}{\textbf{Responsabilidades}}         & \multicolumn{1}{|c|}{\textbf{Colaboradores}} \\ \hline
      \endhead

      \hline \multicolumn{2}{|r|}{{Continua en la siguiente página}}                                                 \\ \hline
      \endfoot

      \hline \hline
      \endlastfoot
      Diseño de la interfaz para cerrar sesión en la aplicación móvil & Usuario                                      \\\hline
      \multicolumn{2}{|c|}{\textbf{Observaciones:} Ninguna}                                                          \\
\end{longtable}

En la tabla \ref{tab:crc-11} se presenta la tarjeta CRC de la historia de usuario 11, la cual corresponde a la funcionalidad de
registro de usuario.

\begin{longtable}{|p{5cm}|p{5cm}|}
      \caption{Tarjeta CRC - Historia 11: Registro de usuario} \label{tab:crc-11}                                                 \\

      \hline \multicolumn{2}{|c|}{\textbf{Registro de usuario}}                                                                   \\ \hline
      \hline \multicolumn{1}{|c|}{\textbf{Responsabilidades}}                      & \multicolumn{1}{|c|}{\textbf{Colaboradores}} \\ \hline
      \endfirsthead

      \multicolumn{2}{c}%
      {{\normalfont \tablename\ \thetable{} -- continuación de la página anterior}}                                               \\
      \hline \multicolumn{1}{|c|}{\textbf{Responsabilidades}}                      & \multicolumn{1}{|c|}{\textbf{Colaboradores}} \\ \hline
      \endhead

      \hline \multicolumn{2}{|r|}{{Continua en la siguiente página}}                                                              \\ \hline
      \endfoot

      \hline \hline
      \endlastfoot
      Desarrollo de la interfaz para el registro de usuario en la aplicación móvil & Usuario                                      \\
      \multicolumn{2}{|c|}{\textbf{Observaciones:} Ninguna}                                                                       \\
\end{longtable}

En la tabla \ref{tab:crc-12} se presenta la tarjeta CRC de la historia de usuario 12, la cual corresponde a la funcionalidad de
asignar miembros al grupo familiar.

\begin{longtable}{|p{5cm}|p{5cm}|}
      \caption{Tarjeta CRC - Historia 12: Asignar miembros al grupo familiar} \label{tab:crc-12}                        \\

      \hline \multicolumn{2}{|c|}{\textbf{Asignar miembros al grupo familiar}}                                          \\ \hline
      \hline \multicolumn{1}{|c|}{\textbf{Responsabilidades}}            & \multicolumn{1}{|c|}{\textbf{Colaboradores}} \\ \hline
      \endfirsthead

      \multicolumn{2}{c}%
      {{\normalfont \tablename\ \thetable{} -- continuación de la página anterior}}                                     \\
      \hline \multicolumn{1}{|c|}{\textbf{Responsabilidades}}            & \multicolumn{1}{|c|}{\textbf{Colaboradores}} \\ \hline
      \endhead

      \hline \multicolumn{2}{|r|}{{Continua en la siguiente página}}                                                    \\ \hline
      \endfoot

      \hline \hline
      \endlastfoot
      Diseño de la base de datos para asignar miembros al grupo familiar & Grupo familiar                               \\\hline
      Desarrollo backend para asignar miembros al grupo familiar         & Usuario, Roles                               \\\hline
      Desarrollo de la interfaz para asignar miembros al grupo familiar  &                                              \\\hline
      \multicolumn{2}{|c|}{\textbf{Observaciones:} Ninguna}                                                             \\
\end{longtable}

En la tabla \ref{tab:crc-13} se presenta la tarjeta CRC de la historia de usuario 13, la cual corresponde a la funcionalidad de
enviar alertas de emergencia.

\begin{longtable}{|p{5cm}|p{5cm}|}
      \caption{Tarjeta CRC - Historia 13: Enviar alertas de emergencia} \label{tab:crc-13}                        \\

      \hline \multicolumn{2}{|c|}{\textbf{Enviar alertas de emergencia}}                                          \\ \hline
      \hline \multicolumn{1}{|c|}{\textbf{Responsabilidades}}      & \multicolumn{1}{|c|}{\textbf{Colaboradores}} \\ \hline
      \endfirsthead

      \multicolumn{2}{c}%
      {{\normalfont \tablename\ \thetable{} -- continuación de la página anterior}}                               \\
      \hline \multicolumn{1}{|c|}{\textbf{Responsabilidades}}      & \multicolumn{1}{|c|}{\textbf{Colaboradores}} \\ \hline
      \endhead

      \hline \multicolumn{2}{|r|}{{Continua en la siguiente página}}                                              \\ \hline
      \endfoot

      \hline \hline
      \endlastfoot
      Diseño de la base de datos para enviar alertas de emergencia & Alarma                                       \\\hline
      Desarrollo backend para enviar alertas de emergencia         & Policía                                      \\\hline
      Desarrollo de la interfaz para enviar alertas de emergencia  & Usuario                                      \\\hline
      \multicolumn{2}{|c|}{\textbf{Observaciones:} Ninguna}                                                       \\
\end{longtable}

En la tabla \ref{tab:crc-14} se presenta la tarjeta CRC de la historia de usuario 14, la cual corresponde a la funcionalidad de
visualizar alertas de emergencia de familiares.

\begin{longtable}{|p{5cm}|p{5cm}|}
      \caption{Tarjeta CRC - Historia 14: Visualizar alertas de emergencia de familiares} \label{tab:crc-14}                        \\

      \hline \multicolumn{2}{|c|}{\textbf{Visualizar alertas de emergencia de familiares}}                                          \\ \hline
      \hline \multicolumn{1}{|c|}{\textbf{Responsabilidades}}                        & \multicolumn{1}{|c|}{\textbf{Colaboradores}} \\ \hline
      \endfirsthead

      \multicolumn{2}{c}%
      {{\normalfont \tablename\ \thetable{} -- continuación de la página anterior}}                                                 \\
      \hline \multicolumn{1}{|c|}{\textbf{Responsabilidades}}                        & \multicolumn{1}{|c|}{\textbf{Colaboradores}} \\ \hline
      \endhead

      \hline \multicolumn{2}{|r|}{{Continua en la siguiente página}}                                                                \\ \hline
      \endfoot

      \hline \hline
      \endlastfoot
      Diseño de la base de datos para visualizar alertas de emergencia de familiares & Grupo familiar                               \\\hline
      Desarrollo backend para visualizar alertas de emergencia de familiares         & Usuario                                      \\\hline
      Desarrollo de la interfaz para visualizar alertas de emergencia de familiares  &                                              \\\hline
      \multicolumn{2}{|c|}{\textbf{Observaciones:} Ninguna}                                                                         \\
\end{longtable}

En la tabla \ref{tab:crc-15} se presenta la tarjeta CRC de la historia de usuario 15, la cual corresponde a la funcionalidad de
visualizar alertas de emergencia de los ciudadanos.

\begin{longtable}{|p{5cm}|p{5cm}|}
      \caption{Tarjeta CRC - Historia 15: Visualizar alertas de emergencia de los ciudadanos} \label{tab:crc-15}                       \\

      \hline \multicolumn{2}{|c|}{\textbf{Visualizar alertas de emergencia de los ciudadanos}}                                         \\ \hline
      \hline \multicolumn{1}{|c|}{\textbf{Responsabilidades}}                           & \multicolumn{1}{|c|}{\textbf{Colaboradores}} \\ \hline
      \endfirsthead

      \multicolumn{2}{c}%
      {{\normalfont \tablename\ \thetable{} -- continuación de la página anterior}}                                                    \\
      \hline \multicolumn{1}{|c|}{\textbf{Responsabilidades}}                           & \multicolumn{1}{|c|}{\textbf{Colaboradores}} \\ \hline
      \endhead

      \hline \multicolumn{2}{|r|}{{Continua en la siguiente página}}                                                                   \\ \hline
      \endfoot

      \hline \hline
      \endlastfoot
      Desarrollo backend para visualizar alertas de emergencia de los ciudadanos        & Alarma                                       \\\hline
      Desarrollo de la interfaz para visualizar alertas de emergencia de los ciudadanos & Usuario                                      \\\hline
      \multicolumn{2}{|c|}{\textbf{Observaciones:} Ninguna}                                                                            \\
\end{longtable}

En la tabla \ref{tab:crc-16} se presenta la tarjeta CRC de la historia de usuario 16, la cual corresponde a la funcionalidad de
cambiar contraseña.

\begin{longtable}{|p{5cm}|p{5cm}|}
      \caption{Tarjeta CRC - Historia 16: Cambiar contraseña} \label{tab:crc-16}                             \\

      \hline \multicolumn{2}{|c|}{\textbf{Cambiar contraseña}}                                               \\ \hline
      \hline \multicolumn{1}{|c|}{\textbf{Responsabilidades}} & \multicolumn{1}{|c|}{\textbf{Colaboradores}} \\ \hline
      \endfirsthead

      \multicolumn{2}{c}%
      {{\normalfont \tablename\ \thetable{} -- continuación de la página anterior}}                          \\
      \hline \multicolumn{1}{|c|}{\textbf{Responsabilidades}} & \multicolumn{1}{|c|}{\textbf{Colaboradores}} \\ \hline
      \endhead

      \hline \multicolumn{2}{|r|}{{Continua en la siguiente página}}                                         \\ \hline
      \endfoot

      \hline \hline
      \endlastfoot
      Diseño de la base de datos para cambiar contraseña      & Usuario                                      \\\hline
      Desarrollo backend para cambiar contraseña              &                                              \\\hline
      Desarrollo de la interfaz para cambiar contraseña       &                                              \\\hline
      \multicolumn{2}{|c|}{\textbf{Observaciones:} Ninguna}                                                  \\
\end{longtable}

En la tabla \ref{tab:crc-17} se presenta la tarjeta CRC de la historia de usuario 17, la cual corresponde a la funcionalidad de
recuperar contraseña.

\begin{longtable}{|p{5cm}|p{5cm}|}
      \caption{Tarjeta CRC - Historia 17: Recuperar contraseña} \label{tab:crc-17}                           \\

      \hline \multicolumn{2}{|c|}{\textbf{Recuperar contraseña}}                                             \\ \hline
      \hline \multicolumn{1}{|c|}{\textbf{Responsabilidades}} & \multicolumn{1}{|c|}{\textbf{Colaboradores}} \\ \hline
      \endfirsthead

      \multicolumn{2}{c}%
      {{\normalfont \tablename\ \thetable{} -- continuación de la página anterior}}                          \\
      \hline \multicolumn{1}{|c|}{\textbf{Responsabilidades}} & \multicolumn{1}{|c|}{\textbf{Colaboradores}} \\ \hline
      \endhead

      \hline \multicolumn{2}{|r|}{{Continua en la siguiente página}}                                         \\ \hline
      \endfoot

      \hline \hline
      \endlastfoot
      Diseño de la base de datos para recuperar contraseña    & Usuario                                      \\\hline
      Desarrollo backend para recuperar contraseña            &                                              \\\hline
      Desarrollo de la interfaz para recuperar contraseña     &                                              \\\hline
      \multicolumn{2}{|c|}{\textbf{Observaciones:} Ninguna}                                                  \\
\end{longtable}

En la tabla \ref{tab:crc-18} se presenta la tarjeta CRC de la historia de usuario 18, la cual corresponde a la funcionalidad de
desarrollar el modelo analítico de BI.

\begin{longtable}{|p{5cm}|p{5cm}|}
      \caption{Tarjeta CRC - Historia 18: Desarrollar el modelo analítico de BI} \label{tab:crc-18}          \\

      \hline \multicolumn{2}{|c|}{\textbf{Desarrollar el modelo analítico de BI}}                            \\ \hline
      \hline \multicolumn{1}{|c|}{\textbf{Responsabilidades}} & \multicolumn{1}{|c|}{\textbf{Colaboradores}} \\ \hline
      \endfirsthead

      \multicolumn{2}{c}%
      {{\normalfont \tablename\ \thetable{} -- continuación de la página anterior}}                          \\
      \hline \multicolumn{1}{|c|}{\textbf{Responsabilidades}} & \multicolumn{1}{|c|}{\textbf{Colaboradores}} \\ \hline
      \endhead

      \hline \multicolumn{2}{|r|}{{Continua en la siguiente página}}                                         \\ \hline
      \endfoot

      \hline \hline
      \endlastfoot
      Seleccionar las preguntas del negocio                   & Usuario                                      \\\hline
      Identificar indicadores y perspectivas                  &                                              \\\hline
      Declarar el grano                                       &                                              \\\hline
      Identificar las dimensiones y sus atributos             &                                              \\\hline
      Crear esquema dimensional                               &                                              \\\hline
      Realizar el Proceso ETL                                 &                                              \\\hline
      Creación del Cubo OLAP                                  &                                              \\\hline
      Visualización de datos (Generar reportes)               &                                              \\\hline
      \multicolumn{2}{|c|}{\textbf{Observaciones:} Ninguna}                                                  \\
\end{longtable}