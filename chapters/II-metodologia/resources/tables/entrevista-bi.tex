\label{app:entrevista-bi}
\begin{longtable}{|l|p{5cm}|p{5cm}|}
    \caption{Entrevista aplicada al Ing. Rubén Nogales experto en Business Intelligence} \label{tab:entrevista-bi-tab}                                                                                                                                                                                                                                                                                                                                                                                                                                                                                                                                                                \\

    \hline \multicolumn{1}{|c|}{\textbf{N. Pregunta}} & \multicolumn{1}{|c|}{\textbf{Respuesta}}                                                                                                                                                                                                                                                                                                                                                              & \multicolumn{1}{|c|}{\textbf{Observación}}                                                                                                                                                                            \\ \hline
    \endfirsthead

    \multicolumn{3}{c}%
    {{\normalfont \tablename\ \thetable{} -- continuación de la página anterior}}                                                                                                                                                                                                                                                                                                                                                                                                                                                                                                                                                                                                     \\
    \hline \multicolumn{1}{|c|}{\textbf{N. Pregunta}} & \multicolumn{1}{|c|}{\textbf{Respuesta}}                                                                                                                                                                                                                                                                                                                                                              & \multicolumn{1}{|c|}{\textbf{Observación}}                                                                                                                                                                            \\ \hline
    \endhead

    \hline \multicolumn{3}{|r|}{{Continua en la siguiente página}}                                                                                                                                                                                                                                                                                                                                                                                                                                                                                                                                                                                                                    \\ \hline
    \endfoot


    \hline \multicolumn{3}{|p{13cm}|}{{
            \textbf{Conclusion:} La entrevista resalta la importancia crítica de una sólida estructura de datos y un enfoque meticuloso en la
            interpretación y presentación de la información en el ámbito de Business Intelligence (BI). Desde la definición de KPI relevantes hasta la selección
            adecuada de herramientas de análisis y visualización, cada aspecto destaca la necesidad de una estrategia integral que permita convertir los datos en
            acciones concretas. Tambien se señala que una efectiva estrategia de BI requiere no solo tecnología avanzada, sino también una comprensión
            profunda de las necesidades del usuario y una atención constante a la calidad y confiabilidad de los datos.
    }}                                                                                                                                                                                                                                                                                                                                                                                                                                                                                                                                                                                                                                                                                \\ \hline
    \endlastfoot

    1                                                 & Debe ser medible, debe tener la facilidad de ser medido, la variable del KPI debe cumplir con la función de agregación, es decir, en su momento dado pueda generar procesos aritméticos no solo en los hechos sino en las dimensiones.                                                                                                                                                                & La relevancia de un KPI radica en su capacidad de ser medible y agregable, lo que sugiere la importancia de que proporcione una visión general y permita comparaciones.                                               \\\hline
    2                                                 & Semanal es una buena frecuencia a menos que los datos que se generen tengan una gran retroalimentación.                                                                                                                                                                                                                                                                                               & La frecuencia semanal de actualización de informes de BI se sugiere como un equilibrio entre actualización oportuna y necesidad de datos actualizados para la toma de decisiones.                                     \\\hline
    3                                                 & Depende del caso, si los datos necesitan ser comparados, lo mejor son gráficos de barras, si son frecuencias, se deben visualizar histogramas, si en algún momento la información es en función del tiempo, deben ser por medio de gráficos de líneas, y si se desea mostrar porcentajes puede ser mediante un gráfico de pie.                                                                        & La adaptabilidad en el formato de presentación de datos destaca la necesidad de ajustarse al tipo de información para facilitar su comprensión y análisis.                                                            \\\hline
    4                                                 & Que los datos no provean la suficiente información, es decir, el dashboard generado no permita ver claramente lo que se desea para poder tomar decisiones, siendo el principal desafío la organización de los datos.                                                                                                                                                                                  & La organización de los datos se identifica como el principal desafío al interpretarlos, lo que destaca la importancia de una estructura clara y comprensible.                                                         \\\hline
    5                                                 & Que sea visual, dinámico, fácilmente entendible, con el cual se pueda realizar filtros y se pueda obtener fácilmente lo que se necesita.                                                                                                                                                                                                                                                              & La efectividad de un sistema de BI para la toma de decisiones se relaciona con su accesibilidad, visualización dinámica y comprensibilidad para los usuarios finales.                                                 \\\hline
    6                                                 & Herramientas para limpieza de datos, completación de datos como cluster, así como herramientas para leer y guardar datos estructurados y no estructurados. También herramientas para visualizar la información como Power BI, Tableau y JS para generar gráficos dinámicos.                                                                                                                           & Se enfatiza la importancia de herramientas de limpieza y visualización de datos, así como la capacidad para manejar datos estructurados y no estructurados.                                                           \\\hline
    7                                                 & Teniendo un buen modelo entidad-relación dimensional, también se debe contar con un modelado de casos de uso dimensional bien establecido.                                                                                                                                                                                                                                                            & Un modelo dimensional sólido y un modelado de casos de uso bien establecido se destacan como clave para abordar problemas de escalabilidad en BI.                                                                     \\\hline
    8                                                 & Se debe realizar un EDA, un Análisis Exploratorio de Datos completo tomando en cuenta que los datos que se obtengan sean de carácter independiente. También definir la distribución de los datos y de cada una de las variables, revisar que los datos sean completos, que no cuenten con ambigüedad, que no haya campos en blanco ni repetidos y también la anonimicidad de los datos es importante. & Un análisis exploratorio completo y la garantía de calidad de datos son esenciales para asegurar la confiabilidad de los informes de BI.                                                                              \\\hline
    9                                                 & Definir los KPI obtenidos de los modelados de caso de uso dimensional.                                                                                                                                                                                                                                                                                                                                & La identificación de requisitos de usuarios se sugiere a través de la definición de KPI basados en modelados de casos de uso dimensional, lo que alinea los informes con las necesidades específicas de los usuarios. \\
\end{longtable}
% \begin{longtable}{|l|p{5cm}|p{3cm}|p{3cm}|}
%     \caption{A sample long table.} \label{tab:long}                                                                                                                                                                                                                                                                                                                                                                                                                                                                                                      \\

%     \hline \multicolumn{1}{|c|}{\textbf{First column}} & \multicolumn{1}{c|}{\textbf{Second column}}                                                                                                                                                                                                                                                                                                                                                           & \multicolumn{1}{c|}{\textbf{Third column}} & \multicolumn{1}{c|}{\textbf{Third column}} \\ \hline
%     \endfirsthead

%     \multicolumn{4}{c}%
%     {{\bfseries \tablename\ \thetable{} -- continued from previous page}}                                                                                                                                                                                                                                                                                                                                                                                                                                                                                \\
%     \hline \multicolumn{1}{|c|}{\textbf{First column}} & \multicolumn{1}{c|}{\textbf{Second column}}                                                                                                                                                                                                                                                                                                                                                           & \multicolumn{1}{c|}{\textbf{Third column}} & \multicolumn{1}{c|}{\textbf{Third column}} \\ \hline
%     \endhead

%     \hline \multicolumn{4}{|r|}{{Continued on next page}}                                                                                                                                                                                                                                                                                                                                                                                                                                                                                                \\ \hline
%     \endfoot

%     \hline \hline
%     \endlastfoot

%     1                                                  & Debe ser medible, debe tener la facilidad de ser medido, la variable del KPI debe cumplir con la función de agregación, es decir, en su momento dado pueda generar procesos aritméticos no solo en los hechos sino en las dimensiones.                                                                                                                                                                & 123.456778                                 &                                            \\
%     2                                                  & Semanal es una buena frecuencia a menos que los datos que se generen tengan una gran retroalimentación.                                                                                                                                                                                                                                                                                               & 123.456778                                 &                                            \\
%     3                                                  & Depende del caso, si los datos necesitan ser comparados, lo mejor son gráficos de barras, si son frecuencias, se deben visualizar histogramas, si en algún momento la información es en función del tiempo, deben ser por medio de gráficos de líneas, y si se desea mostrar porcentajes puede ser mediante un gráfico de pie.                                                                        & 123.456778                                 &                                            \\
%     4                                                  & Que los datos no provean la suficiente información, es decir, el dashboard generado no permita ver claramente lo que se desea para poder tomar decisiones, siendo el principal desafío la organización de los datos.                                                                                                                                                                                  & 123.456778                                 &                                            \\
%     One                                                & Que sea visual, dinámico, fácilmente entendible, con el cual se pueda realizar filtros y se pueda obtener fácilmente lo que se necesita.                                                                                                                                                                                                                                                              & 123.456778                                 &                                            \\
%     One                                                & Herramientas para limpieza de datos, completación de datos como cluster, así como herramientas para leer y guardar datos estructurados y no estructurados. También herramientas para visualizar la información como Power BI, Tableau y JS para generar gráficos dinámicos.                                                                                                                           & 123.456778                                 &                                            \\
%     One                                                & Teniendo un buen modelo entidad-relación dimensional, también se debe contar con un modelado de casos de uso dimensional bien establecido.                                                                                                                                                                                                                                                            & 123.456778                                 &                                            \\
%     One                                                & Se debe realizar un EDA, un Análisis Exploratorio de Datos completo tomando en cuenta que los datos que se obtengan sean de carácter independiente. También definir la distribución de los datos y de cada una de las variables, revisar que los datos sean completos, que no cuenten con ambigüedad, que no haya campos en blanco ni repetidos y también la anonimicidad de los datos es importante. & 123.456778                                 &                                            \\
%     One                                                & Definir los KPI obtenidos de los modelados de caso de uso dimensional.                                                                                                                                                                                                                                                                                                                                & 123.456778                                 &                                            \\
%     One                                                & abcdef ghjijklmn                                                                                                                                                                                                                                                                                                                                                                                      & 123.456778                                 &                                            \\
%     One                                                & abcdef ghjijklmn                                                                                                                                                                                                                                                                                                                                                                                      & 123.456778                                 &                                            \\
%     One                                                & abcdef ghjijklmn                                                                                                                                                                                                                                                                                                                                                                                      & 123.456778                                 &                                            \\
%     One                                                & abcdef ghjijklmn                                                                                                                                                                                                                                                                                                                                                                                      & 123.456778                                 &                                            \\
% \end{longtable}