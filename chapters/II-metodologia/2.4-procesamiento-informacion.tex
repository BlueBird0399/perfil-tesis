\subsection{Procesamiento y análisis de datos}

Este apartado se centrará en la conclusión de la entrevista y en el procesamiento de
los datos obtenidos del protocolo técnico. Estos datos serán fundamentales para desarrollar
un modelo analítico de Business Intelligence (BI) dirigido al seguimiento de las tendencias de
siniestros en la ciudad de Ambato.
\bigbreak

En base a la información recopilada de las entrevistas, se puede concluir que:

\begin{itemize}
    \item La relevancia de los KPIs es crucial para la toma de decisiones informadas en BI. Es fundamental que los KPIs sean medibles y fácilmente agregables, permitiendo análisis aritméticos para proporcionar una visión clara del rendimiento.
    \item La actualización regular de informes es esencial. Se sugiere una frecuencia semanal, aunque puede variar dependiendo de la retroalimentación necesaria de los datos.
    \item La elección del formato de presentación de datos juega un papel importante en la comprensión y análisis de la información. Los gráficos de barras son preferibles para comparaciones, mientras que los gráficos de líneas son ideales para datos temporales.
    \item Se requiere un nivel alto y constante de capacitación, especialmente para los miembros más recientes, debido a la necesidad de adaptarse a nuevas tecnologías.
    \item La interpretación efectiva de datos en BI enfrenta desafíos, especialmente en la organización de los datos para una visualización clara y la identificación de patrones significativos.
    \item Un sistema de BI efectivo debe ser visual, dinámico y fácilmente entendible, permitiendo filtros para obtener la información necesaria de manera rápida y precisa.
    \item Es esencial contar con medidas de seguridad adecuadas para proteger la información sensible relacionada con incidentes delictivos.
    \item La recopilación continua de datos se realiza en una matriz de Excel, destacando la necesidad de modernizar los métodos de recopilación y análisis de datos.
    \item Los sistemas de análisis de incidentes delictivos deben proporcionar información completa sobre eventos delictivos y detalles de agresores para facilitar la identificación de sospechosos potenciales.
    \item Se requiere un nivel alto y constante de capacitación para utilizar eficazmente sistemas de análisis de incidentes delictivos, especialmente debido a la evolución de las tecnologías.
\end{itemize}

En relación con el protocolo técnico, los datos recolectados a través del formulario web serán procesados y almacenados en una base de datos especifica para ello.
En los cuales se aplicara un proceso ETL asi como se emplearán algoritmos de aumento de datos, con la finalidad de obtener una cantidad considerable
de elementos con los que trabajar. Esta información será fundamental para la creación de un
cubo de Procesamiento Analítico en Línea, del inglés (OLAP), que posibilitará el análisis
exhaustivo de dichos datos y, a su vez, permitirá la generación de informes detallados.
