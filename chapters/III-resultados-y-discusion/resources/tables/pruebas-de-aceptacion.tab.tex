\begin{longtable}{|p{6.7cm}|p{6.7cm}|}
    \caption{Prueba de aceptación 1: Iniciar sesión sistema web} \label{tab:prueba-1}
    \\
    \hline
    \multicolumn{2}{|c|}{\textbf{Prueba de aceptación}}                                                                                                                                                      \\
    \hline

    \endfirsthead

    \hline
    \endhead

    \hline
    \multicolumn{2}{|c|}{{Continua en la siguiente página}}                                                                                                                                                  \\
    \hline
    \endfoot

    \hline
    \endlastfoot
    \multicolumn{1}{|p{6.7cm}|}{\textbf{Número} 1 } & \multicolumn{1}{|p{6.7cm}|}{\textbf{Historia de usuario:} 1}                                                                                           \\
    \hline
    \multicolumn{2}{|p{13.4cm}|}{\textbf{Nombre de la historia:} Iniciar sesión sistema web }                                                                                                                \\
    \hline
    \multicolumn{2}{|p{13.4cm}|}{\textbf{Descripción:} Como usuario administrador del sistema web, quiero iniciar sesión en el sistema para acceder a sus funcionalidades}                                   \\
    \hline
    \multicolumn{2}{|p{13.4cm}|}{\textbf{Condiciones de ejecución:} Ser un usuario administrador}                                                                                                            \\
    \hline
    \multicolumn{2}{|p{13.4cm}|}{\textbf{Entrada/pasos de ejecución:} El usuario debe ingresar sus credenciales de acceso}                                                                                   \\
    \hline
    \multicolumn{2}{|p{13.4cm}|}{\textbf{Resultado esperado:} Dado que el usuario ingrese su correo y contraseña y seleccione ingresar, entonces el sistema web le permitirá ingresar a sus funcionalidades} \\
    \hline
    \multicolumn{2}{|p{13.4cm}|}{\textbf{Evaluación de prueba:} Satisfactoria}                                                                                                                               \\
    \hline
\end{longtable}


\begin{longtable}{|p{6.7cm}|p{6.7cm}|}
    \caption{Prueba de aceptación 2: Cerrar sesión sistema web} \label{tab:prueba-2}
    \\
    \hline
    \multicolumn{2}{|c|}{\textbf{Prueba de aceptación}}                                                                                                                                \\
    \hline

    \endfirsthead

    \hline
    \endhead

    \hline
    \multicolumn{2}{|c|}{{Continua en la siguiente página}}                                                                                                                            \\
    \hline
    \endfoot

    \hline
    \endlastfoot
    \multicolumn{1}{|p{6.7cm}|}{\textbf{Número} 2 } & \multicolumn{1}{|p{6.7cm}|}{\textbf{Historia de usuario:} 2}                                                                     \\
    \hline
    \multicolumn{2}{|p{13.4cm}|}{\textbf{Nombre de la historia:} Cerrar sesión sistema web }                                                                                           \\
    \hline
    \multicolumn{2}{|p{13.4cm}|}{\textbf{Descripción:}  Como usuario del sistema web, quiero cerrar sesión en el sistema para finalizar mi sesión de trabajo}                          \\
    \hline
    \multicolumn{2}{|p{13.4cm}|}{\textbf{Condiciones de ejecución:} Ser un usuario administrador y acceder al sistema}                                                                 \\
    \hline
    \multicolumn{2}{|p{13.4cm}|}{\textbf{Entrada/pasos de ejecución:} El usuario debe seleccionar la opción "cerrar sesión" en el menú de opciones}                                    \\
    \hline
    \multicolumn{2}{|p{13.4cm}|}{\textbf{Resultado esperado:} Dado que el usuario seleccione la opción "cerrar sesión", entonces el sistema debe redirigirlo a la pantalla de ingreso} \\
    \hline
    \multicolumn{2}{|p{13.4cm}|}{\textbf{Evaluación de prueba:} Satisfactoria}                                                                                                         \\
    \hline
\end{longtable}


\begin{longtable}{|p{6.7cm}|p{6.7cm}|}
    \caption{Prueba de aceptación 3: Gestionar usuarios} \label{tab:prueba-3}
    \\
    \hline
    \multicolumn{2}{|c|}{\textbf{Prueba de aceptación}}                                                                                                              \\
    \hline

    \endfirsthead

    \hline
    \endhead

    \hline
    \multicolumn{2}{|c|}{{Continua en la siguiente página}}                                                                                                          \\
    \hline
    \endfoot

    \hline
    \endlastfoot
    \multicolumn{1}{|p{6.7cm}|}{\textbf{Número} 3 } & \multicolumn{1}{|p{6.7cm}|}{\textbf{Historia de usuario:} 3}                                                   \\
    \hline
    \multicolumn{2}{|p{13.4cm}|}{\textbf{Nombre de la historia:} Gestionar usuarios }                                                                                \\
    \hline
    \multicolumn{2}{|p{13.4cm}|}{\textbf{Descripción:} Como usuario administrador, quiero administrar los usuarios que se encuentran registrados en el sistema}      \\
    \hline
    \multicolumn{2}{|p{13.4cm}|}{\textbf{Condiciones de ejecución:} Ser un usuario administrador}                                                                    \\
    \hline
    \multicolumn{2}{|p{13.4cm}|}{\textbf{Entrada/pasos de ejecución:} El usuario administrador debe haber accedido al sistema e ingresar en la pantalla de usuarios} \\
    \hline
    \multicolumn{2}{|p{13.4cm}|}{\textbf{Resultado esperado:} Dado que el usuario administrador ingrese a la pantalla de usuarios, cuando quiera ingresar,
    modificar o deshabilitar un usuario, entonces el sistema permitirá la actualización de la información en la base de datos}                                       \\
    \hline
    \multicolumn{2}{|p{13.4cm}|}{\textbf{Evaluación de prueba:} Satisfactoria}                                                                                       \\
    \hline
\end{longtable}


\begin{longtable}{|p{6.7cm}|p{6.7cm}|}
    \caption{Prueba de aceptación 4: Gestionar tipos de incidentes} \label{tab:prueba-4}
    \\
    \hline
    \multicolumn{2}{|c|}{\textbf{Prueba de aceptación}}                                                                                                                         \\
    \hline

    \endfirsthead

    \hline
    \endhead

    \hline
    \multicolumn{2}{|c|}{{Continua en la siguiente página}}                                                                                                                     \\
    \hline
    \endfoot

    \hline
    \endlastfoot
    \multicolumn{1}{|p{6.7cm}|}{\textbf{Número} 4 } & \multicolumn{1}{|p{6.7cm}|}{\textbf{Historia de usuario:} 4}                                                              \\
    \hline
    \multicolumn{2}{|p{13.4cm}|}{\textbf{Nombre de la historia:} Gestionar tipos de incidentes }                                                                                \\
    \hline
    \multicolumn{2}{|p{13.4cm}|}{\textbf{Descripción:} Como usuario administrador, quiero gestionar los tipos de incidentes para definir las categorías de los incidentes}      \\
    \hline
    \multicolumn{2}{|p{13.4cm}|}{\textbf{Condiciones de ejecución:} Ser un usuario administrador}                                                                               \\
    \hline
    \multicolumn{2}{|p{13.4cm}|}{\textbf{Entrada/pasos de ejecución:} El usuario administrador debe haber accedido al sistema e ingresar en la pantalla de tipos de incidentes} \\
    \hline
    \multicolumn{2}{|p{13.4cm}|}{\textbf{Resultado esperado:} Dado que el usuario administrador ingrese a la pantalla de tipos de incidentes, cuando quiera visualizar,
    crear, actualizar, deshabilitar o habilitar un tipo de incidente, entonces el sistema permitirá la actualización de la información en la base de datos}                     \\
    \hline
    \multicolumn{2}{|p{13.4cm}|}{\textbf{Evaluación de prueba:} Satisfactoria}                                                                                                  \\
    \hline
\end{longtable}


\begin{longtable}{|p{6.7cm}|p{6.7cm}|}
    \caption{Prueba de aceptación 5: Gestionar zonas de vigilancia} \label{tab:prueba-5}
    \\
    \hline
    \multicolumn{2}{|c|}{\textbf{Prueba de aceptación}}                                                                                                                         \\
    \hline

    \endfirsthead

    \hline
    \endhead

    \hline
    \multicolumn{2}{|c|}{{Continua en la siguiente página}}                                                                                                                     \\
    \hline
    \endfoot

    \hline
    \endlastfoot
    \multicolumn{1}{|p{6.7cm}|}{\textbf{Número} 5 } & \multicolumn{1}{|p{6.7cm}|}{\textbf{Historia de usuario:} 5}                                                              \\
    \hline
    \multicolumn{2}{|p{13.4cm}|}{\textbf{Nombre de la historia:} Gestionar zonas de vigilancia }                                                                                \\
    \hline
    \multicolumn{2}{|p{13.4cm}|}{\textbf{Descripción:} Como usuario administrador, quiero gestionar las zonas de vigilancia para definir las áreas de monitoreo}                \\
    \hline
    \multicolumn{2}{|p{13.4cm}|}{\textbf{Condiciones de ejecución:} Ser un usuario administrador}                                                                               \\
    \hline
    \multicolumn{2}{|p{13.4cm}|}{\textbf{Entrada/pasos de ejecución:} El usuario administrador debe haber accedido al sistema e ingresar en la pantalla de zonas de vigilancia} \\
    \hline
    \multicolumn{2}{|p{13.4cm}|}{\textbf{Resultado esperado:} Dado que el usuario administrador ingrese a la pantalla de zonas de vigilancia, cuando quiera visualizar, crear,
    actualizar, deshabilitar o habilitar una zona de vigilancia, entonces el sistema permitirá la actualización de la información en la base de datos}                          \\
    \hline
    \multicolumn{2}{|p{13.4cm}|}{\textbf{Evaluación de prueba:} Satisfactoria}                                                                                                  \\
    \hline
\end{longtable}


\begin{longtable}{|p{6.7cm}|p{6.7cm}|}
    \caption{Prueba de aceptación 6: Asignar policías a las zonas de vigilancia} \label{tab:prueba-6}
    \\
    \hline
    \multicolumn{2}{|c|}{\textbf{Prueba de aceptación}}                                                                                                                                                                                                                                                                                                        \\
    \hline

    \endfirsthead

    \hline
    \endhead

    \hline
    \multicolumn{2}{|c|}{{Continua en la siguiente página}}                                                                                                                                                                                                                                                                                                    \\
    \hline
    \endfoot

    \hline
    \endlastfoot
    \multicolumn{1}{|p{6.7cm}|}{\textbf{Número} 6 } & \multicolumn{1}{|p{6.7cm}|}{\textbf{Historia de usuario:} 6}                                                                                                                                                                                                                                             \\
    \hline
    \multicolumn{2}{|p{13.4cm}|}{\textbf{Nombre de la historia:} Asignar policías a las zonas de vigilancia }                                                                                                                                                                                                                                                  \\
    \hline
    \multicolumn{2}{|p{13.4cm}|}{\textbf{Descripción:} Como usuario administrador, quiero asignar y desasignar miembros de la policía a las zonas de vigilancia}                                                                                                                                                                                               \\
    \hline
    \multicolumn{2}{|p{13.4cm}|}{\textbf{Condiciones de ejecución:} Ser un usuario administrador}                                                                                                                                                                                                                                                              \\
    \hline
    \multicolumn{2}{|p{13.4cm}|}{\textbf{Entrada/pasos de ejecución:} El usuario administrador debe haber accedido al sistema e ingresar en la pantalla de asignación de policías a zonas de vigilancia}                                                                                                                                                       \\
    \hline
    \multicolumn{2}{|p{13.4cm}|}{\textbf{Resultado esperado:} Dado que el usuario administrador ingrese a la pantalla de asignación de policías a zonas de vigilancia, cuando quiera visualizar, asignar o desasignar un miembro de la policía a una zona de vigilancia, entonces el sistema permitirá la actualización de la información en la base de datos} \\
    \hline
    \multicolumn{2}{|p{13.4cm}|}{\textbf{Evaluación de prueba:} Satisfactoria}                                                                                                                                                                                                                                                                                 \\
    \hline
\end{longtable}


\begin{longtable}{|p{6.7cm}|p{6.7cm}|}
    \caption{Prueba de aceptación 7: Gestionar alertas de incidentes} \label{tab:prueba-7}
    \\
    \hline
    \multicolumn{2}{|c|}{\textbf{Prueba de aceptación}}                                                                                                                                                                                                                                                                            \\
    \hline

    \endfirsthead

    \hline
    \endhead

    \hline
    \multicolumn{2}{|c|}{{Continua en la siguiente página}}                                                                                                                                                                                                                                                                        \\
    \hline
    \endfoot

    \hline
    \endlastfoot
    \multicolumn{1}{|p{6.7cm}|}{\textbf{Número} 7 } & \multicolumn{1}{|p{6.7cm}|}{\textbf{Historia de usuario:} 7}                                                                                                                                                                                                                 \\
    \hline
    \multicolumn{2}{|p{13.4cm}|}{\textbf{Nombre de la historia:} Gestionar alertas de incidentes }                                                                                                                                                                                                                                 \\
    \hline
    \multicolumn{2}{|p{13.4cm}|}{\textbf{Descripción:} Como usuario administrador, quiero visualizar mediante un mapa las alertas de emergencia enviadas por los ciudadanos así como su posición en tiempo real}                                                                                                                   \\
    \hline
    \multicolumn{2}{|p{13.4cm}|}{\textbf{Condiciones de ejecución:} Ser un usuario administrador}                                                                                                                                                                                                                                  \\
    \hline
    \multicolumn{2}{|p{13.4cm}|}{\textbf{Entrada/pasos de ejecución:} El usuario administrador debe haber accedido al sistema e ingresar en la pantalla de visualización de alertas de incidentes}                                                                                                                                 \\
    \hline
    \multicolumn{2}{|p{13.4cm}|}{\textbf{Resultado esperado:} Dado que el usuario administrador ingrese a la pantalla de visualización de alertas de incidentes, cuando quiera visualizar las alertas de emergencia en un mapa junto con su posición en tiempo real, entonces el sistema mostrará dicha información correctamente} \\
    \hline
    \multicolumn{2}{|p{13.4cm}|}{\textbf{Evaluación de prueba:} Satisfactoria}                                                                                                                                                                                                                                                     \\
    \hline
\end{longtable}


\begin{longtable}{|p{6.7cm}|p{6.7cm}|}
    \caption{Prueba de aceptación 8: Visualizar mapa de calor} \label{tab:prueba-8}
    \\
    \hline
    \multicolumn{2}{|c|}{\textbf{Prueba de aceptación}}                                                                                                                                                                                                                                                                             \\
    \hline

    \endfirsthead

    \hline
    \endhead

    \hline
    \multicolumn{2}{|c|}{{Continua en la siguiente página}}                                                                                                                                                                                                                                                                         \\
    \hline
    \endfoot

    \hline
    \endlastfoot
    \multicolumn{1}{|p{6.7cm}|}{\textbf{Número} 8 } & \multicolumn{1}{|p{6.7cm}|}{\textbf{Historia de usuario:} 8}                                                                                                                                                                                                                  \\
    \hline
    \multicolumn{2}{|p{13.4cm}|}{\textbf{Nombre de la historia:} Visualizar mapa de calor }                                                                                                                                                                                                                                         \\
    \hline
    \multicolumn{2}{|p{13.4cm}|}{\textbf{Descripción:} Como usuario administrador, quiero visualizar mediante un mapa de calor los incidentes delictivos suscitados en las diferentes zonas}                                                                                                                                        \\
    \hline
    \multicolumn{2}{|p{13.4cm}|}{\textbf{Condiciones de ejecución:} Ser un usuario administrador}                                                                                                                                                                                                                                   \\
    \hline
    \multicolumn{2}{|p{13.4cm}|}{\textbf{Entrada/pasos de ejecución:} El usuario administrador debe haber accedido al sistema e ingresar en la pantalla de visualización del mapa de calor de incidentes delictivos}                                                                                                                \\
    \hline
    \multicolumn{2}{|p{13.4cm}|}{\textbf{Resultado esperado:} Dado que el usuario administrador ingrese a la pantalla de visualización del mapa de calor, cuando quiera visualizar los incidentes delictivos en un mapa mediante un mapa de calor, entonces el sistema mostrará dicha información de manera clara y representativa} \\
    \hline
    \multicolumn{2}{|p{13.4cm}|}{\textbf{Evaluación de prueba:} Satisfactoria}                                                                                                                                                                                                                                                      \\
    \hline
\end{longtable}


\begin{longtable}{|p{6.7cm}|p{6.7cm}|}
    \caption{Prueba de aceptación 9: Iniciar sesión aplicación móvil} \label{tab:prueba-9}
    \\
    \hline
    \multicolumn{2}{|c|}{\textbf{Prueba de aceptación}}                                                                                                                                                                                                                                                                                               \\
    \hline

    \endfirsthead

    \hline
    \endhead

    \hline
    \multicolumn{2}{|c|}{{Continua en la siguiente página}}                                                                                                                                                                                                                                                                                           \\
    \hline
    \endfoot

    \hline
    \endlastfoot
    \multicolumn{1}{|p{6.7cm}|}{\textbf{Número} 9 } & \multicolumn{1}{|p{6.7cm}|}{\textbf{Historia de usuario:} 9}                                                                                                                                                                                                                                    \\
    \hline
    \multicolumn{2}{|p{13.4cm}|}{\textbf{Nombre de la historia:} Iniciar sesión aplicación móvil }                                                                                                                                                                                                                                                    \\
    \hline
    \multicolumn{2}{|p{13.4cm}|}{\textbf{Descripción:} Como ciudadano, quiero iniciar sesión en el sistema móvil para acceder a sus funcionalidades}                                                                                                                                                                                                  \\
    \hline
    \multicolumn{2}{|p{13.4cm}|}{\textbf{Condiciones de ejecución:} Ser un ciudadano con cuenta registrada en el sistema móvil}                                                                                                                                                                                                                       \\
    \hline
    \multicolumn{2}{|p{13.4cm}|}{\textbf{Entrada/pasos de ejecución:} El ciudadano ingresa su correo electrónico y contraseña en la pantalla de inicio de sesión del sistema móvil}                                                                                                                                                                   \\
    \hline
    \multicolumn{2}{|p{13.4cm}|}{\textbf{Resultado esperado:} Dado que el ciudadano ingrese correctamente su correo y contraseña y presione el botón de inicio de sesión, la aplicación lo redirigirá a la pantalla principal del sistema móvil. En caso de ingresar incorrectamente los datos, la aplicación mostrará un mensaje de error adecuado.} \\
    \hline
    \multicolumn{2}{|p{13.4cm}|}{\textbf{Evaluación de prueba:} Satisfactoria}                                                                                                                                                                                                                                                                        \\
    \hline
\end{longtable}


\begin{longtable}{|p{6.7cm}|p{6.7cm}|}
    \caption{Prueba de aceptación 10: Cerrar sesión aplicación móvil} \label{tab:prueba-10}
    \\
    \hline
    \multicolumn{2}{|c|}{\textbf{Prueba de aceptación}}                                                                                                                                                                             \\
    \hline

    \endfirsthead

    \hline
    \endhead

    \hline
    \multicolumn{2}{|c|}{{Continua en la siguiente página}}                                                                                                                                                                         \\
    \hline
    \endfoot

    \hline
    \endlastfoot
    \multicolumn{1}{|p{6.7cm}|}{\textbf{Número} 10 } & \multicolumn{1}{|p{6.7cm}|}{\textbf{Historia de usuario:} 10}                                                                                                                \\
    \hline
    \multicolumn{2}{|p{13.4cm}|}{\textbf{Nombre de la historia:} Cerrar sesión aplicación móvil }                                                                                                                                   \\
    \hline
    \multicolumn{2}{|p{13.4cm}|}{\textbf{Descripción:} Como ciudadano, quiero cerrar sesión en la aplicación móvil para finalizar mi sesión de trabajo}                                                                             \\
    \hline
    \multicolumn{2}{|p{13.4cm}|}{\textbf{Condiciones de ejecución:} Ser un ciudadano con sesión iniciada en la aplicación móvil}                                                                                                    \\
    \hline
    \multicolumn{2}{|p{13.4cm}|}{\textbf{Entrada/pasos de ejecución:} El ciudadano presiona el botón de cerrar sesión desde cualquier pantalla de la aplicación móvil}                                                              \\
    \hline
    \multicolumn{2}{|p{13.4cm}|}{\textbf{Resultado esperado:} Dado que el ciudadano presione el botón de cerrar sesión, la aplicación lo redirigirá a la pantalla de inicio de sesión, finalizando correctamente su sesión activa.} \\
    \hline
    \multicolumn{2}{|p{13.4cm}|}{\textbf{Evaluación de prueba:} Satisfactoria}                                                                                                                                                      \\
    \hline
\end{longtable}


\begin{longtable}{|p{6.7cm}|p{6.7cm}|}
    \caption{Prueba de aceptación 11: Registro de usuario} \label{tab:prueba-11}
    \\
    \hline
    \multicolumn{2}{|c|}{\textbf{Prueba de aceptación}}                                                                                                                                                                                                                                                             \\
    \hline

    \endfirsthead

    \hline
    \endhead

    \hline
    \multicolumn{2}{|c|}{{Continua en la siguiente página}}                                                                                                                                                                                                                                                         \\
    \hline
    \endfoot

    \hline
    \endlastfoot
    \multicolumn{1}{|p{6.7cm}|}{\textbf{Número} 11 } & \multicolumn{1}{|p{6.7cm}|}{\textbf{Historia de usuario:} 11}                                                                                                                                                                                                \\
    \hline
    \multicolumn{2}{|p{13.4cm}|}{\textbf{Nombre de la historia:} Registro de usuario }                                                                                                                                                                                                                              \\
    \hline
    \multicolumn{2}{|p{13.4cm}|}{\textbf{Descripción:} Como ciudadano, quiero ingresar mis datos y fotografía mediante un formulario}                                                                                                                                                                               \\
    \hline
    \multicolumn{2}{|p{13.4cm}|}{\textbf{Condiciones de ejecución:} Ser un ciudadano que desea registrarse en el sistema}                                                                                                                                                                                           \\
    \hline
    \multicolumn{2}{|p{13.4cm}|}{\textbf{Entrada/pasos de ejecución:}
    \begin{enumerate}[label=\arabic*.]
        \item El ciudadano accede a la pantalla de registro en la aplicación móvil.
        \item Ingresa sus datos personales (nombre, correo electrónico, etc.) y selecciona una fotografía.
        \item Completa y envía el formulario de registro.
    \end{enumerate}
    }                                                                                                                                                                                                                                                                                                               \\
    \hline
    \multicolumn{2}{|p{13.4cm}|}{\textbf{Resultado esperado:} Dado que el ciudadano complete y envíe el formulario de registro con sus datos personales y fotografía, el sistema deberá almacenar correctamente la información en la base de datos, permitiendo así el registro exitoso del usuario en el sistema.} \\
    \hline
    \multicolumn{2}{|p{13.4cm}|}{\textbf{Evaluación de prueba:} Satisfactoria}                                                                                                                                                                                                                                      \\
    \hline
\end{longtable}


\begin{longtable}{|p{6.7cm}|p{6.7cm}|}
    \caption{Prueba de aceptación 12: Asignar miembros al grupo familiar} \label{tab:prueba-12}
    \\
    \hline
    \multicolumn{2}{|c|}{\textbf{Prueba de aceptación}}                                                                                                                                                                                                                                                           \\
    \hline

    \endfirsthead

    \hline
    \endhead

    \hline
    \multicolumn{2}{|c|}{{Continua en la siguiente página}}                                                                                                                                                                                                                                                       \\
    \hline
    \endfoot

    \hline
    \endlastfoot
    \multicolumn{1}{|p{6.7cm}|}{\textbf{Número} 12 } & \multicolumn{1}{|p{6.7cm}|}{\textbf{Historia de usuario:} 12}                                                                                                                                                                                              \\
    \hline
    \multicolumn{2}{|p{13.4cm}|}{\textbf{Nombre de la historia:} Asignar miembros al grupo familiar }                                                                                                                                                                                                             \\
    \hline
    \multicolumn{2}{|p{13.4cm}|}{\textbf{Descripción:} Como ciudadano, quiero asignar miembros al grupo familiar para recibir alertas de emergencia}                                                                                                                                                              \\
    \hline
    \multicolumn{2}{|p{13.4cm}|}{\textbf{Condiciones de ejecución:} Ser un ciudadano registrado en el sistema y tener acceso a la funcionalidad de gestión de grupo familiar}                                                                                                                                     \\
    \hline
    \multicolumn{2}{|p{13.4cm}|}{\textbf{Entrada/pasos de ejecución:}
    \begin{enumerate}[label=\arabic*.]
        \item El ciudadano accede a la pantalla de gestión de grupo familiar en la aplicación móvil.
        \item Selecciona la opción para añadir un nuevo miembro al grupo.
        \item Ingresa los datos del nuevo miembro (nombre, correo electrónico, etc.).
        \item Confirma la asignación del nuevo miembro.
    \end{enumerate}
    }                                                                                                                                                                                                                                                                                                             \\
    \hline
    \multicolumn{2}{|p{13.4cm}|}{\textbf{Resultado esperado:} Dado que el ciudadano complete los pasos para añadir un nuevo miembro al grupo familiar, el sistema deberá registrar correctamente al nuevo miembro como parte del grupo, permitiendo así que reciba alertas de emergencia junto con el ciudadano.} \\
    \hline
    \multicolumn{2}{|p{13.4cm}|}{\textbf{Evaluación de prueba:} Satisfactoria}                                                                                                                                                                                                                                    \\
    \hline
\end{longtable}


\begin{longtable}{|p{6.7cm}|p{6.7cm}|}
    \caption{Prueba de aceptación 13: Enviar alerta de incidente} \label{tab:prueba-13}
    \\
    \hline
    \multicolumn{2}{|c|}{\textbf{Prueba de aceptación}}                                                                                                                                                                                                    \\
    \hline

    \endfirsthead

    \hline
    \endhead

    \hline
    \multicolumn{2}{|c|}{{Continua en la siguiente página}}                                                                                                                                                                                                \\
    \hline
    \endfoot

    \hline
    \endlastfoot
    \multicolumn{1}{|p{6.7cm}|}{\textbf{Número} 13 } & \multicolumn{1}{|p{6.7cm}|}{\textbf{Historia de usuario:} 13}                                                                                                                                       \\
    \hline
    \multicolumn{2}{|p{13.4cm}|}{\textbf{Nombre de la historia:} Enviar alerta de incidente }                                                                                                                                                              \\
    \hline
    \multicolumn{2}{|p{13.4cm}|}{\textbf{Descripción:} Como ciudadano, quiero enviar una alerta de incidente para notificar a las autoridades sobre un incidente delictivo}                                                                                \\
    \hline
    \multicolumn{2}{|p{13.4cm}|}{\textbf{Condiciones de ejecución:} Ser un ciudadano registrado en el sistema y tener acceso a la funcionalidad de enviar alertas de incidentes.}                                                                          \\
    \hline
    \multicolumn{2}{|p{13.4cm}|}{\textbf{Entrada/pasos de ejecución:}
    \begin{enumerate}[label=\arabic*.]
        \item El ciudadano abre la aplicación móvil y accede a la pantalla principal.
        \item Selecciona el tipo de incidente (robo, agresión, etc.) y proporciona una descripción breve.
        \item Presiona el botón de alerta durante 3 para confirmar el envió de la alerta.
    \end{enumerate}
    }                                                                                                                                                                                                                                                      \\
    \hline
    \multicolumn{2}{|p{13.4cm}|}{\textbf{Resultado esperado:} Dado que el ciudadano complete los pasos para enviar una alerta de incidente, el sistema deberá registrar la alerta y notificar a los entes de control y a los miembros del grupo familiar.} \\
    \hline
    \multicolumn{2}{|p{13.4cm}|}{\textbf{Evaluación de prueba:} Satisfactoria}                                                                                                                                                                             \\
    \hline
\end{longtable}

\begin{longtable}{|p{6.7cm}|p{6.7cm}|}
    \caption{Prueba de aceptación 14: Visualizar alertas de emergencia de familiares} \label{tab:prueba-14}                                                                                                                                                                                    \\
    \hline
    \multicolumn{2}{|c|}{\textbf{Prueba de aceptación}}                                                                                                                                                                                                                                        \\
    \hline
    \endfirsthead
    \hline
    \endhead
    \hline
    \multicolumn{2}{|c|}{{Continúa en la siguiente página}}                                                                                                                                                                                                                                    \\
    \hline
    \endfoot
    \hline
    \endlastfoot
    \multicolumn{1}{|p{6.7cm}|}{\textbf{Número} 14} & \multicolumn{1}{|p{6.7cm}|}{\textbf{Historia de usuario:} 14}                                                                                                                                                                            \\
    \hline
    \multicolumn{2}{|p{13.4cm}|}{\textbf{Nombre de la historia:} Visualizar alertas de emergencia de familiares}                                                                                                                                                                               \\
    \hline
    \multicolumn{2}{|p{13.4cm}|}{\textbf{Descripción:} Como ciudadano, quiero visualizar mediante un mapa las alertas de emergencia enviadas por mis familiares así como su posición en tiempo real}                                                                                           \\
    \hline
    \multicolumn{2}{|p{13.4cm}|}{\textbf{Condiciones de ejecución:} Ser un ciudadano con miembros de grupo familiar registrados}                                                                                                                                                               \\
    \hline
    \multicolumn{2}{|p{13.4cm}|}{\textbf{Entrada/pasos de ejecución:} El ciudadano debe iniciar sesión en el sistema y acceder a la sección de alertas recibidas.}                                                                                                                             \\
    \hline
    \multicolumn{2}{|p{13.4cm}|}{\textbf{Resultado esperado:} Dado que el ciudadano ingrese al apartado de alertas recibidas, el sistema deberá mostrar las alertas de emergencia enviadas por los miembros de su grupo familiar, así como la ubicación en tiempo real en el mapa de alertas.} \\
    \hline
    \multicolumn{2}{|p{13.4cm}|}{\textbf{Evaluación de prueba:} Satisfactoria}                                                                                                                                                                                                                 \\
    \hline
\end{longtable}


\begin{longtable}{|p{6.7cm}|p{6.7cm}|}
    \caption{Prueba de aceptación 15: Visualizar alertas de emergencia de ciudadanos} \label{tab:prueba-15}                                                                                                                                              \\
    \hline
    \multicolumn{2}{|c|}{\textbf{Prueba de aceptación}}                                                                                                                                                                                                  \\
    \hline
    \endfirsthead
    \hline
    \endhead
    \hline
    \multicolumn{2}{|c|}{{Continúa en la siguiente página}}                                                                                                                                                                                              \\
    \hline
    \endfoot
    \hline
    \endlastfoot
    \multicolumn{1}{|p{6.7cm}|}{\textbf{Número} 15} & \multicolumn{1}{|p{6.7cm}|}{\textbf{Historia de usuario:} 15}                                                                                                                                      \\
    \hline
    \multicolumn{2}{|p{13.4cm}|}{\textbf{Nombre de la historia:} Visualizar alertas de emergencia de ciudadanos}                                                                                                                                         \\
    \hline
    \multicolumn{2}{|p{13.4cm}|}{\textbf{Descripción:} Como policía, quiero visualizar mediante un mapa las alertas de emergencia enviadas por los ciudadanos así como su posición en tiempo real y el tipo de incidente}                                \\
    \hline
    \multicolumn{2}{|p{13.4cm}|}{\textbf{Condiciones de ejecución:} Ser un policía autenticado en el sistema}                                                                                                                                            \\
    \hline
    \multicolumn{2}{|p{13.4cm}|}{\textbf{Entrada/pasos de ejecución:} El policía accede a la pantalla principal de la aplicación.}                                                                                                                       \\
    \hline
    \multicolumn{2}{|p{13.4cm}|}{\textbf{Resultado esperado:} Dado que el policía ingrese a la aplicación, el sistema deberá mostrar las alertas de emergencia enviadas por los ciudadanos,  así como la ubicación en tiempo real en el mapa de alertas} \\
    \hline
    \multicolumn{2}{|p{13.4cm}|}{\textbf{Evaluación de prueba:} Satisfactoria}                                                                                                                                                                           \\
    \hline
\end{longtable}


\begin{longtable}{|p{6.7cm}|p{6.7cm}|}
    \caption{Prueba de aceptación 16: Cambiar contraseña} \label{tab:prueba-16}                                                                                                                                                                             \\
    \hline
    \multicolumn{2}{|c|}{\textbf{Prueba de aceptación}}                                                                                                                                                                                                     \\
    \hline
    \endfirsthead
    \hline
    \endhead
    \hline
    \multicolumn{2}{|c|}{{Continúa en la siguiente página}}                                                                                                                                                                                                 \\
    \hline
    \endfoot
    \hline
    \endlastfoot
    \multicolumn{1}{|p{6.7cm}|}{\textbf{Número} 16} & \multicolumn{1}{|p{6.7cm}|}{\textbf{Historia de usuario:} 16}                                                                                                                                         \\
    \hline
    \multicolumn{2}{|p{13.4cm}|}{\textbf{Nombre de la historia:} Cambiar contraseña}                                                                                                                                                                        \\
    \hline
    \multicolumn{2}{|p{13.4cm}|}{\textbf{Descripción:} Como ciudadano, quiero cambiar mi contraseña para mejorar la seguridad de mi cuenta}                                                                                                                 \\
    \hline
    \multicolumn{2}{|p{13.4cm}|}{\textbf{Condiciones de ejecución:} Ser un ciudadano autenticado en el sistema}                                                                                                                                             \\
    \hline
    \multicolumn{2}{|p{13.4cm}|}{\textbf{Entrada/pasos de ejecución:} El ciudadano accede a la sección de cambio de contraseña en el menú, ingresa su contraseña actual, la nueva contraseña y confirma el cambio.}                                         \\
    \hline
    \multicolumn{2}{|p{13.4cm}|}{\textbf{Resultado esperado:} Dado que el ciudadano ingrese su contraseña actual correctamente y las nuevas contraseñas coincidan, el sistema deberá actualizar la contraseña del ciudadano y confirmar el cambio exitoso.} \\
    \hline
    \multicolumn{2}{|p{13.4cm}|}{\textbf{Evaluación de prueba:} Satisfactoria}                                                                                                                                                                              \\
    \hline
\end{longtable}


\begin{longtable}{|p{6.7cm}|p{6.7cm}|}
    \caption{Prueba de aceptación 17: Recuperar contraseña} \label{tab:prueba-17}                                                                                                                                                    \\
    \hline
    \multicolumn{2}{|c|}{\textbf{Prueba de aceptación}}                                                                                                                                                                              \\
    \hline
    \endfirsthead
    \hline
    \endhead
    \hline
    \multicolumn{2}{|c|}{{Continúa en la siguiente página}}                                                                                                                                                                          \\
    \hline
    \endfoot
    \hline
    \endlastfoot
    \multicolumn{1}{|p{6.7cm}|}{\textbf{Número} 17} & \multicolumn{1}{|p{6.7cm}|}{\textbf{Historia de usuario:} 17}                                                                                                                  \\
    \hline
    \multicolumn{2}{|p{13.4cm}|}{\textbf{Nombre de la historia:} Recuperar contraseña}                                                                                                                                               \\
    \hline
    \multicolumn{2}{|p{13.4cm}|}{\textbf{Descripción:} Como ciudadano, quiero recuperar mi contraseña en caso de olvidarla}                                                                                                          \\
    \hline
    \multicolumn{2}{|p{13.4cm}|}{\textbf{Condiciones de ejecución:} El ciudadano debe estar registrado en el sistema y tener un correo electrónico válido asociado a su cuenta}                                                      \\
    \hline
    \multicolumn{2}{|p{13.4cm}|}{\textbf{Entrada/pasos de ejecución:}
    \begin{enumerate}
        \item El ciudadano selecciona la opción "Olvidé mi contraseña" en la pantalla de inicio de sesión.
        \item El ciudadano ingresa su dirección de correo electrónico asociada a su cuenta.
        \item El sistema envía un correo electrónico con una clave provisional.
        \item El ciudadano ingresa al sistema con su clave provisional.
        \item El ciudadano cambia su contraseña.
    \end{enumerate}}                                                                                                                                \\
    \hline
    \multicolumn{2}{|p{13.4cm}|}{\textbf{Resultado esperado:} Dado que el ciudadano complete los pasos para recuperar la contraseña, el sistema deberá permitirle establecer una nueva contraseña y notificarle del cambio exitoso.} \\
    \hline
    \multicolumn{2}{|p{13.4cm}|}{\textbf{Evaluación de prueba:} Satisfactoria}                                                                                                                                                       \\
    \hline
\end{longtable}


\begin{longtable}{|p{6.7cm}|p{6.7cm}|}
    \caption{Prueba de aceptación 18: Desarrollar modelo analítico de BI en Power BI} \label{tab:prueba-18}                                                      \\
    \hline
    \multicolumn{2}{|c|}{\textbf{Prueba de aceptación}}                                                                                                          \\
    \hline
    \endfirsthead
    \hline
    \endhead
    \hline
    \multicolumn{2}{|c|}{{Continúa en la siguiente página}}                                                                                                      \\
    \hline
    \endfoot
    \hline
    \endlastfoot
    \multicolumn{1}{|p{6.7cm}|}{\textbf{Número} 18} & \multicolumn{1}{|p{6.7cm}|}{\textbf{Historia de usuario:} 18}                                              \\
    \hline
    \multicolumn{2}{|p{13.4cm}|}{\textbf{Nombre de la historia:} Desarrollar modelo analítico de BI en Power BI}                                                 \\
    \hline
    \multicolumn{2}{|p{13.4cm}|}{\textbf{Descripción:} Como administrador, quiero visualizar informes detallados sobre los incidentes ocurridos usando Power BI} \\
    \hline
    \multicolumn{2}{|p{13.4cm}|}{\textbf{Condiciones de ejecución:} Ser un usuario administrador autenticado en el sistema con acceso a Power BI}                \\
    \hline
    \multicolumn{2}{|p{13.4cm}|}{\textbf{Entrada/pasos de ejecución:}
    \begin{enumerate}
        \item El administrador inicia sesión en Power BI.
        \item El administrador accede al panel de informes de incidentes.
        \item El administrador selecciona el tipo de informe que desea visualizar (Informe general, información georreferenciada).
        \item Power BI genera y muestra el informe detallado con gráficos interactivos.
        \item El administrador realiza clic en las gráficas para obtener información específica.
    \end{enumerate}}                                    \\
    \hline
    \multicolumn{2}{|p{13.4cm}|}{\textbf{Resultado esperado:}
    \begin{itemize}
        \item Power BI debe generar y mostrar informes detallados sobre los incidentes ocurridos con gráficos interactivos.
        \item Al realizar clic en las gráficas, Power BI debe mostrar información relacionada con los datos seleccionados.
    \end{itemize}}                                           \\
    \hline
    \multicolumn{2}{|p{13.4cm}|}{\textbf{Evaluación de prueba:} Satisfactoria}                                                                                   \\
    \hline
\end{longtable}
