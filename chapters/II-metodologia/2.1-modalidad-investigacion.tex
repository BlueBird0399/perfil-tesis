En esta sección se describe la metodología que se empleará en el desarrollo de la presente investigación,
tras haber analizado tanto los antecedentes investigativos como el marco teórico.


\subsection{Modalidad de investigación}
En esta sección se muestran los enfoques a los que se encuentra orientada la investigación propuesta para este proyecto. A continuación se detalla
cada una de ellas.
\bigbreak
\textbf{Investigación cuantitativa}\\

En el presente proyecto se emplea una metodología de investigación cuantitativa. Esta se centra en la recopilación de
datos numéricos y estadísticos que permiten establecer las tendencias de siniestralidad a través del análisis del número
de crímenes. Esta información se utiliza como base fundamental para el diseño de un modelo analítico de BI.

\bigbreak
\textbf{Investigación aplicada}\\
Se emplea una investigación aplicada, ya que el desarrollo del proyecto propuesto utilizará la teoría, métodos,
técnicas, tecnologías y conocimientos adquiridos a lo largo de ciclos académicos previos. Este enfoque permitirá
aplicar de manera práctica y específica el conocimiento acumulado durante la trayectoria educativa.

% \bigbreak
% \textbf{Investigación bibliográfica}\\
% La Investigación es Bibliográfica y Documental porque es necesario recopilar información de
% documentos como artículos académicos, tesis y libros que sirvan de apoyo
% para la contextualización de la propuesta a desarrollar.

% \bigbreak
% \textbf{Investigación de campo}\\
% La Investigación es de Campo ya que la información y características del problema
% serán extraídas mediante el contacto directo en el lugar de los hechos, es decir, el
% Centro de Transferencia y Desarrollo de Tecnologías y la Dirección de Vinculación
% con la Sociedad.

% \bigbreak
% \textbf{Investigación aplicada}\\
% La Investigación es Aplicada debido a que se emplearán conocimientos
% adquiridos a lo largo de la carrera para el desarrollo de la propuesta.