\section{Recomendaciones}

\begin{itemize}
      \item Si el sistema se implementa para su uso en conjunto con la policía, se recomienda realizar una capacitación
            previa a los oficiales de policía, con el fin de que puedan utilizar la herramienta de manera eficiente y
            efectiva, así como una concienciación a los ciudadanos para evitar el mal uso de la herramienta.

      \item Para mejorar la calidad de la información así como reducir las falsas alertas se recomienda la integración
            de tecnologías para streaming de video, audio e imágenes, y con ello poder  de realizar un análisis en
            tiempo real de las emergencias.

      \item Dados el tipo de datos que se manejan y la cantidad de información que se debe procesar, se recomienda
            utilizar servicios de computación en la nube como azure analysis services, con el fin de mejorar el
            rendimiento y la escalabilidad de la solución del cubo OLAP.

      \item Cuando la cantidad de datos incremente de forma significativa, se recomienda realizar un análisis para la
            incorporación de técnicas de inteligencia artificial, con el fin de realizar predicciones y análisis más
            avanzados.
\end{itemize}