\sectiontitle{resumen ejecutivo}
\addcontentsline{toc}{chapter}{\bfseries\uppercase{resumen \uppercase{ejecutivo}}}
La ciudad de Ambato enfrenta desafíos significativos en la recolección y análisis de datos sobre
delitos y siniestros, lo cual impacta negativamente la toma de decisiones en materia de seguridad
pública. Los métodos tradicionales, basados en registros manuales y sistemas fragmentados, resultan
ineficientes y propensos a errores, afectando la precisión y oportunidad en el seguimiento de
tendencias delictivas.
\bigbreak

El presente proyecto de investigación propone el desarrollo de un sistema que combina una aplicación
móvil y una plataforma web, con el objetivo de optimizar la obtención y análisis de datos georreferenciados
sobre delitos y crear un modelo analítico de Business Intelligence (BI). El sistema busca mejorar
la eficiencia operativa, reducir los errores humanos y proporcionar una herramienta que facilite la
administración y el acceso a la información tanto para las autoridades como para la ciudadanía.
\bigbreak

El proyecto se desarrolló utilizando la metodología de desarrollo ágil XP (eXtreme Programming), lo
que permitió una mayor flexibilidad y adaptabilidad a los cambios durante el proceso de desarrollo.
Este enfoque facilitó iteraciones rápidas y la incorporación de retroalimentación continua, asegurando
que el producto final cumpliera con los requisitos a medida que se avanzaba en el desarrollo del sistema.
\bigbreak

El proyecto se divide en dos aplicaciones principales: web y móvil. Para el desarrollo de la aplicación
móvil se utilizó Flutter, debido a su rendimiento multiplataforma. Para la aplicación web se empleó
NextJS y NestJS, tecnologías que permiten la creación de aplicaciones web robustas y escalables.
Adicionalmente, se empleó una base de datos geoespacial de PostgreSQL para gestionar eficientemente
grandes volúmenes  de datos georreferenciados y desarrollar un modelo analítico de BI que permita
visualizar y analizar tendencias de criminalidad.
\vfill
\textbf{Palabras clave:} Rappid Application Development, Sistema web, aplicación móvil, Business Intelligence,
georreferenciación, seguridad pública, Flutter, NextJS, NestJS
