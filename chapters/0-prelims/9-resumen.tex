\sectiontitle{resumen ejecutivo}
\addcontentsline{toc}{chapter}{\bfseries\uppercase{resumen \uppercase{ejecutivo}}}
% Las Unidades de Producción de la Universidad Técnica de Ambato son entidades
% autogestionadas que generan ingresos a partir de la oferta de productos y servicios,
% representando un beneficio significativo para la universidad. Sin embargo, enfrentan desafíos en la
% gestión de la información y carecen de
% un medio que les permita compartir información con la dirección financiera, encargada de
% recibir los pagos, para su visualización y verificación, lo cual supone un problema a la
% hora de completar el proceso de entrega de productos y servicios.
% \bigbreak
% El presente proyecto propone la implementación de una aplicación web que establezca un
% proceso en línea para adquirir y procesar los pagos de servicios y productos,
% permitiendo a los usuarios adjuntar y verificar información relativa a los
% procesos que les corresponden. Además, se busca agilizar el registro y
% validación de datos mediante el reconocimiento óptico de caracteres (OCR) y
% procesamiento de imágenes.
% \bigbreak
% La aplicación web se divide en dos componentes principales: backend y frontend. En el
% backend, se utiliza Python con FastAPI para el desarrollo de la interfaz de programación de aplicaciones
% (API), integrando OCR con Pytesseract y OpenCV.
% Adicionalmente se empleó PostgreSQL como base de datos y MinIO para gestionar imágenes y archivos.
% En el frontend, se emplea React y bibliotecas adicionales.
% \bigbreak
% Finalmente, como resulatado se obtuvo una aplicación web que minimiza los incovenientes del proceso
% actual y que, adicionalmente, descarta el uso de documentación física evitando así posibles pérdidas de
% información que demoren la culminación del proceso.
% \vfill
% \textbf{Palabras clave:} Rappid Application Development, OCR, Tesseract, procesamiento de imagenes, aplicación web
